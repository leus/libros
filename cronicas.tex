\documentclass[10pt,twoside,openright]{memoir}
\usepackage[utf8]{inputenc}
\usepackage{verse}
\usepackage[spanish]{babel}
\usepackage{ebgaramond}
\usepackage[protrusion=true,expansion=true]{microtype}

\usepackage[paperwidth=5in,paperheight=8in,bindingoffset=.75in]{geometry}
\usepackage{tgtermes}


\setlength\epigraphrule{0pt}

\title{Crónicas Portalianas}
\author{Enrique Bunster}

\makeatletter
\def\maketitle{%
	\null
	\thispagestyle{empty}%
	\vfill
	\begin{center}\leavevmode
		\normalfont
		{\LARGE\raggedleft \@author\par}%
		\hrulefill\par
		{\huge\raggedright \@title\par}%
		\vskip 1cm
		%    {\Large \@date\par}%
	\end{center}%
	\vfill
	\null
	\cleardoublepage
}
\makeatother

\begin{document}

\let\cleardoublepage\clearpage

\maketitle

\frontmatter
\null\vfill
\begin{flushleft}
	\textit{CRÓNICAS PORTALIANAS}
	
	Es propiedad.
	
	Derechos Reservados
	
	para todos los países
	
	© por Editorial Del Pacífico S.A.

	Edición de 3000 ejemplares.
	
	Alonso Ovalle 766,
	
	Santiago de Chile	

	\bigskip
	
	
	Impreso en Chile.
	
	TALLERES GRÁFICOS CORPORACIÓN LTDA.
	
	Alonso Ovalle 748.
	
	
	
	
\end{flushleft}
\let\cleardoublepage\clearpage

\mainmatter
\sloppy


	
Dedico esta obra a mi esposa, Carmen:

\begin{verse}
Gracias por su poesía,\\
gracias por sus ojos,\\
por su voz,\\
gracias por su gracia\\
y por el armiño de alegría\\
que ha colocado sobre mis hombros.
\end{verse}

\chapter{LIRCAY, ¡QUE BATALLA!}

Advirtamos, de entrada, que ninguna acción militar de nuestra historia
---exceptuada la de Maipú--- reunió a tan ilustre elenco de celebridades
como el terrible y olvidado encuentro de Lircay. Combatieron allí tres
gobernantes: el ex Director Supremo Ramón Freire y los futuros
presidentes Joaquín Prieto y Manuel Bulnes, el noble inglés George de
Vic Tupper y dos antiguos oficiales de Napoleón y veteranos de Waterloo:
Benjamín Viel y José Rondizzoni. Historiadores y estrategos afirman que
Prieto dio en esos aledaños de Talca la más impecable y científica
batalla de los fastos nacionales. Y todavía hay que añadir que Lircay,
por sus efectos y consecuencias, se cuenta entre los hechos de armas
decisivos del destino de Chile. Maipú consolida la independencia, Yungay
salva a chilenos y peruanos de la dominación de un César boliviano,
Iquique deja ganada la guerra del Pacífico, Lircay (17 de abril de 1830)
da comienzo al régimen portaliano que organizará la República y dejará
aplastada la anarquía hasta mil novecientos setenta.

El desmoronamiento del orden se produjo casi al día siguiente de la
abdicación de O'Higgins. Este hombre superdotado había sido capaz de
gobernar seis años sin tambalearse en su puesto. Después de él
sobreviene el carrusel político de pipiolos, carrerinos, pelucones,
o'higginistas, populacheros, federalistas, pandillistas, estanqueros,
unitarios y neutros; se desata el caudillismo, enfermedad pegajosa de la
América española, y se suceden las Juntas de Gobierno, los cuartelazos y
la seguidilla de gobernantes que no acababan de acomodarse en su sillón
cuando tenían que abandonarlo. Todavía no partía O'Higgins para el Perú
y ya la guarnición de Tucapel había asesinado a su jefe para unirse a la
banda de los Pincheira. En un asalto a Linares estos facinerosos
reforzados robaron y degollaron a voluntad y raptaron a cuantas mujeres
quisieron llevarse consigo. Las tropas enviadas en su persecución se
entendieron con ellos para formar una montonera que sembró el terror y
la ruina en los campos. La alarma producida en Santiago facilitó el
ascenso del general Freire al poder; pero el que tanto censuró al Padre
de la Patria y pretendió emularlo y eclipsarlo, cayó al cabo de tres
años de vacilante administración liberal-pelucona. Su sucesor, Blanco
Encalada, resistió cinco meses. las obstrucciones del propio Parlamento
que lo ungiera Presidente de la República. Su reemplazante, el
vicepresidente Eyzaguirre, sin dinero para pagar a las tropas ni a los
funcionarios, fue derribado por el amotinado coronel Campino, que entró
al Congreso a caballo. Reelegido Freire como transacción, duró un año,
siendo sucedido por el pipiolo Pinto, que aguantaría dos. Llegó un
momento en que nadie entendía a nadie y la desmoralización cundía a la
par con el bandidaje y la criminalidad. El ejército de los Pincheira
tuvo hasta mil mujeres cautivas. En un año hubo en Santiago ochocientos
asesinatos. Eran corrientes los negociados y los robos al Fisco y la
basura de las calles se amontonaba hasta frente a las puertas de la
Catedral. Al sur del Maule imperaba el Ralo, fiera humana que hizo
ochenta y siete homicidios y culminó matando a su padre. Hubo un motín
militar en Talca y luego otro, en San Fernando, cuyo caudillo, el
coronel Urriola, derrotó a las fuerzas gobiernistas en Ochagavía y entró
a la capital a tambor batiente. Ya en plena chacota, vencedores y
vencidos fraternizaron en los comedores del café La Nación, mientras
Pinto continuaba en el Palacio sin ser molestado e indeciso entre irse y
quedarse. De ahí en adelante los hechos toman el carácter de una
pelotera lisa y llana. Una fracción de los oficiales de Urriola se
arrepintió de su actitud subversiva y solicitó el perdón del Presidente.
Desinflada la revolución, el cabecilla y sus secuaces quedaron impunes.
Entretanto se promulgaba la Constitución de 1828, y dentro de sus
términos realizáronse las elecciones generales y presidenciales. Fue un
torneo de fraudes y crímenes a bala y cuchillo estimulado por las
campañas de insultos de los periódicos El Hambriento y El Canalla.
Hastiado del poder y sus amarguras, Pinto pasó por alto Su reelección y
renunció una y otra 'vez hasta conseguir que se diese el mando
provisional al anciano don Francisco Ramón Vicuña. Demasiado tarde,
porque ya el general Joaquín Prieto había perdido la paciencia y se
disponía a enderezar el país \emph{numu militariilitari}. Detrás de él, todavía en
relativo anonimato, estaba don Diego Portales, que había manifestado en
su tertulia del escaño de piedra de la Alameda: ``No podemos continuar
así''. En afortunada maniobra inicial el coronel Bulnes sublevó a favor
de Prieto las guarniciones desde el Bío-Bío al Maule y las concentró en
Rancagua. Con la guerra civil ad-portas, el asustado Vicuña buscó una
salida llamando a elecciones presidenciales. Pero nadie cedía ni quería
oír a nadie. A consecuencia del tumulto y las rechiflas de la asamblea
reunida en el Consulado, el Presidente se retiró a Palacio. Los
asambleístas se fueron tras él, desarmaron a la guardia e invadieron en
choclón la sede del Gobierno. Entonces el viejecillo acorralado y
desesperado empezó a hacer cosas lindantes con la locura. Temiendo le
arrebatasen la banda, se la sacó y la metió en su sombrero; luego pidió
a gritos que llamaran a Freire, tal vez para implorarle socorro.
Creyendo que quería deponer el mando en su favor, los pipiolos y los
oportunistas hicieron entrar al general en apoteosis, lo obligaron a
sentarse en el sillón presidencial y le terciaron la banda que alguien
acababa de sacar como un prestidigitador del colero de Vicuña. Viendo
que sus parciales habían desaparecido, Vicuña escapó a la calle con su
hijo, la única persona que le fue leal.

Al día siguiente Chile amaneció con dos gobiernos: el legítimo, que
todavía no renunciaba, y el de una Junta formada por Freire, Alcalde y
Ruiz-Tagle. Inmediatamente se sublevó contra la Junta el francés
Benjamín Viel, plegándose al ejército constitucionalista mandado por
Lastra. En Talcahuano, el inglés De Vic Tupper intentó apoderarse del
bergantín de guerra Aquiles, siendo rechazado a cañonazos por su
comandante Angulo.

Mientras esto ocurría, las huestes de Prieto y Bulnes se estacionaban en
un lugar al S. O. de Santiago. Hasta ahí llegó Portales en su birlocho
llevando fondos para pagar a las tropas. De improviso los dos ejércitos
se encontraron cerca del camino a San Bernardo. Después de un combate
favorable a Freire se acordó la tregua para buscar una última solución
pacífica. De aquí surgió el insólito arreglo de que
ambas fuerzas se colocasen bajo el mando de Freire mientras se elegía
una Junta provisional. El triunvirato Ovalle-Errázuriz-Trujillo procedió
a romper con Freire y dispersar la división de Lastra como medida previa
para convocar a los plenipotenciarios de provincias. Esta reunión eligió
un gobierno interino presidido por el pelucón Francisco Ruiz-Tagle y
el vice José Tomás Ovalle. Fórmula que duró cuarenta y dos días, con el
presbítero Meneses como Ministro del Interior y de Guerra y Marina y en
medio de un colosal desbarajuste, hasta que Portales gritó ``¡Basta!''
haciendo renunciar a su primo Ruiz-Tagle para entregar la Presidencia a
Ovalle. Allí acabaron la zarabanda de
partidos, la ineptitud y el aprendizaje republicano. Fue como una cuerda
que se corta. Ya nadie se atrevía a ocupar un Ministerio. Entonces fue
cuando don Diego Portales declaró que él estaba dispuesto a aceptar
cualquier nombramiento, ``hasta el de Ministro Salteador''. Y en un
decreto que la posteridad juzga providencial, el Presidente José Tomás
Ovalle le confió las carteras del Interior. de Relaciones Exteriores y
de Guerra y Marina.

El avispero político se aquietó como por obra de milagro. Cansados todos
de la lucha estéril, conscientes al fin de su inexperiencia y temerosos
de una catástrofe irreparable, pusieron su confianza en el
hombre distinto, enigmático, sin
bandera definida y enemigo de programas y discursos. Los historiadores
dicen que ``se echaron en sus brazos''. del mismo modo que un paciente se
entrega al cirujano después de probar inútilmente cataplasmas y
toronjiles.

Hasta ayer, Portales era el discutido personaje que intentó servir la
Deuda Externa con la renta del estanco del tabaco y los naipes; ahora
encarnaba la esperanza de un gobierno estable y salvador. ¿Por qué?
Sencillamente porque ya no quedaba nadie más a quien probar, y porque su
misterioso poder de sugestión estaba operando.

Lo primero de todo, para el triple Ministro, era acabar de una vez con
Freire, que persistía en el ensueño
de creerse insustituible y preparaba una nueva asonada. El nunca
escarmentado general había desembarcado en Constitución, después de una
fallida campaña sobre Coquimbo, y tenía sus fuerzas reunidas con las de
los napoleónicos Viel y Rondizzoni. Mandaba ahora un ejército de mil
ochocientos hombres y cuatro cañones
arrastrados por bueyes.

Portales ordenó al general Joaquín Prieto, Intendente de Concepción,
echársele encima con todo el potencial disponible. En pocos días el
coronel De la Cruz, jefe de Estado Mayor, y el coronel Bulnes; jefe de
la caballería, tuvieron listos dos mil doscientos hombres y doce cañones
tirados por mulas y caballos, que concentraron al norte de Talca. Estos
tres veteranos de la Independencia pertenecían a esa categoría
 de militares desinteresados,
patriotas y decididos a los que O'Higgins había enseñado para qué se
lleva el uniforme.

Difícil hubiera sido, en esas vísperas, anticipar el desenlace de la
contienda. Sólo era posible predecir que uno de los bandos iba a
aniquilar al Otro. porque nada iguala al odio que arde en un
enfrentamiento intestino decisivo. Encina dice que los soldados de ambos
ejércitos se detestaban más que realistas y patriotas y que en las filas
de Prieto corría la consigna de ``no dejar gringo vivo'', refiriéndose a
Viel, Rondizzoni y Tupper. Prieto, de cuarenta y cuatro años, era un esbelto 
varón de piel clara y ojos  azules, cuyo aire
apacible en nada delataba al militar aguerrido. Llevaba ventaja numérica
en hombres y en artillería, pero Freire contaba con una infantería
más homogénea y mejor disciplinada y
con el consejo de dos oficiales fogueados en Austerlitz y Waterloo.
Todos estos hombres, ahora dispuestos a destrozarse, eran antiguos
amigos y camaradas que sirvieron juntos a las órdenes de San Martín; y
Prieto había sido subalterno de Freire en las campañas del Sur, de
manera que conocía sus cualidades y defectos y podía casi adivinarle el
pensamiento.

La noche del 14 al 15 de abril Freire cruzó el Maule de sur a norte,
utilizando la balsa del camino público, y la mañana siguiente le
encontró acampado en Talca. Los vecinos, sabedores de la proximidad del
contendor, atrancaron sus puertas y comercios para esperar con el
resuello contenido lo que estaba por suceder. Sabían lo que era eso,
porque doce años atrás habían escuchado el pandemónium de la sorpresa
nocturna en la cercana Cancha Rayada.


Habiendo observado desde lejos la presencia de Freire en la ciudad,
Prieto avanzó con sus tropas hasta situarse en las faldas del cerro de
Baeza, a una legua hacia el nororiente de los suburbios, para provocarlo
a combatir. Como no lo consiguiera, durante la noche llevó a cabo su
segunda jugada en el tablero, retrocediendo sigilosamente hasta el río
Lircay para precaverse de un ataque sorpresivo.

Algo hizo a Rondizzoni prever o adivinar esta maniobra en tinieblas, y
como consecuencia. los batallones de Freire amanecieron trasladados
quinientos metros hacia el río, tendidos en línea de combate y
protegidos por la topografía del terreno.


Hasta aquí, por sus previsiones y movimientos, parecía que el comando de
Freire era el mejor. Pero Prieto recién empezaba a desplegar su ciencia
de ajedrecista. ¿Hubiera creído nadie que iba a abandonar sus posiciones
para dirigirse hacia el sur alejándose del enemigo? Pues esto fue lo que
hizo, y todavía, dejando distanciadas la caballería y parte de la
artillería como en actitud de proteger su marcha.
Así desfiló a través del llano que
corre entre el cerro de Baeza y Cancha Rayada. Engañado por las
apariencias, Freire juzgó que su contrincante se retiraba hacia
Concepción; pero el movimiento sólo tenía por objeto interponérsele para
impedir su repliegue hacia Talca; y cuando cayó en la cuenta, ya era
tarde. Ahora era Prieto el que tenía a sus espaldas la ciudad, donde
podría apoyarse cómodamente y dando frente a la llanura, campo ideal
para un ejército más fuerte en
artillería, aunque el terreno cruzado de cercos y zanjas favorecía en
algo al que cifraba sus esperanzas en la infantería.

Pero la habilidad de Prieto no paró
allí. En repentina carrera las fuerzas que dejara
distanciadas se desplazaron para ir a
reunírsele en los extramuros de Talca, donde ya los infantes estaban
parapetados en las últimas tapias y casas del lado norponiente. Doble
ventaja, porque a medida que transcurriesen las horas Freire iba a tener
en su contra el factor de combatir de cara al
sol.

Con sus sucesivos movimientos, Prieto había sacado a Freire de la ciudad
sin disparar un tiro, para ocupar su lugar y dejarle donde él estuvo
antes: en campo abierto. Era como sí ambos ejércitos hubieran girado
sobre un eje equidistante invirtiendo sus posiciones.

Colocado donde a él convenía, Prieto dio de comer a su tropa, y luego,
calmosamente, ordenó romper el fuego sobre una vanguardia enemiga
situada a dos cuadras de distancia.
Freire contestó sin vacilar, creyendo erróneamente que combatían en
igualdad de condiciones. A los pocos minutos vino a darse cuenta del
infierno en que había ido a meterse. En medio del tronar de los cañones
la batalla de cuatro mil hombres se desató a las puertas de la ciudad
empavorecida. Atacada de frente por las baterías y desde los flancos por
la fusilería, la línea de Freire comenzó a ceder. De ordinario los
historiadores describen los hechos de armas como si fuesen partidos de
pelota; y una batalla no es eso. Es una matanza épica entre dos
muchedumbres uniformadas de vistosos colores y guiadas por toques de
corneta y banderas tremolantes; es una tormenta de cañonazos que
ensordecen el ámbito y se multiplican en ecos lejanos; un aguacero de
proyectiles que detonan y silban en concierto horroroso; griterío de
furor homicida y de agonía entremedio del humo y los relinchos de los
caballos que caen despanzurrados o se desbocan tirando patadas y
pisoteando a los heridos; olor a pólvora, a sudor, a bosta y a sangre;
cureñas y pircas  demolidas que
saltan por los aires; mutilados que se suicidan entre alaridos de dolor
insoportable. Eso fue la batalla de Lircay, operación sin anestesia para
extirpar el tumor maligno de la
anarquía.

Viendo Freire que su línea se desintegraba, ordenó replegarla al norte,
hasta cerca del río Lircay, donde Prieto había estado acampado en la
mañana. Pudo hacerlo sin grandes pérdidas, ayudado por los obstáculos
que demoraron el avance de la caballería y cañones enemigos. Pero su
suerte estaba echada. Como Rondizzoni le hiciera notar que las nuevas
 posiciones no eran mejores que las
anteriores, ni para combatir ni para retirarse, Freire pronunció esta
frase lapidaria: ``Pues, coronel, aquí tenemos que echar
el resto''. Sabiendo que no había más
alternativas, mandó lanzar la caballería del Pudeto en masa contra la
infantería perseguidora antes de que los cañones acudiesen a protegerla.
Pero no contó con la intervención de Bulnes, al que Prieto dictó la
maniobra justa para desbaratar esa carga y trocarla en
 desastre. Haciendo que se
interpusiera con sus jinetes del Carampangue y simulando luego que se
replegaba, consiguió que Rondizzoni le siguiese, hasta alejarle
suficientemente de la infantería de Prieto; y entonces presentó combate,
reforzado con un escuadrón de reserva de los Cazadores de Maipo, y en
menos de diez minutos de refriega deshizo a sablazos a la perpleja y
revuelta caballería atacante. Rondizzoni cayó herido. Viel huyó vadeando
el Lircay como pudo y Freire le siguió a mata caballo, dejando el campo
sembrado de bajas, armamento y bártulos abandonados y pastizales en
llamas.

Pero la veloz caballería de Bulnes se había adelantado para cortar el
paso al resto de los fugitivos, a la
vez que Prieto y De la Cruz los cercaban por la retaguardia y los
flancos; y la lucha, reanudada en las cercanías del Lircay, se prolongó
por dos horas con exacerbado furor. Una bala de fusil mató al coronel
Elizalde, que sustituía a Freire en el mando, y quince de sus oficiales
sucumbieron peleando revueltos con los soldados. De Vic Tupper, a punto
de escapar, fue rodeado, despedazado a bayoneta y luego ultimado a tiros
de pistola. Dejó viuda a doña Isidora Zegers, la mujer más brillante de
su tiempo. Otro inglés, el marino Robert Bell, murió a filo de sables.
Así pagaron su intromisión en la política interna del país como secuaces
de dos guerras fratricidas consecutivas.


Apenas doscientos hombres se salvaron
del exterminio aprovechando el indescriptible tumulto final; pero antes
de ponerse el sol habían caído prisioneros de los cazadores de Bulnes.

La jornada costaba a Prieto un centenar de vidas, pero Freire perdía
cuatrocientas, cerca de mil heridos y el resto apresado o disperso. De
manera que la batalla de Lircay no dejó nada del ejército vencido; nada
sino su jefe, que escapó vivo por casualidad, y Viel, que de algún modo
llegó hasta Coquimbo para caer en manos del general Aldunate.

Amaneció el nuevo día sin que nadie
se diese cuenta de que despuntaba otra era de la historia nacional. El
pasado quedaba sepultado y el porvenir se abría como un paisaje infinito
con las riendas del Gobierno en las manos de Diego Portales, el patricio
de treinta y seis años y sin antecedentes de estadista que iba a
convertir a Chile en la primera república ordenada del continente.

Cuando el parte de la victoria llegó a Santiago, conducido por un correo
expreso, cundió un júbilo que alcanzaría hasta los arrabales; tal eran
el cansancio y el disgusto que habían dejado siete años de desorden. En
vano quiso el Gobierno amortiguar el impacto emocional de la noticia en
aras de la concordia cívica. Salieron a la calle improvisadas pobladas
de manifestantes, y por la noche se elevaron centellas y cohetes. Y esto
sucedía en una época en que la masa popular aún no participaba ni se
interesaba en el quehacer político.

La breve y tajante campaña militar había pacificado de paso a Valparaíso
y Coquimbo y ahora dejaba sometidas a Concepción, Valdivia y Chiloé, los
últimos focos subversivos. Todos los derrotados de Lircay fueron dados
de baja en las filas. No hubo represalias partidarias y sólo unos pocos
sospechosos quedaron bajo vigilancia policial. Algo decía a la gente que
una cosa nueva estaba tomando forma. En sólo doce días de ejercicio del
poder, Su Señoría había sosegado por igual a moros y cristianos. De la
guerrilla de partidos se pasaba a la conducción despersonalizada del
Estado; de la ambición egoísta al servicio de los intereses supremos;
del caos a la autoridad inflexible. Prieto, magistral vencedor, rehuía
los honores y volvía sin ruido a ocupar su cargo de Intendente de
Concepción.

Ahora la patria era como una silla a la sombra de un árbol.

Tan cierto es que en el peor momento el organismo social sabe encontrar
al hombre preciso, extrayéndolo de sus últimas reservas, para conjurar
el mal y corregir el rumbo. Hasta podría decirse que, a no mediar esa
larga noche de política aldeana y sin brújula, acaso Portales no habría
dejado el comercio por el gobierno, y el país, dentro de lo probable,
jamás hubiera alcanzado la disciplina y la marcha tranquila y rectilínea
que él había comenzado a imprimirle.

Sólo un hombre, don Ramón Freire, no logró entender el cambio profundo
que se operaba. Aunque se le trató
con respeto, como a exgobernante, y se le ofreció una salida decorosa al
extranjero, marchóse a Aconcagua con el loco plan de reclutar soldados
para volver a la capital a sublevar la guarnición. No hubo más remedio
que apresarlo, borrarlo del escalafón militar y deportarlo al Perú.

\chapter{PORTALES GOBERNANDO}

Como O'Higgins, Montt, Balmaceda o cualquier estadista de alto vuelo, don
Diego Portales no logró poner de acuerdo a sus críticos contemporáneos.
El general Bulnes le llamó ``sabio y digno Ministro, cuyos heroicos y
patrióticos esfuerzos han contribuido tanto al lustre de que goza la
República''. Don Manual José Gandarillas le consideraba ``un loco, un
quemado''. Don Joaquín Tocornal se preguntó al borde de su tumba: ``¿Quién
ha hecho el bien de un modo más gratuito y más completamente
desinteresado?''. Don José Antonio Rodríguez Aldea le calificó de ``falso,
inconsecuente, voluntarioso y de odics implacables''. Su mortal enemigo,
el mariscal boliviano Andrés de Santa Cruz, confesó: ``Siempre tuve de él
un alto concepto''. 

De igual manera discrepan los historiadores. Para Vicuña Mackenna
``Portales es la más alta figura de
nuestra historia''. El consabido José Victorino Lastarria escribió: ``Un
pillo de los que tiene nuestra tierra a puñados''. Don Francisco Antonio Encina
dice que ``por el vigor de su pensamiento político es uno de los cerebros más
poderosos entre los que han gobernado pueblos'', y juzga así su tránsito a la
inmortalidad: ``Aún no se inhumaban sus restos, empezó su transfiguración en símbolo 
de la unidad del  alma nacional y de una nueva conciencia cívica''.

Pero esta disparidad violenta de opiniones no hace sino acrecentar a los
ojos del estudioso el interés que irradian la personalidad y la obra del
prócer. Lo que sí produce desaliento,
y hasta miedo, es que el común de los chilenos le ignore hoy casi por
completo. ¿A dónde va un pueblo que sepultó en el olvido al constructor
de la República, a quien hizo de Chile
la primera nación verdaderamente
civilizada de América del Sur? El
obelisco que recuerda su martirio en Valparaíso es de una
insignificancia que da vergüenza. Ningún regimiento ni buque de guerra
lleva el nombre del que cubrió de laureles a la Marina y el Ejército. En
la capital se le recuerda en una deprimente calleja de barrio bajo. Es
común oírle llamar pelucón y dictador, justo lo que nunca fue. Y el
broche de oro se lo debemos al ilustre gobierno del señor Salvador
Allende, que permitió que al liceo Diego Portales le cambiaran su nombre
por el de Ernesto Che Guevara.

Conté en otro lugar que cuando cl Presidente Ovalle le confió tres de
las cuatro carteras del Gabinete, Portales era un joven de treinta y
seis años. Hay que repetir también, para sopesar el misterio de su
carrera asombrosa, que carecía de toda experiencia política; y
agreguemos que su preparación básica escolar fue inferior al bajo
término medio de la época. Hijo de un padre que engendró veintitrés
niños, el pequeño Diego Josef Pedro Víctor Portales y Palazuelos quedó
fuera del presupuesto familiar de educación y su madre hubo de enseñarle
las primeras letras; ``y no •tuve otra escuela'', recordaría más tarde el
autodidacto.

Viudo de su prima Josefa Portales y Larraín, a
cuyo lado conociera ``una dicha
infinita'', llevó por ella luto vitalicio; pero la soledad en plena
juventud y su temperamento irrefrenable hicieron de él un vividor
alegre, astro de la zamacueca en las chinganas de la calle de Duarte
(hoy Lord Cochrane) y certero galán de amores fugaces. Pero ningún
afecto consiguió llenar el vacío que en su corazón había dejado la
esposa, y de esta frustración sentimental irremediable surgió como
sucedáneo el sentimiento avasallador de su vida: el amor a la patria.
Nadie, tal vez, ha amado a la tierra natal como este chileno acusado de
insensible e indiferente. Decía que Chile era la perla del Nuevo Mundo,
el país privilegiado y llamado a constituirse en modelo y ejemplo de sus
vecinos; sostenía que en su mar no debía dispararse un cañonazo que no
fuera de saludo a la estrella de su pabellón. Por eso ardió de coraje
cuando lo vio hundido en la anarquía, desquiciado por caudillos
incapaces y rodando por 'la
pendiente, de la ruina. Y ahora comprendemos que aquellas vicisitudes
fueron el precio que la naciente república tuvo que pagar para que el
estadista saliera a la luz y empuñara el timón del buque al garete.

Nada tenía que ver su figura con el hércules de bronce de la plaza de la
Constitución. Era de porte mediano, delgado, de ojos claros y finas
manos de señorito. Tampoco hay que
fiar de sus retratos, basados todos en el de Domeniconi, cuyo boceto se
hizo después del asesinato de El Barón sobre su cadáver desangrado y
desfigurado por las heridas. Influido por dicho cuadro, Vicuña Mackenna
le atribuía una piel pálida, cuando su carácter y temperamento inducen a
pensar que la tenía sanguínea y exuberante. Y nada de su peculiar manera
de ser captaron tampoco el pintor ni el escultor. Era de índole pícara,
risueño entre los amigos, dicharachero, bromista, zumbón y mal hablado:
la antítesis del pavo real que ha solido campear en nuestra arena
política. De su persona, sin embargo, emanaba una misteriosa corriente
de sugestión; fenómeno observado por Zapiola, quien refiere que en la
famosa tertulia del escaño de piedra de la Alameda los oyentes imitaban
inconscientemente sus gestos y
posturas.

Entró al Gobierno sin programa, promesas ni compromisos con partido
alguno. Y, lo que es más, sin la menor ambición de gloria o provecho. Su
ideario, por llamarlo así, es
brevísimo y se le extrae con pinzas de sus cartas y de sus artículos de
prensa. No de sus discursos, porque
no discurseaba. Apenas son cinco o seis conceptos fundamentales,
descollando el que preconizaba ``un gobierno fuerte, centralizado, cuyos
hombres sean verdaderos modelos de virtud y
patriotismo, y así enderezar a los
ciudadanos por el camino del orden y de las virtudes. Cuando se hayan
moralizado, venga el gobierno completamente liberal; libre y lleno de
ideales, donde tengan parte todos''. Su rígido sentido de la igualdad de
derechos le hacía clamar porque la justicia se ejerciera sin distingos
de posición social ni de fortuna. No creía en el destino
agrícola del país, y acuñó esta frase visionaria cuyo mandato está aún por cumplirse: ``Los chilenos tendrán que ser un pueblo comerciante y marinero''. 
En el plano internacional, ``cuidado con
salir de una dominación para caer en otra''. Preveía la conquista de la
América Latina, ``no por las armas sino por la influencia en toda
esfera. Esto sucederá, tal vez hoy no, pero mañana sí\ldots''

Y nada de teorías políticas, económicas ni sociológicas; nada copiado de
otros países, recurso fácil y funesto en que caen el estadista mediocre
y el demagogo. Como un nuevo Quijote, estaba en el poder para enderezar
entuertos, corregir vicios y probar su devoción a la Dulcinea encarnada
en la patria, mediante una serie de mil pequeñas medidas de buen sentido
y desprovistas de espectacularidad, que llevaría a cabo con laboriosidad
de hormiga y dureza de taladro.

Para consagrarse a esta misión debió descuidar sus no muy prósperos
negocios: un fundito pedregoso y la pequeña goleta con que fletaba carga
surtida a lo largo de la costa.

Un resentido lo acusó de pillo\ldots, y la historia dice que por norma no
cobraba sus sueldos de Ministro.

Lo llamaban dictador\ldots, y sostenía que no debe contrariarse la opinión
responsable; ``antes que doblegarla hay que hacer lo posible por
orientarla''. De acuerdo con este principio, su primera actitud fue
oponerse al endurecimiento de la Ley de Imprenta, defendiendo el derecho
que tenía la oposición a fiscalizar y criticar libremente al Gobierno y
aportar ideas constructivas. Haciendo uso de esa libertad apareció El
Defensor de los Militares, y poco después
El Trompeta, periódicos que se
lanzaron a atacar sin piedad ni decoro al Ministro ignorando el
altruismo que Su Señoría deseaba inculcar. Sólo vino a castigarse a sus
editores con la prisión o el destierro cuando llegaron a incitar a la
revuelta y al asesinato de los
gobernantes.

Califican a Portales de pelucón\ldots, y lo que hizo fue tener a raya a
pelucones, pipiolos y demás como partidos, mientras llamaba a servir en
la Administración a los hombres selectos que había en sus filas. Como en
una nueva Inglaterra, dejó a los opositores libre acceso al Gobierno,
cosa inconcebible en la América de
mandones absolutistas de esos años y que sus propios -compatriotas no
fueron capaces de comprender. El mejor ejemplo de esta política se
encuentra en la Corte Suprema, donde el anti portaliano Carlos Rodríguez
y otros pipiolos, que eran mayoría, permanecieron en sus cargos de
ministros de justicia por orden expresa de Su Señoría.

Otro de sus sencillos principios era que el gobernante debe educar a sus
colaboradores con el ejemplo, y así era el primero en llegar a las
oficinas ministeriales y el último en retirarse. Su abrumadora actividad
no le dejaba más de cinco horas para dormir. Casi todo el fardo
administrativo descansaba sobre sus hombros, porque el Presidente Ovalle
apenas podía firmar el despacho, minada su salud por la tisis que pronto
iba a llevarle al sepulcro.

Sólo un Ministerio no quiso Portales
para sí: el de Hacienda. Y como sostenía que las finanzas del país no
podían confiárseles a un ignorante, buscó con ojo clínico al hombre
capaz de resolver la bancarrota fiscal y la miseria particular. Este
hombre era el liberal don Manuel Rengifo, un comerciante diez veces
quebrado y vuelto a levantar; el mismo cuya efigie pensativa adorna
hasta hoy nuestros billetes de Banco.
Para
hacerle lugar, Portales pidió la renuncia al presbítero Meneses, cuya
salida del Gabinete dejó reducida a
casi nada la influencia de los pelucones. Ante la sorpresa general,
Rengifo se atrevió a disminuir la planta del Ejército y a despedir a
decenas de funcionarios superfluos o incompetentes. Llevó su celo de
economista hasta suprimir el servicio de taquígrafos del Congreso,
``porque sin medidas de coraje y sacrificio no sale un país del pozo de
la ruina''. Decretó que hasta la más ínfima orden de pago del Estado
debía llevar su firma. Con esto se acabaron los fraudes y
malversaciones. Semanalmente daba a la publicidad el balance de la
Tesorería. Hizo devolver a las congregaciones religiosas los predios
expropiados por Freire y cuya administración sólo arrojaba pérdidas al
Estado metido a empresario. Impuso drásticas penas al contrabando y
reorganizó los almacenes de depósito del puerto de Valparaíso, que
pronto convertirían a éste en el emporio mercantil del Pacífico.
Reajustó la economía a las condiciones de la era republicana, fomentando
el intercambio preferencial con los países vecinos, y alentando con ello
a agricultores y mineros a producir más y a exportar. Resultados de su
gestión: el primer presupuesto equilibrado que veían los chilenos desde
1810, el pago puntual de los sueldos de militares y funcionarios y la
reanudación del servicio de la Deuda Externa; todo ello sin recurrir a
nuevos impuestos ni gravámenes.

Mientras Rengifo restauraba la economía, Portales
 se encargaba de pacificar el campo,
donde cuatreros y salteadores habían casi paralizado la agricultura y
tenían a los jueces inhibidos por el miedo a las venganzas. Para
combatirlos se creó la policía mixta, formada por tropa montada y los
propios terratenientes secundados por sus peones e inquilinos. Con
energía inmisericorde fueron perseguidas y deshechas las bandas armadas
ilegales (disfrazadas a veces de
montoneras revolucionarias). En una
batida cerca del río Loncomilla el hacendado Pedro Montecinos alcanzó y
dio muerte de una puñalada a El Ralo, autor de ochenta y siete
asesinatos, incluido el degüello de su
propio padre. Eliminado este
monstruo, el escarmiento hizo desaparecer al resto de la ralea de
antisociales. Sólo quedaban los ``guerrilleros'' de Pincheira, a los que
ya les llegaría su hora. 

 El prematuro fin de Ovalle sólo
interrumpió por tres días ---los del duelo nacional--- la dinámica tarea
constructiva de Portales. En los seis meses del gobierno interino de don
Fernando Errázuriz se dio tiempo para completar el ordenamiento del
Ejército, reorganizar la Academia Militar y crear la Guardia
 Cívica. Estas milicias acogieron en
sus cuarteles al grueso de la juventud consciente, que encontró allí una
escuela de disciplina, y fueron el muro de contención levantado contra
las aventuras caudillistas, fuesen las conspiraciones de Freire desde el
Perú o el sueño de Rodríguez Aldea de hacer retornar a O'Higgins. El
Ministro en persona tomó el mando del batallón número 4 de infantería y
se entregó al estudio. de la táctica militar y al entrenamiento dé su
tropa, que cada mañana ejercitaba en los patios de la Casa de Moneda. De
ahí el uniforme de teniente coronel de milicias con que aparece en el
retrato de Domeniconi. Andando el tiempo, la Guardia Cívica iba a salvar
en El Barón, si no la vida de su fundador, el régimen que él construyó y
dejó a sus conciudadanos como un legado.

En su paciente esfuerzo por afianzar la unidad y robustecer el sentido
de la nacionalidad, todavía embrionario, instituyó la celebración del 18
de septiembre con la pompa militar y festejos populares que el tiempo
haría tradicionales.

Portales, autodidacto que leía a Ovidio, Pope y Shakespeare, quiso
también hacer algo por la incipiente educación y cultura de su medio.
Reorganizó y extendió el plan de estudios del decaído Instituto
Nacional. Contrató al naturalista francés Claudio Gay
 para que escribiera su monumental
Historia física y política de Chile y dibujara el finísimo Atlas que se
imprimió en Europa. Necesitado el país de una prensa apolítica,
universal y cultural, había concebido
desde un comienzo la fundación de El
Araucano, en el que otro sabio, el venezolano Andrés Bello, iba a
introducir las columnas de literatura, artes, ciencias y comentarios
internacionales, nunca vistas hasta entonces en el diarismo santiaguino.
Inaudito periódico semioficial, hoy día inimaginable, que no aceptaba
diatribas personales ni contestaba a los ataques malévolos al
Gobierno\ldots

Pero los pilares del ``Estado en forma'' que se estaba creando fueron la
Ley de Elecciones y las iniciales reformas legislativas. La primera
permitió a la ciudadanía asistir al inusitado espectáculo de unos
tranquilos y limpios comicios de electores, cabildantes, parlamentarios,
Intendentes, jueces de letras y Presidente y Vicepresidente de la
República. Algo ya casi olvidado por los chilenos y todavía no soñado en
la mayoría de los turbulentos países del continente. Se había concedido
derecho a votar a las capas
superiores de la clase obrera. La
rigurosa vigilancia y la pena de cárcel eliminaron los fraudes (la mitad
de las inscripciones pipiolas debieron anularse por falsas) y el voto
era ahora estrictamente personal y los escrutinios se hacían bajo el
control del cura de cada parroquia, el regidor más antiguo y tres
ciudadanos elegidos por sorteo. Allí estaba la mano de Bello y la de
Egaña, como  lo estuvo en las
reformas de la legislación civil, criminal y procesal, donde los
jurisperitos interpretaron la dura concepción portaliana de la sanción
del crimen introduciendo este principio revolucionario: ``la embriaguez
no es un atenuante sino un agravante''. Otra innovación drástica
declaraba que ``la transacción entre las partes no tiene fuerza respecto
de las penas del delito'', vale decir, que el castigo de la ley era
insoslayable.

 En quince meses el triple Ministro
había sacado a su pueblo del caos pacificando los ánimos, implantando
hábitos de orden y trabajo, levantando una valla armada contra los
pronunciamientos de cuartel, reconstruyendo la deshecha economía pública
y privada y enseñando los rudimentos de la democracia representativa.

Sobre esta base ya era posible mirar hacia el futuro sin temores, y el
prodigioso estadista pensó que ahora podría retornar a sus negocios
abandonados y dedicar una noche que otra a sus amadas expansiones con
arpa y vihuela\ldots{}

Al presentar su dimisión a Errázuriz dejaba al país convertido en ``la
perla del Nuevo Mundo'' que había soñado. Especie de milagro conseguido
sin caer en la tiranía, sin coartar la libertad de opinión
hablada o escrita, sin despojar a
nadie, sin crear un impuesto ni sacrificar otras vidas que la de los
malhechores.

 Pero las elecciones que ungieron
Presidente de la República al general Prieto elevárosle a él mismo a la
Vicepresidencia ---honor que no buscó--- y el vencedor
de Lircay le pidió, le rogó y le
exigió que aceptara además la cartera de Guerra y Marina de su Gabinete.

¿Por qué no fue él Presidente, cuando dos veces le ofrecieron la
candidatura? Sencillamente porque no tenía esa ambición, y su respuesta
es célebre: ``No cambio la Presidencia por una zamacueca''.

¿Y por qué fue Prieto el elegido? Porque él, Portales, el salvador de
Chile, había indicado su nombre a los partidos mayoritarios\ldots

\chapter{AVENTURA DE LA GOLETA ``POMARE'' EN OCEANIA}

Don Juan Francisco Doursther, ilustre abuelo de los Tocornal, había
llegado a Chile en 1826 para Ocupar el puesto de cónsul de los Países
Bajos en Valparaíso. Trajo como secretario al belga Jacques Antoine
Moerenhout, antiguo oficial de los ejércitos de Napoleón. Doursther, de
veintiséis años y ``hombre de gran temple y de conducta extremadamente
honorable'', al decir de Moerenhout, instaló su oficina en la calle de la
Planchada, hoy Serrano, seguramente en una casa de adobes y tejas, que
mejores no las había en el puertecillo de veintitantas mil almas y
transitado por carretas, birlochos, jinetes y recuas de mulas.

Apagado el dinamismo de la Independencia, la vieja caleta de Quintil se
había vuelto a adormilar, pero subsistía el espíritu de empresa que
O'Higgins alentara con sus medidas en favor del comercio naviero. El ojo
observador de Doursther se fijó en cierta iniciativa audaz que un grupo
de inversionistas estaba llevando a cabo. Era una sociedad con nueve mil
pesos de capital destinada a la pesca de perlas y nácares en Polinesia y
en cuya nómina figuran los señores Pedro Alessandri, José Manuel Cea (el
socio de Portales), Francisco Javier Urmeneta y la señora o
señorita María López. Estos pioneros
chilenos de los Mares del Sur habían comprado la goleta Sociedad, de
cien toneladas de carga, y bajo el mando del capitán T. West la
despacharon al archipiélago de Gambier. En Francia, país que
centralizaba el mercado de nácares, pagábase a razón de ciento sesenta
mil francos la tonelada, aparte de las perlas, de manera que en una sola
expedición afortunada (teóricamente) podía rescatarse el capital\ldots{}

Esta expectativa tentadora fue la que indujo al cónsul belga a crear la
firma Doursther Serruys y Compañía,
que en 1828 comenzó a operar con el bergantín Volador, apenas más
espacioso que la cascarilla de Alessandri y Cía. A su bordo embarcóse
Moerenhout, cuya pluma de narrador privilegiado contaría
después las peripecias de esta
incursión y de otras en un libro
clásico: \emph{Voyages aux iles du Grand Océan}.

Pero don Juan Francisco Doursther quiso él mismo conocer la emoción de
la aventura exótica, y en noviembre del año 31 delegó el Consulado en su
secretario y partió en la goleta de tres palos Reina
Pomaré con rumbo a Tahití.

Los cómodos viajeros modernos difícilmente pueden imaginar lo que era
una travesía del Pacífico en un
velero de ciento ochenta toneladas, haciendo escalas en puertos sin
autoridades ni recursos y tratando con gentes hostiles y primitivas. Ya
Moerenhout sabía lo que es
estrellarse en un arrecife de coral y don Pedro Alessandri había quedado
escamado con la belicosa recepción de los isleños de Mangareva.

Lo que iba a sucederle a Doursther parece ahora un episodio de novela de
Salgari.

 La veloz Pomaré llegó a su destino
en cincuenta y un días después de recalar en Pascua, Pitcairn y Gambier;
pero junto con echar el ancla en Papeete surgió el primer contratiempo.
Aunque el nombre del barquito constituía un homenaje a la joven reina
tahitiana, ésta consideró que era irrespetuoso y el capitán Clark tuvo
que hacerlo borrar del tablero de popa.

La isla más bella y seductora de Los mares, dibujada por cumbres
vertiginosas, lagunas transparentes,
playas y palmeras de ensueño y patria
de gentiles mujeres, no era sino un lugar de paso en donde los buques
aportaban para recoger a los buzos indígenas; los únicos seres humanos
capaces de permanecer dos minutos bajo el agua forcejeando con las
ostras y capeando a los tiburones. Con veinticuatro de estos
especialistas, más un intérprete, se dirigió la Pomaré al archipiélago
Tuamotú, lugar sembrado de restos de naufragios y donde la goleta casi
encalla en una islilla que no aparecía en la carta. Fueron a fondear en
el atolón del Harpa, anillo de arena que apenas sobresale del mar y
forma un lago salado de decenas de millas cuadradas donde proliferan los
bancos perlíferos. Paraje de placidez panorámica infinita\ldots, Y habitado
por aborígenes que en un principio se
mostraron amistosos. El rey de la aldea ofreció el concurso de sus
desnudos mocetones y con genial desfachatez se
instaló como alojado en la cámara del
navío.

Cuatro embarcaciones fueron arriadas y bogaron en demanda de los bandos,
situados ocho millas laguna adentro. Todo marchó a pedir de boca los
tres primeros días. Los botes
regresaban colmados, guiándose en la noche por la señal luminosa que el
buque izaba en su aparejo.

Al amanecer de la cuarta jornada Doursther sintió un vocerío y carreras
en el puente y vio que unos indígenas
penetraban violentamente en su camarote. La goleta había sido abordada
por una flotilla de canoas cuando el
capitán se hallaba en los bancos y la tripulación dormía desprevenida; y
el rey en persona dirigía el asalto. De un brinco Doursther dejó la
litera, cogió su pistola y disparó cuatro tiros consecutivos, hiriendo
en el pecho al primer atacante que asomó la cabeza. Fue todo lo que
alcanzó a hacer. Luchando solo, no pudo impedir que el rey, el secuaz
herido y finalmente una pandilla le derribasen, golpeándole con puños y
pies, para maniatarlo por último y sacarlo a cubierta más muerto que
vivo. Simultáneamente había sido reducida la tripulación,
de piloto a cocinero; triste grupo de
indiferentes que se dejaron amarrar sin amago de resistencia.

Fueron todos trasladados a tierra y atados cada uno al tronco de un
cocotero. Permanecían libres los buzos traídos de Papeete, y uno de
ellos, compadecido, entregó a Doursther su ropa, galletas de. mar, una
botella de ron, agua y cigarros. Entretanto, las mujeres y niños del
lugar elevaban griteríos indescriptibles a la vista del isleño baleado,
que se paseaba impávido y cubierto de sangre. Durante horas
interminables el desdichado empresario estuvo esperando la muerte a
manos de esos aborígenes con fama de antropófagos, y su desesperación le
hizo pensar en el suicidio; pero hasta su cortaplumas le había sido
robado.

A las once de la mañana reapareció el capitán Clark, al que habían hecho
prisionero después de una lucha que dejó en su nariz, ojos y boca la
huella de los puñetazos. Hasta ponerse el sol no varió la situación, con
la agravante de que reforzaron las amarras y Doursther no pudo siquiera
satisfacer la más urgente necesidad
natural, Al caer la noche sus guardias lo hicieron. tenderse en el
suelo, boca abajo, y so pretexto de impedir que huyese se acostaron
encima de su cuerpo magullado y adolorido. El infeliz no pudo soportarlo
y empezó a pedir a gritos que acabasen de quitarle la vida. Sólo
entonces le fue permitido ponerse de espaldas, las manos amarradas por
delante, para respirar tranquilo hasta rayar el
día.

El único de los blancos que no había sido apresado era Middleton, el
intérprete contratado en Tahití, que se hallaba pescando al producirse
el asalto. Tan pronto como estuvo de vuelta, Doursther le pidió que
preguntara a los salvajes cuál era el motivo de su actitud y qué se
proponían hacer con él y con los suyos\ldots Las respuestas eran vagas y
contradictorias: que estaban ofendidos por el nombre irrespetuoso del
buque, que Clark les trataba con dureza, que el capitán de otro barco no
les pagó lo convenido, que los buzos (falso) deseaban ser repatriados\ldots
Era todo como una pesadilla, entre largos silencios y las risitas
incongruentes del rey.

Resuelto a salvar siquiera el pellejo, el empresario prepuso que sacaran
de la bodega de a bordo lo que quisieran, y que, si era el buque lo que
pensaban quitarle, que le entregasen un par de botes o canoas para
abandonar la isla\ldots Nuevas risitas y silencios. Se limitaron a saquear
la goleta llevándose ropa, armas de fuego, herramientas y víveres. A
manera de trofeo, el rey guardó para sí los libros de navegación.

De improviso, al día siguiente, el risueño monarca ordenó quitar las
amarras a los prisioneros y devolverles algunas de sus pertenencias.
Entregaron al jefe de la expedición cuarenta libras de galletas, treinta
de carne salada, tres botellas de vino, dos docenas de cocos, dos libras
de té, veinte de tabaco, una sartén y
dos tazas. Con esto tenían que
sobrevivir, quién sabe por cuánto tiempo, los robinsones de Valparaíso,
mientras la bandera estrellada de la
Reina Pomaré les recordaba la patria que temían no volver a ver.

Llegóse por último al acuerdo, inopinado como todo lo anterior, de que
Middleton con dos marineros saliese en la goleta a repatriar a los
buzos, maniobra que aprovecharía el
intérprete para tratar de ponerse al habla con las autoridades
tahitianas y pedir el rescate de los cautivos.

Al dejar el surgidero el improvisado marino perdió el ancla y su cadena,
pero consiguió alejarse del arrecife navegando a la buena de Dios, sin
saber utilizar la carta, las tablas ni el sextante. Milagro es que haya
podido cruzar los procelosos canales del archipiélago, sembrados de
corales ahogados y donde se entrechocan turbulentas corrientes y mareas.

Quedaban once hombres abandonados a su suerte en el atolón, pero libres
al fin de la presencia de los trescientos nativos, que se alejaron
cuando la desgracia de los extranjeros dejó de parecerles divertida.

Hay atolones limpios y sucios, y el del Harpa .es
de estos últimos, vale decir que se
halla plagado de ratas, moscas, mosquitos, hormigas y lagartos. Para
colmo estaban en la estación de las lluvias, que allí
son torrenciales, y no disponían de
una sábana con qué protegerse. Doursther tenía las muñecas y tobillos
desollados por las ligaduras, el capitán Clark estaba casi ciego y uno
de los marineros sufría dolorosas
heridas y contusiones. Reducidos a
esta mísera condición se dispusieron a buscar la supervivencia en ese
paraíso engañoso de las Tuamotú, llamado con justicia el Archipiélago
Peligroso.

La primera medida de Doursther fue
hacer levantar dos chozas de ramas de cocotero: una para la marinería y
la otra para sí, para el capitán, el piloto, el contramaestre, el
carpintero y el mayordomo. El frágil cobertizo tamizaba Los rayos del
sol tropical pero apenas si disminuía el paso de la lluvia, de modo que
los moradores se mojaban lo mismo adentro que afuera. Con hojas de
hierbas recubrieron el suelo arenoso, y una estera fue la cama de
Doursther. Las rústicas habitaciones estaban a distancia de
 doce millas de la entrada de la
laguna, el lugar hacia donde iban a vivir espiando el arribo de sus
libertadores. Como no podían saber cuándo llegarían, acordaron racionar
los víveres con rigurosa parsimonia; previsión que no cabía en la mente
de los marineros, los cuales en una semana consumieron el total de su
reserva.

Cuando ya entreveían el fantasma del hambre apareció un grupo de
aborígenes trayéndoles peces y langostas, como si nada hubiera sucedido
entre ellos y sus víctimas. En un descuido de sus superiores, que
insistían en economizar el alimento, los
marineros se dieron tal panzada de
pescado que volvió a agotarse la despensa. Su incorregible indisciplina
obligó a tomar una decisión terminante: cesaba la comunidad de los
víveres y cada cual debería procurárselos en lo sucesivo como pudiese.
Entonces los bribones resolvieron dedicarse a la pesca, y para no tener
que compartir la comida fuéronse a vivir a la entrada de la laguna.

A cada día de ocio y ansiedad sucedíase una noche en vela espantando los
mosquitos y las ratas que pasaban por encima de sus caras. Una lluvia de
siete días consecutivos dejó a Doursther exhausto y enfermo. Cuando
menos lo esperaba llegó el rey a instalarse en la choza para hacerle
compañía. A instancias suyas mudaron la residencia cerca de la aldea,
donde la presencia de los perros mantenía alejados a los roedores y
lagartos. Pero a cambio de esta comodidad el jefecillo les prohibió
cortar ramas de cocoteros para hacerse una nueva choza, y en adelante
vivieron a la sombra de los árboles. Como lecho tenían que elegir entre
la hierba anegada por los chubascos o los ásperos trozos de coral que
les enconaban la piel y rompían la ropa.

La vecindad del poblado era más deprimente que útil a causa del atraso
prehistórico de sus habitantes, que vivían en la ociosidad,
despiojándose y comiéndose Los piojos y arrojando a las mujeres los
desechos de pescados y cocos. Un día que el capitán se acercó a mirar
una tortuga, sin intención ni de tocarla, azuzaron a los perros, que le
persiguieron mordiéndole las piernas.

Al, cabo de un mes no cicatrizaban del todo las heridas de Doursther,
que un curandero indígena trataba con hojas de un árbol medicinal. La
escasa alimentación, a veces el hambre, le habían debilitado hasta el
punto de embotarle los sentidos. Estaba a merced de la caridad, comiendo
de las sobras de los isleños; la ración habitual era un coco y cierta
plantita hervida en agua salada, que se hacían pagar en tabaco. Cuando
el hambriento conseguía dormir soñaba que estaba en su hogar de Namur,
con su familia, saboreando manjares y licores exquisitos.

Cada vez que Clark quiso salir de pesca, se negaron a prestarle una
canoa por el solo placer de negársela. Curiosamente, el único gesto
humanitario provino del indígena baleado por Doursther, que obsequió a
éste unos trozos de carne de tortuga. El resto de los aldeanos dedicóse
a robar los últimos objetos que quedaban a los blancos: la sartén, las
dos tazas y una navaja que usaban para limpiar el pescado y los cocos.

Tan pronto como las lluvias lo permitieron, Doursther se dedicó a lavar
su ropa, ya medio despedazada, sin que ninguno de sus hombres se
ofreciese a ayudarle. Y cuando encendió una fogata para protegerse de
los insectos y del fresco nocturno, el señor piloto, el señor
contramaestre, el señor mayordomo y el señor carpintero sentáronse en
primera fila sin preocuparse del jefe de la empresa, que quedó de pie y
soportando el frío hasta que Clark les explicó en qué consisten el
respeto y. la buena crianza.

Se cumplían treinta y nueve días de permanencia en ese paradisíaco
infierno, cuando un bote con seis hombres armados de fusil entró a la
laguna del atolón. Si Doursther temió antes sucumbir de miseria, creyó
ahora morir de alegría, pues había calculado que, en el mejor de los
casos, y suponiendo que Middleton lograra llegar a Tahití, no podría
estar de vuelta en menos de un par de meses.

Dándose cuenta de lo que estaba por ocurrir, el
rey congregó a su gente con la
intención de impedir el desembarco y ordenó disparar contra la
embarcación los fusiles que robaran del armero de la Pomaré. Pero acto
seguido se echó a temblar y quedose clavado en el arsenal al ver a los
invasores saltar a tierra.

El oficial que los mandaba se presentó a Doursther como el capitán
Ebrill, un irlandés al mando del bergantín-goleta Elisa, de matrícula de
Valparaíso.

Este barquichuelo de cincuenta
toneladas acababa de ser adquirido por Doursther Serruys y Compañía y
encontrábase en Papeete cuando el intérprete llegó en busca de socorro;
de ahí la presteza con que Ebrill se había hecho presente en el Harpa,
adonde arribó con dieciocho días de travesía desde Tahití. El Elisa,
falto de viento, aguardaba al otro extremo de la laguna, pero el bote
traía provisiones que llegaban en momentos de angustiosa necesidad,
cuando hacía treinta y seis horas que los cautivos no probaban bocado.
Olfateando la comida y la liberación, los leales marineros de Clark
corrieron a reunirse en torno a su capitán\ldots

Casi hundiéndose bajo el peso de
veinte personas, la embarcación bogó en demanda del bergantín. Entonces
supo Doursther por qué había venido el Elisa y no la Pomaré a
rescatarlo. Mientras Middleton andaba en tierra, la goleta había sido
saqueada hasta dejarla inutilizada para hacerse a la mar, y el seguro
contra robo pierde su validez cuando la nave ha sido dejada sin
vigilancia\ldots ¡Tal es el precio que suele pagarse por crear una empresa!

Pero Doursther había salvado la vida, y su exaltación era tal que al
llegar a borde fue presa de un ataque de nervios.

Antes de emprender el regreso, Ebrill y Clark quisieron cumplir dos
diligencias que estimaban necesarias para la edificación moral de los
isleños. La primera fue mandar a
tierra un piquete con bayoneta calada encargado de destruir las chozas,
las canoas, los aparejos de pesca, los utensilios, las esteras y los
depósitos de agua; todo lo cual se amontonó para ser
quemado
en gigantesca hoguera. Una vez arrasada la aldea, apresaron al rey y a
tres de sus más crueles secuaces, y amarrándoles al pie del palo mayor,
les dieron a cada uno, por turno, cuatro docenas de latigazos; luego los
arrojaron por la borda para que se fueran nadando hasta la playa.

\chapter{EL PERIODISTA DIEGO PORTALES}

\epigraph{``Sólo en los países libres son libres los escritores''.}{CAMILO HENRIQUEZ}

En septiembre de 1830 el Ministro Portales encargó a don Manuel José
Gandarillas la fundación de El
Araucano. Como parte de la obra portaliana,
este periódico semiindependiente iba
a probar su solidez publicándose hasta 1877, año en que se transformaría
en el Diario Oficial para perdurar en una existencia que ya parece
eterna.

Desde los tiempos de la Aurora de Chile las hojas impresas habían
proliferado como las callampas después de la lluvia. Con la única
excepción de El Mercurio de Valparaíso, fueron simples aventuras
editoriales, generalmente de índole política y sin base comercial capaz
de sustentarlas; duraban semanas o días y desaparecían en medio de la
indiferencia de su ínfimo público. Sus nombres excéntricos o
grandilocuentes, concebidos tal vez para producir expectación o miedo,
ahora mueven a risa como las máscaras que nos asustaban de niños: El
Sepulturero (27 números), La Antorcha de los Pueblos (6 números), El
Azote de la Mentira (8), El Clamor de la Verdad (1), El Muchacho del
Cura Monardes (1), El Hambriento, El Canalla, La Laucha, El Defensor de
los Militares denominados constitucionales\ldots{}

Fue
este último el que atacó al Gobierno de don José Tomás Ovalle, y por
consiguiente a su triple Ministro, ofendiéndoles hasta el punto de
convencer a Portales de la necesidad de crear un órgano de informaciones
oficiales. Siempre hay que recordar que este estadista autoritario, pero
absurdamente acusado de déspota, no sólo admitía, sino que hasta
alentaba, la libertad de prensa y aun en pleno régimen de Facultades
Extraordinarias era lícito criticar su política. El Defensor de los
Militares denominados constitucionales no fue censurado en su campaña a
favor de los pipiolos destituidos a raíz de la batalla de Lircay; pero
cuando llamó a Ovalle leso y borrico y a Portales ignorante, necio,
bribón y criminal, don Diego estimó
que hasta ahí no más se podía llegar. El Defensor fue acusado ante el
Tribunal de Imprenta por el fiscal de la Corte de Apelaciones, y de este
juicio salió el editor Anacleto Lecuna condenado a cuatro años de cárcel
conmutables en destierro. Escaparon indemnes los redactores, que eran
todos militares, excepto el pedagogo español José Joaquín de Mora; y el
furibundo diarito no pudo tomarse otro desquite, antes de morir, que el
de lanzar una última andanada ponzoñosa contra El Araucano recién salido
a la palestra.

Gandarillas,
apodado El Tuerto, estuvo ligado a Portales por relaciones alternadas de
amistad y rivalidad; pero seguramente fue hechura suya y coincidió con
su criterio de que el nuevo periódico no tenía por qué cantar loas
sistemáticas al Gobierno. Ante todo, debía ser objetivo e informativo,
debía estar correctamente escrito y
exponer• su opinión sin temor de disentir con el pensamiento de palacio.
A esta original concepción debió El Araucano el inmediato favor del
público y su larga vida, así como El Mercurio porteño había encontrado
la clave del éxito en su índole de diario mercantil.

El Araucano vio la luz el 17 de septiembre, como
regalo de víspera de Fiestas
Patrias, compuesto e impreso en los talleres de don Ramón Rengifo.
Gandarillas tenía el antecedente ilustre de haber sido redactor y
tipógrafo de la Aurora de Chile, y Portales era también un fogueado
periodista, cosa que hoy sólo saben
unos pocos eruditos. Había sido inspirador y redactor de El Hambriento,
hoja fundada el año 27 con el fin de atacar y ridiculizar a los
pipiolos. El, estilo inconfundible
del Epistolario portaliano está patente en las adivinanzas en verso, en
los traviesos ``juegos de prendas'' y en las ``noticias marítimas'' en
que daba a sus enemigos nombres de buques y enumeraba sus defectos,
pasiones y mezquindades como mercancías de peso abrumador que venían en
sus bodegas. Hasta creó un personaje, el escribano Perales, para
satirizar la corrupción de la Justicia. Especie de Topaze sin
caricaturas. concebido un siglo antes que el de Jorge Délano. En 1828
dirigió El Almirez, que tuvo como redactores al impresor Rengifo y a don
Victorino Garrido; periodiquito de tono chistoso y mordaz dedicado a
defender el Estanco y que dejó de existir en el número 2. Poco después
adquirió en dos mil cuatrocientos pesos la imprenta de El Telégrafo, de
Valparaíso, para editar por su cuenta El Vigía y proseguir disparando
metralla contra los pipiolos. Colaboró en El Avisador de Valparaíso,
donde puede leerse (2-VII-29) su artículo sobre ``Normas para elegir
profesores extranjeros'': sátira directa contra el pedagogo José Joaquín
de Mora y redactada, a manera de virtuosismo, en una sola frase de dos
carillas de extensión. Otro artículo suyo, en El Crisol, refiere la
historia del Estanco demostrando cómo la incurable maledicencia nacional
había acusado a los concesionarios de obtener cuantiosas utilidades
ilegales en una empresa de bien público que sólo les dejó pérdidas.

Como periodista, no menos que como estadista, este hombre de dotes
misteriosas era un autodidacto.

Cuando Ovalle le confió las carteras del Interior, de Relaciones y de
Guerra y Marina, era un político infalible (aunque odiaba la política),
un gobernante de madurez definitiva (aunque sin experiencia anterior) y
un escritor de antología (aunque nunca presumió de tal y escribía como
jugando). Introducido en la prensa, no por vocación sino por necesidad,
impuso un estilo, fundó uno de los periódicos de más larga trayectoria
en nuestro medio y sentó precedentes de
ética periodística gubernamental. Es
la ética nueva para el Chile de entonces y olvidada en el de hoy, que
encontramos en la preciosa cantera de sus
cartas:

queremos aproximarnos a la
Inglaterra en cuanto sea posible en el modo de hacer la oposición; que
el decreto que autoriza al Gobierno para suscribirse a los periódicos
con el objeto de fomentar las prensas y los escritores no excluya a los
de la oposición; que siempre que ésta se haga sin faltar a las leyes ni
a la decencia, el buen gobierno debe apetecerla. Si no hay causa para
atacarlo, silencio; y si las hay, echarlas a la luz con sus pelos y sus
lanas.''

``La justicia expresada con buenas razones tiene gran poder, al paso que
lo pierde cuando se sostiene con intemperancia.''

``Lleve el Gobierno una marcha franca, legal, decente y honrada, y ni se
nublará el horizonte ni tendrá que temer, aunque se nuble\ldots''


 Tal era la filosofía y tal la norma
de El Araucano, cuya eficiencia y longevidad nada tienen que ver con la
modestia de sus cuatro páginas de tamaño tabloide. Hoy parecería un
volante, pero adentro había unos editoriales, unas notas y unas
traducciones del hombre que iba a depurar la gramática castellana, que
tradujo a Horacio y a Lord Byron, que iba a redactar el Código Civil y a
organizar la Universidad de Chile. Porque el fundador quiso darse el
lujo de que don Andrés Bello, la primera cabeza intelectual de América,
tomase a su cargo las secciones literaria, científica y jurídica para
dar al periódico las alas de alto vuelo que requería.

Vendíase El Araucano al precio de un real, digamos uno o dos centavos, y
se sabe que el tiraje corriente en la época no pasaba de cuatrocientos
ejemplares. Todo era así, a escala de miniatura, en los días de esfuerzo
y pobreza en que se organizaba la República, cuando el Presupuesto
Nacional era de dos millones de pesos, el Ejército contaba tres mil
hombres, la Armada dos buques viejos y Portales no tenía un frac de
ceremonia que ponerse y rehusaba sus sueldos para aliviar las
estrecheces del Fisco.

El pequeño Araucano, compuesto a tres columnas y
 programado para aparecer los
sábados por la tarde, salió a la calle con un humilde y solitario aviso.
La tienda de don Antonio Ramos anunciaba (posiblemente exhibidas entre
comestibles, géneros y remedios de
hierbas) cinco novedades literarias, entre ellas \emph{El chileno
consolado en los presidios}, de don Juan Egaña, impreso en Inglaterra, y
\emph{Cartas peruanas, o preservativo contra los libros impíos y
seductores que corren en el país}. Una noticia de Londres daba cuenta de
que escuadrillas españolas habían salido para Cuba y Filipinas con el
supuesto propósito de intentar la reconquista de México. La alarmante
información era traducida de \emph{The Times} de fecha 10 de mayo y
había tardado tres meses en llegar a Valparaíso, a vela, vía Cabo de
Hornos. Era un flash de la época. El editorial decía con orgullo que
``el uso de la imprenta goza en Chile de la más absoluta libertad'', y
declaraba que el semanario se
proponía ``agradar e instruir a los verdaderos amantes de la
ilustración, sin fomentar rencores ni dar pábulo a esas pasiones
lastimosas que se alimentan con las discordias, con las animosidades,
con la burla del hombre y con la ofensa del ciudadano''. Fiel a este
principio, no publicaría ni contestaría ataques personales. Advertía que
los editores podían verse precisados alguna vez a sostener providencias
del Gobierno, o a defender su comportamiento, ``y lo previenen para que
en ningún tiempo se les tache de inconsecuentes''. Un largo artículo
sepultaba con •todos los honores a los líderes de ayer que pudiesen
levantarse contra el régimen: ``Ya no domina el concepto de don Bernardo
O'Higgins, ya el prestigio de don Ramón Freire se extinguió como un
meteoro, ya don Francisco Antonio Pinto acabó su carrera''. Y cumpliendo
sin demora lo anunciado, defender el comportamiento del Gobierno, el
articulista anónimo pasaba revista a
sus logros reconocidos: ``la pureza en la inversión de las rentas
públicas, la rectitud en la distribución de los empleos, la empeñosa
constancia en desterrar esa manía de los empeños, la contracción por
mejorar la educación, por arreglar las milicias, por evitar los
crímenes, por corregir las infracciones a la ley\ldots''

No enteraba El Araucano tres meses de vida cuando le salió al paso \emph{El
Trompeta}, pasquín pipiolo dispuesto a ensordecerlo con su grito
antigobiernista, y en cuyo cuerpo de redactores se contaba el
incorregible español José Joaquín de Mora. Como olvidado de su
experiencia con El Defensor, y de que estaban vigentes las Facultades
Extraordinarias, Mora disparó contra los gobernantes la letrilla que
comienza:

El uno subió al poder\\
con la intriga y la maldad;\\
y al otro sin saber cómo\\
lo sentaron donde está.

El uno cubiletea\\
y el otro firma no más;\\
el uno se llama Diego\\
y el otro José Tomás.

Sobrepasando todo lo anterior, calificó a Ovalle de asno y de niño bajo
tutela, y luego le mandó a su Ministro este recuerdo: ``Se necesita un
siglo y cuarenta y tres liceos para borrar de Chile el espíritu de
venalidad introducido y propagado por el pillo de los
pillos, don Diego Portales''. Llamó
al personal del Gobierno ``gavilla de ladrones'' y sostuvo el derecho de
la ciudadanía a quitar la vida a los tiranos y sus

cómplices.

Ni Mora ni sus conmilitones habían sabido distinguir entre libertad y
licencia, y desafiaban por segunda vez el principio portaliano de que la
autoridad no puede por motivo alguno tolerar el insulto ni el llamado a
la sedición. Si todavía no conocían a Portales, ahora lo conocieron. Un
rápido proceso silenció a El Trompeta con una sentencia que dio con los
redactores en la cárcel y con Mora expulsado del país en un buque de
carga que lo. llevó al Perú.

En el curso del año 32, bajo el
gobierno de Prieto, pareció a Portales que El Araucano se volvía
monótono y perdía terreno en la competencia periodística. Fue una de las
causas que determinaron la caída en desgracia de Gandarillas,
reemplazado a la postre por Bello como director y primer redactor. De
ahí el alborozo de don • Diego ante la aparición de El Hurón, periódico
de garra fundado por los habitués de su propia tertulia política. La
probabilidad de que haya colaborado en él se desprende de su carta a don
Antonio Garfias, escrita desde Valparaíso, donde le dice: ``El país
necesita de un buen papel al lado del monótono Araucano''. ``Si el (nuevo)
periódico anda bien, yo les ayudaré con algunos articulillos que usted
deberá presentarles a los editores como que son suyos''.

Por entonces colaboraba en El Mercurio, el único diario vivo más antiguo
que el creado por él. Aunque no firmaba, se identifican sus artículos
por el estilo, por referencias en el Epistolario y por la insistencia
sobre su tema favorito: los vicios de la administración de justicia. Con
su sentido del gobierno fuerte y su concepto del castigo implacable del
delito, denunciaba que en Chile ``la ley no sirve para otra cosa que no
sea producir la anarquía, la ausencia de sanción, el libertinaje, el
pleito eterno y el compadrazgo''

Ciertas alusiones del diario porteño a Gandarillas, cuando éste estaba
ya fuera de El Araucano, hicieron creer al director despedido que
Portales le atacaba escudado en el anonimato. Para hacerle saber que
estaba en el error, mandó esta carta a Garfias, su corresponsal e
intermediario: ``Sé por muy buen origen
que el pobre tuerto G. está en El
Monte hecho una fiera conmigo. Su estupidez y ceguedad llegan hasta
el extremo de figurarse que yo soy
el autor de los artículos de El Mercurio, y dice que lo sabe
positivamente. Compadezcamos a este pobre hombre y deseemos que
restablezca su salud para alivio de su familia\ldots''

Escribiéndole a Ochoa, redactor en jefe de El Mercurio, había
recomendado que ese diario secundase la política del Gobierno sin caer
en el servilismo ni comprometer su independencia. No dejó nunca de
defender el ideal de la prensa libre y la necesidad de una oposición
constructiva. En tal sentido su Araucano mantúvose en la línea de honor
que le impusiera y dio un ejemplo que parecería increíble si no
pudiésemos comprobarlo en la colección del periódico y en la
correspondencia privada (carta a O'Higgins) del Presidente Prieto:

``Aunque El Araucano es el órgano de que se vale el Gobierno para las
comunicaciones oficiales, -está tan lejos de tenerlo a su devoción que
en estos mismos días se ha hecho la guerra en sus columnas a ciertos
puntos de reforma constitucional en que eran bien conocidos el interés y
los deseos del Ejecutivo''.

Lo que Prieto le contaba a O'Higgins es que el periódico semigobiernista
había combatido ciertos artículos o capítulos de la Constitución del
33\ldots, ¡la Constitución inspirada por Prieto y Portales y parcialmente
elaborada por Bello, redactor de El Araucano!

\chapter{LOS AMORES DE PORTALES}{(Según el testimonio de su correspondencia)}

\epigraph{\ldots{}fue un hombre inverosímil, paradójico, increíble.''}{RAMON SOTOMAYOR VALDES}

El 15 de agosto de 1819 don Diego Portales contrajo matrimonio con su
prima doña María Josefa Portales y Larraín. De esa fecha arranca una de
las historias sentimentales más
conmovedoras de que haya recuerdo. También señala ella la raíz del
acontecimiento que fijó el carácter y el destino del más célebre
estadista nacional.

Al establecerse ---como entonces se decía del acto de tomar esposa---,
Portales contaba veintiséis años de edad. Por lo que se sabe fue María
Josefa la primera mujer en quien puso los ojos; y según su propia
confesión, gozó en este amor de ``una dicha infinita''.
Si es verdad que don Diego descendía
de los Borgia (como lo afirman sus biógrafos con el árbol genealógico a
la vista), no desmintió en esa pasión la exuberancia de su estirpe. Amó
a Chepita con la intensidad de que sólo puede ser capaz quien posea a la
vez las tendencias del misticismo y el sensualismo.

La posteridad desconoce el aspecto material de la inspiradora de este
amor, pues no quedó de ella pintura ni descripción. Lo cual no es de
extrañar, desde que a su propio marido no le conocemos la verdadera
fisonomía. El único retrato del natural que de él existe (debido al
italiano Domeniconi), fue realizado después de su muerte y sirviéndole
de modelo el rostro desfigurado por las balas y las bayonetas de los
asesinos.

La idílica unión no alcanzó a durar dos años. Una enfermedad fulminante
se llevó a María Josefa en el invierno de 1821 y dejó a su compañero
hundido en el abismo de la desesperación.

De
la magnitud de este golpe sólo puede dar una idea la resolución que el
viudo se impuso como un voto sagrado: la de quedarse solo por el resto
de sus días. Parece ser que en cierto momento pensó en tomar los hábitos
religiosos. Confesaba y comulgaba a diario, y hasta solía vérsele en el
coro de su parroquia, siguiendo ``con voz acentuada y fino oído'' los
cantos litúrgicos.

Temiendo que cayese en la misantropía o en la locura, su anciano padre
quiso salvarlo con el consejo de la sabiduría vulgar: el de que tomase
una nueva esposa\ldots A ello contestó don Diego en una carta patética,
pieza maestra de su Epistolario, y que es el exacto reflejo de su estado
anímico de entonces:

``Con el correr de los días, que cada
vez me son más penosos, la ausencia eterna de Chepita no ha hecho más
que aumentar la pena que me aflige. Tengo el alma- destrozada, no
encontrando sino en la religión el consuelo que mi corazón necesita. He
llegado a persuadirme de que no pudiendo volver a contraer esponsales
por el dolor constante que siempre me causará el recuerdo de mi santa
mujer, por la comparación de una dicha tan pura como. fue la mía, con
otra que no sea la misma, no me queda otro camino que entregarme a las
prácticas devotas, vistiendo el hábito de algún convento. Con ello
perseguiría lo que como hombre todavía no consigo ni creo conseguiré
jamás: dejar en el olvido el recuerdo de mi
dulce Chepa. Por eso sus empeños
para que contraiga nuevamente, me parecen algo así como un consejo
terrible, y por lo mismo, inaceptable. Viviré siempre en -el celibato
que Dios ha querido depararme, después de haber gozado una dicha
infinita. Crea usted que las mujeres no existen para mi destrozado
corazón: prefiero a Dios y a la oración antes de tentar seguir el camino
que inicié con tanta felicidad\\ldots''

Fue esa obsesión de olvidar la que le indujo a alejarse del país, yendo
a establecerse en el Perú con el pretexto de abrazar la carrera
mercantil.

Cumpliría, ciertamente, la promesa
de no volver a casarse\ldots Pero
nadie, y acaso ni él mismo, hubiera podido prever el raro vuelco que
iban a experimentar sus sentimientos. Aquella fiebre religiosa comenzó
pronto a enfriarse, hasta devenir en
un escepticismo burlón. No mucho después le diría a un amigo de su
intimidad: ``Usted cree en la religión, mientras yo
creo en los curas'', queriendo
significar que sólo se servía de la Iglesia para sus fines utilitarios.

Y conforme se retiraba el místico por un lateral de la escena, hacía el
sensual su entrada espectacular por
el foro. Un nuevo Portales, el vividor alegre, el eximio catador de
mujeres, surgía ante el asombro de sus conocidos.

Casi recién llegado a Lima, escribíale a don José
Manuel Cea, su socio y confidente:
``Vivo aquí en compañía de Julia''. Era su primer amorío en tierra
peruana; y de su durabilidad decía en la misma carta: ``Estoy dispuesto a
darle la patada. Vivir con mujeres es una broma, sobre todo cuando son
intrigantes''.

En este breve lapso se había convertido en un conocedor del mujerío
limeño ---con tal asiduidad debió frecuentarlo---, y he aquí cómo
resumía su experiencia: ``Decididamente prefiero las chilenas a las
peruanitas. Estas son muy refinadas y falsas, muy ardientes y
ambiciosas, muy celosas y desconfiadas y
amaneradas''.

Parecía harto de ellas, pero en
realidad no hacía sino empezar la larga serie de sus enredos. Con fecha
 19 de mayo de 1822 le cuenta a Cea:
``Me cargué con un hijo a quien pienso reconocer. La historia es conocida
de usted. Lo que siento es que sea peruano''. El 13 de septiembre, cuatro
meses después, le informa de otro
lío peor, por causa del cual ha ido a parar a los tribunales: ``Si este
pleito se alarga y el doctor no anda listo, tendré que cargarme con una
mujer que de todo tiene menos de moral, y con un señorito que me echaría
en cara mi desvergüenza. Para dicha mía, la que ha sido mi querida no
tenía una fama muy limpia. El caballero Heres la había prostituido,
después don Toribio Carvajal, y por último Portales, que se ha llevado
la peor parte. Yo no habría entrado en relaciones con tal mujer
desvergonzada si hubiera sabido
estas circunstancias, que me hacen repudiarla con toda la fuerza de mis
odios; pero tuvo audacia para fingirme inocencia y para hacerme creer
que estaba virgen\ldots''

La celebridad de sus aventuras le hizo rápidamente famoso, y su prestigio 
acabó por despertar el interés de las damas de la alta aristocracia. Ha 
debido ayudarle su figura de caballero pálido, de noble continente y 
maneras exquisitas, enfundado en el frac impecable que le fue característico.
En ese conspicuo medio social encontró a la que iba a ser Su más escandalosa, 
pero también su más durable conquista: Constanza Nordenflycht.

Era ésta, cuando él la conoció, una
niña de diez y seis años y de extraordinaria belleza, hija huérfana de
doña Josefa Cortés y Azúa, encopetada matrona limeña, y del barón
Timoteo de Nordenflycht, sabio prusiano y antiguo consejero del rey de
Sajonia. Poseedora, como podemos presumir, de un temperamento
incontrolable, se enamoró locamente de don Diego, y ante la negativa de
éste a tomarla por esposa, optó por entregársele en condición de
concubina. Su pasión fue tal que un tiempo después, cuando Portales
volvió a la patria, fracasado y empobrecido, se vino siguiéndolo y no
paró hasta volver a reunirse con él. ¡Hay que imaginarse la polvareda de
chismes, aspavientos y maldiciones que la pareja habrá dejado en las
orillas del Rímac!

De esta unión ilícita, que duró
toda una vida, nacieron otros tres inocentes: Rosalía, Ricardo y Juan
Santiago Portales y Nordenflycht, que sólo vinieron a ser legitimados
después de los días de sus padres
por un decreto del Gobierno.

 Su principio de que ``vivir con
mujeres es una broma'', mantuvo a Portales a distancia de la que le amó
con tanta valentía y lealtad; y sus encuentros con ella fueron
esporádicos y cada vez menos frecuentes. Excepto una breve temporada en
que vivieron juntos en Valparaíso, la ardiente peruana residió en la
capital en compañía de una parienta, y solía pasar hasta un año sin ver
a su amante. Sin embargo, Portales no la abandonó, y aun veló por su
bienestar y el de sus hijos. Constanza, para él, fue el cariño
triste, medio olvidado, pero al cual
se vuelve de tarde en tarde.

En una de sus múltiples cartas confidenciales, el extraño amador
confiesa este propósito: ``Declaro a usted que no he contraído con ella
obligación alguna, y que para la puntual asistencia que ha recibido
siempre de mí, no he tenido otro móvil que mi propio honor, la compasión
y el deber de reparar Los daños que ella hubiese recibido por mi causa''.

La misma vida galante del Perú continuó en Chile, y no varió ni en los
años en que el hombre de negocios, convertido en estadista, llegó a ser
la primera personalidad de la nación.

Memorable en la tradición santiaguina es la Filarmónica, especie de
cabaré particular instalado por don Diego y sus íntimos, y mantenido a
sus expensas, para divertirse a puertas cerradas. Funcionaba en la calle
de las Ramadas (hoy Esmeralda), cerca del famoso teatro al aire libre
donde los prisioneros españoles representaron Otelo en honor de Lord
Cochrane. Aun en la época más laboriosa de su gestión
administrativa ---cuando, bajo el
gobierno de Ovalle, desempeñaba todas las carteras del Gabinete---
Portales no dejó de asistir por lo menos una vez a la
semana a las reuniones de la
Filarmónica. La concurrencia femenina era a base de señoritas de vida
decente, aunque no excesivamente recatadas, que gustaban de bailar al
son de arpas y guitarras. Entre ellas destacó Rosita Mueno, rutilante
belleza que dio tema a la chismografía local, y cuyo nombre anduvo
mezclado con el del Ministro. Es fama que éste no bebía, pero podía
estarse hasta las 12 de la noche ---límite de las trasnochadas de ese
entonces--- rasgueando la, guitarra o ``haciendo raya'' en el tablado. Por
algo declaró a sus partidarios políticos que no cambiaría la Presidencia
de la República por una zamacueca.

Iguales placeres solía procurarse hasta en la soledad del campo, allá en
su fundo de El Rayado, La Ligua,
donde su genio versátil le llevó a dedicarse a la agricultura. El
improvisado campesino vivía sin ver otras caras que la de su
servidumbre; pero a ciertos intervalos rompía este aislamiento para
entregarse a la desenfrenada
expansión de sus sentidos. Un volador lanzado desde el patio de la casa,
al anochecer, era la señal de que Su Señoría estaba de recepción e
invitaba a las niñas y galanes de la aldea. Tales fiestas eran de una
alegría estruendosa y degeneraban a veces en bacanales. Para
amenizarlas, el anfitrión hacía traer desde Valparaíso la banda de
músicos del cuerpo cívico, o contrataba a las más afamadas cantoras del
lugar. También había números de bufones, como un zapateador y una pareja
de idiotas cuya gracia consistía en trenzarse a moquetes hasta quedar
irreconocibles.

Al igual que los convites de la Filarmónica, los
de El Rayado eran un pretexto para
revistar y renovar el elenco femenil; y el cohete que volaba por el
cielo de La Ligua fue la luz que atrajo a más de una mariposa
desprevenida, que de allí salió con las alas chamuscadas. (Es sabido que
una de ellas echó al mundo otro Portalito ilegítimo.)

El hecho concreto es que Portales no podía prescindir ni por un momento
del sexo contrario. En una de sus clásicas confidencias se lee lo
siguiente:

``¿Sabe usted que la maldita ausencia de las señoras no me deja comer ni
dormir tranquilo? Examino mi conciencia y encuentro que las quiero del
mismísimo modo que el señor San José a Nuestra Señora la Virgen
Santísima.''

Las quería a todas, al conjunto, y en razón directa crecía su repulsión
a la idea de amarrarse con una. Por eso llegó a decir: ``El santo estado
del matrimonio es el santo estado de los tontos''.

Contra esa resolución tenaz ---y sin duda enfermiza--- de no volver a
entregarse, se estrelló durante catorce años la invariable solicitud de
Constanza Nordenflycht. Todos los recursos de la seducción y la ternura
no bastaron al propósito de doblegarlo. Bastará saber que el piano de la
casa de las Ramadas, a cuyo son bailaba don Diego con la Mueno y sus
otras amigas, se lo había proporcionado ella, para contribuir a su
diversión y saberlo contento\ldots{}

Sólo en una circunstancia llegó ese carácter tremendo a dar señales de
ablandarse, y fue ante la inminencia del fallecimiento de su querida.

En el invierno de 1832, Constanza contrajo la escarlatina, epidemia que
por primera vez se hacía sentir, y los médicos desesperaron de salvarla.
Desde Valparaíso, donde entonces residía como comerciante y gobernador,
Portales escribió a su agente en la capital, don Antonio Garfias,
encargándole casarlo por poder en artículo de muerte. Y este gesto
supremo no era por lástima de la moribunda, sino con objeto de legitimar
a sus hijos. Lo dice él mismo, una y otra vez: ``Formada mi firme
resolución de morir soltero\ldots ''; ``No tendría consuelo en la vida y me
desesperaría si me viera casado\ldots''

En ninguna de sus cartas ---y se conservan seiscientas--- expresó de
manera tan patética su horror al santo estado de los tontos. ``Me avanzo
a aconsejarle, si es posible'', le dice a Garfias, ``se case a mi
 nombre después de muerta la
consorte. Creo que no faltaría a su honradez consintiendo en un engaño
que a nadie perjudica y que va a
hacer bien a unas infelices e inocentes criaturas\ldots''

Contra todas las previsiones de los doctores, la
enferma desahuciada reaccionó y
recobró su salud, y el gobernador de Valparaíso se libró de tomar por
mujer a una difunta.

\chapter{LA TRAGEDIA DEL CAPITAN PADDOCK}

Ocupaba Portales el puesto de
gobernador de Valparaíso, en uno de. sus intervalos ministeriales,
cuando a mediados de diciembre de 1832 arribó a ese puerto la fragata
ballenera Catherine, de la matrícula de Nantucket. Mandaba este buque el
capitán Henry Paddock, de treinta y dos años, ``de honrada y apacible
figura'' al decir de la fama. Era un personaje de los que Melville
pintaría más tarde en Moby Dick, de aquellos que, en el curso de sus
expediciones, después de estarse hasta tres años sin ver la tierra,
solían venir a la costa de Chile para refrescar sus tripulantes,
convertidos en salvajes, y para carenar sus naves, carcomidas por la sal
y la broma. A Paddock no le traía primordialmente ninguno de estos
menesteres, sino la grave urgencia de enderezar su empresa, que
resultaba
hasta ese día un completo fracaso. Tras un fatigoso crucero de doce
meses, todo su botín consistía en unos doscientos barriles de aceite de
esperma, con cuyo producto no alcanzaría a pagar sus gastos. Para colmo,
sólo le quedaban víveres para dos semanas y no tenía dinero ni crédito
con que procurárselos, no conocía a nadie en Valparaíso ni hablaba el
idioma del país\ldots Nada habría sido esto si sólo fuese un empleado de
sus armadores; pero el caso es que fue él quien promovió la expedición y
bajo su directa responsabilidad se habían invertido los capitales,
prestados por una banking house norteamericana. La inminencia de la
quiebra había afectado su ánimo hasta hacerle caer en profunda
depresión. Lo que no hace sino probar que era un hombre de honor y que
la catástrofe de que sería hechor y víctima tuvo precisamente origen en
aquellos caballerosos escrúpulos, que acabaron por aniquilar su razón.

Ningún síntoma especial habían advertido sus compañeros; y fue él mismo
quien se encargó de ponerles sobre alarma, estando ya sujeto a su boya,
al confiarles que se sentía ``como poseído de una enfermedad que no sabía
definir''.

Sus averiguaciones en tierra lo llevaron a conectarse con una casa
compatriota, la de Alsop and Company, cuyo jefe, George Kern, le acogió
con buena voluntad. Pretensión de Paddock era que éste aviase el buque
para seguir hasta el Antártico ---pradera de los mayores cardúmenes de
cetáceos--- a fin de reponerse de sus pérdidas. El manager dejó la
proposición en estudio, pero accedió a pagar y alimentar a la dotación
de la fragata durante su estada en Valparaíso con la garantía de su
cargamento de aceite y de su provisión de tabaco, estimada en seis
quintales.

El marino se retiró dando muestras de alegría\ldots; pero llegó a bordo en
una disposición anímica
enteramente
opuesta, quejándose de que los Alsop se proponían esquilmarlo para
quedarse con el buque\ldots

Estuvo tres días sin salir de la cámara, alternativamente colérico y
meditabundo, mientras la manía persecutoria le iba progresando como una
gangrena. Llegó a manifestar su convicción de que sus ``enemigos''
tramaban contra su vida.

Al cuarto día, habiéndose en apariencia tranquilizado, volvió a
desembarcar; pero a su regreso, por la tarde, ya venía con las señales
inequívocas del trastorno mental. Llamando aparte a sus oficiales, presa
de gran agitación, les dijo que sus perseguidores le habían dado veneno;
y se tomó en su presencia unas píldoras que traía consigo. Visto el
cariz de su conducta, resolvieron el primer oficial y el sobrecargo
ponerle bajo vigilancia, encargando de ésta al mayordomo. El cual
declara en el proceso que aquella noche, al mirar al interior del
camarote por la claraboya, sorprendió al capitán bebiéndose a sorbos el
aceite de la lámpara.

Míster Kern habíale cobrado entretanto una honda simpatía. Informado de
las incidencias de la Catherine, fue a bordo para proponerle se
trasladase a su casa, ``donde podría descansar y estar protegido''.

Paddock se dejó llevar de mala gana, y en su obsesión de que estaba en
peligro echóse al bolsillo una navaja para la defensa de su persona.

Muy pronto y muy caro iba a pagar su amigo la imprudencia de haberle
permitido portar este adminículo.

El demente alcanzó a estar unas horas en el hogar de su benefactor,
desconfiando hasta de su esposa, que se desvivía por atenderle, y
promoviendo por último una escena de casa de orates al caer de rodillas
implorándoles que no le dejasen solo y que cerrasen puertas y ventanas.

Sin comprender, parece, la gravedad del enfermo, Kern prosiguió la
negociación iniciada cuatro días antes; y no bien recobró aquél su
serenidad salió en su compañía para dar remate a sus trámites.

A las tres de la tarde se encontraba en la oficina de Alsop (situada al
pie del barranco del Almendro); y el
capitán oía de labios del manager esta venturosa noticia: habíasele
concedido un pagaré cuyo monto bastaba para solucionar su conflicto.
¡Paddock, pues, estaba salvado! ¡La Catherine podía seguir al Antártico!

Inmediatamente, y a la vista suya, un empleado extendió el documento y
se lo pasó para que estampase en él su firma\ldots

Pero entonces, justamente entonces, sobrevino el desastre, Paddock dejó
el asiento, atacado de repentina convulsión de rabia, y saltando sobre
el oficinista le hundió la navaja en el corazón. La víctima se desplomó
sin una queja y quedó muerto a sus pies.

Esta escena, y las siguientes, pasaron con tal rapidez y en tal
confusión que no hubo dos testigos que coincidiesen en sus versiones.

Presa del pavor, Kern trató de ganar
la calle para huir o pedir socorro. Pero el loco, de otro salto, lo
alcanzó en la puerta y le asestó una puñalada en medio del pecho. Herido
de muerte, Kern pudo sin embargo seguir huyendo; pero el esfuerzo de la
carrera ayudó a desangrarlo y, al llegar a la plazuela de la Aduana,
rodó por el suelo y expiró.

Un gentío, entretanto, perseguía al matador dando voces de atajarlo y
arrojándole una lluvia de piedras. Manchado de sangre, los ojos fuera de
las órbitas, Paddock corría
sorteando carretas y caballerías y paralizando de pánico a la gente. En
un brusco viraje metióse en una casa de comercio, cuya puerta hizo
saltar de un empellón, e irrumpió en el escritorio repartiendo
cuchilladas. Encontrábanse allí el dueño del negocio, señor Squella; don
José Joaquín Larraín, marqués de Montepío; el comerciante Ramón Gallo,
un empleado y el portero. Antes de que
saliesen de su asombro había
ocurrido una carnicería: Larraín yacía muerto, Squella gravemente herido
y su empleado con la cara cruzada de tajos.

Saliendo de nuevo a la calle, la fiera humana se lanzó en dirección al
barrio portuario. Al pasar ante la Capitanía dos jornaleros intentaron
detenerlo, pero ahí quedaron tendidos: uno apuñalado en un brazo y el
otro en una pierna.

Poco más allá caminaba un oficial de marina mercante: nada menos que el
capitán William Wheelwright, futuro pionero de la navegación a vapor.
Verlo y echársele encima fueron para Paddock una sola cosa. El agredido
alcanzó a ponerse en guardia y paró cinco o seis navajazos con las
manos, Que le quedaron despedazadas; pero no pudo esquivar un corte
profundo cerca del corazón, que le hizo caer desvanecido.

A los gritos de los perseguidores, dos lancheros
 que salían del Resguardo corrieron
a atajar al alienado. El primero, que lo embistió de frente, recibió una
herida en el cuello, a cuya consecuencia murió una hora después. El otro
tuvo la inspiración de golpearlo por la espalda, arrojándole una piedra
a la nuca. Paddock se desplomó aturdido, todavía echando espumarajos; y
allí mismo fue amarrado de pies y manos para ser conducido a la cárcel.

En un par de minutos había dejado cuatro muertos. y seis heridos, como
si un toro bravo hubiese pasado embistiendo por las calles.

Una
hora más tarde, el ex cazador de ballenas comparecía ante el juez Fermín
Rojas para prestar declaración. Había recobrado su equilibrio y su
``honrada y apacible figura''. Sea por efecto del aturdimiento, o por
causa de su propia enfermedad, no tenía el menor recuerdo de lo que
acababa de ocurrir. Al describírsele el reguero de sangre que dejara
tras de sí, reaccionó como ante una enorme calumnia. Lo
único que sabía es que alguien lo
había atacado en el Resguardo rompiéndole el cráneo con una piedra.
``Ignora la causa de su prisión'', dice la diligencia judicial, ``y cree
que se halla preso por petición de ciertas personas de la casa Alsop,
que se habrían complotado para quitarle la vida''.

A la luz de la medicina legal moderna, sus actos fueron los de un
irresponsable y él estuvo tan limpio de culpa como un niño
dormido\ldots{}

Para desgracia suya, no era éste el parecer del gobernador Portales,
quien dictaminó que se trataba de un asesino alevoso y que como tal
debía ser castigado. ``Yo aseguro'', dijo y repitió por escrito, ``que el
reo no está loco'', Debía, pues, ajusticiársele, y en este predicamento
recomendó al juez la mayor prisa en la substanciación del proceso. Al
hacerlo invocó razones no menos
discutibles que la primera, tales como ``el alboroto de la plebe, la
altanería e impunidad de los extranjeros y el supuesto cohecho que la
justicia habría recibido para absolver al reo por su peso en oro''.

En otras palabras, que Paddock debía morir de todas maneras, estuviese o
no en su juicio, fuese o no un criminal, porque el gobernador de
Valparaíso así lo quería.

Inútiles fueron, por consiguiente, los esfuerzos que se hicieron por
salvar al infeliz. El informe del cirujano de ciudad, doctor Leighton,
declarándolo enfermo; la defensa del abogado de turno, los artículos de
don Andrés Bello en El Araucano, los memoriales del encargado de
negocios de los Estados Unidos pidiendo una investigación ecuánime, todo
rebotó contra la decisión del hombre que ``mandaba
a los que mandaban'' y cuyo poder
omnímodo, incontrarrestable, constituye un misterio todavía no aclarado
por la historia.

El 21 de diciembre, a las ocho de la noche ---horas después de la
captura del ''asesino''---, ya el fiscal había pedido la pena de muerte.
A las dos de la tarde del día siguiente Paddock estaba condenado a ser
fusilado y colgado en un sitio público. El 24, la
Corte de Apelaciones de Santiago
confirmaba la sentencia; y el 12 de enero levantábase el patíbulo en el
cabezal del muelle de carga ---el punto más visible de la ciudad--- al
pie de la grúa mayor, que haría las veces de horca.

Todo esto, como lo hizo notar Bello, sin que se conociesen los
fundamentos de la sentencia, porque la ley de entonces no obligaba a los
jueces a consignarlos; de manera que los contemporáneos no supieron en
rigor cuáles fueron los motivos o los pretextos que determinaron el
fallo.

Ante la suprema notificación, Paddock no manifestó rebeldía ni miedo,
como si su suerte le fuese indiferente. Pero el colmo de su tragedia es
que no lograra hacer memoria y que hasta los últimos momentos creyese
que lo sacaban de este mundo por una intriga de Kern, al que seguía
creyendo vivo.

Pasó sus postreros días leyendo la
Biblia con una conformidad que fue la prueba final de su demencia.

Lleváronlo al lugar del suplicio en una silla, amarrado al respaldo y
sólidamente maniatado. No soltó por ello el libro santo, que leyó y
sostuvo entre sus manos hasta el instante de la descarga.

Conforme fuera prescrito, colgaron el cadáver del gancho de la grúa, y
durante veinticuatro horas se le dejó expuesto a las miradas de los
curiosos.

\chapter{``ESTA SERVIDO SU MATE, MISTER DARWIN''}

Una noche en que Darwin descansaba cerca de Tagua-Tagua con su guía,
Mariano González, oyó a dos individuos comentar su presencia en Chile en
tono cargado de suspicacia:

---¿Qué piensa usted de ese señor cuya única ocupación es buscar
escarabajos y lagartos y partir piedras?

---Yo creo que aquí hay gato
encerrado. Nadie es bastante rico para gastar tanta plata en cosas tan
inútiles\ldots Si nosotros mandáramos a Inglaterra a alguien que hiciera lo
mismo, estoy seguro de que el rey lo expulsaba por sospechoso.

Uno de los interlocutores era el notario de San Fernando, lo que da una
idea del nivel intelectual de la sociedad chilena en los -días del
gobierno de Prieto, cuando la obra culturizante de Portales, Bello y Gay
no empezaba aún a fructificar.

El sospechoso Darwin, de veinticinco
años de edad, había llegado a fines de julio de 1834, pasajero del
Beagle, bergantín de 230 toneladas que daba la vuelta al globo en misión
hidrográfica y científica al mando del capitán FitzRoy. Fruto de este
viaje memorable iba a ser la teoría del Origen de las Especies con que
el oscuro egresado del Cambridge deslumbraría al mundo de la ciencia.
Instalado en una sección de la
estrecha cámara del comandante, sin goce de sueldo y constantemente
mareado y enfermo, cumplía su
cometido con dedicación imperturbable, estudiando desde la geología de
los lugares de desembarco hasta su flora y su fauna, la climatología, la
zoología marina, los conchales de las costas, las costumbres de los
aborígenes, las ruinas de piedra, la mineralogía y los fósiles, que eran
para él como documentos de los archivos de la prehistoria. De esta
circunnavegación de la Tierra, que tomaría casi cinco años de su vida,
quedó constancia en sus trabajos voluminosos, en el cargamento de
minerales, conchas, herbarios y pájaros embalsamados que remitió al
Museo Británico, y en las seiscientas páginas del \emph{Voyage of a
naturalist round the world}, libro apasionante y el primero que hay que
leer para informarse de cómo era Chile en el momento en que se estaba
estructurando la República. (De paso anotemos que todo
cuanto refiere de su gobierno es la
política adoptada con los araucanos,
a cuyos caciques pagaban un sueldo para que se estuviesen tranquilos\ldots)

Viniendo directamente desde la Tierra del Fuego ---donde Jemmy Button y
otros dos yaganes fueron desembarcados después de recibir en Inglaterra
un barniz de civilización--- el Beagle fondeó de noche
en Valparaíso, de suerte que su
acogedor panorama de amanecida surgió ante los ojos de Darwin en
violento contraste con la desolación del salvaje Chile
austral. La ciudad le hizo recordar
a Santa Cruz de Tenerife, aunque no consistía más que en una calle
de casas blanqueadas con cal y
cubiertas de tejas, cuya fila sólo se ensanchaba en los barrancos
rojizos que separaban los cerros desprovistos de arboledas.
Este cuadro pintoresco arrancó a su
pluma un grito de alegría exultante:
``¡Qué cambio!, ¡cuán delicioso nos parece todo aquí; tan transparente es
la atmósfera, tan puro y azul es el cielo, tanto brilla el sol, tanta
vida parece rebosar la Naturaleza!'' Desde el centro de la bahía
divisábase el lejano Aconcagua con
su poncho de nieve, y a tal
distancia las triangulaciones de los oficiales de FitzRoy establecieron
su exacta altitud: 6.900 metros sobre el mar. Era el Beagle la nave de
investigación mejor equipada que la marina británica hubiese puesto a
flote; bastará saber que traía veintidós cronómetros y que aparte de
descubrir el canal fueguino al que dio su nombre, las cartas de nuestras
costas y bahías levantadas por sus hidrógrafos hasta hoy se admiran por
su precisión rigurosa.

Aprovechando la escala, que sería
larga, el naturalista desembarcó con sus mantas y botas de montar, su
instrumental científico y sus redecillas para coger mariposas, y fuese a
tierra como chupado por una corriente de aire. Con haber permanecido
allí  tres semanas, no dejó ni una
frase descriptiva del ameno villorrio porteño, cual si no hubiese tenido
tiempo de mirar las calles polvorientas atiborradas de jinetes y
carretas entoldadas, los balcones coloniales adornados de tachos con
flores y los burritos dedicados al acarreo del agua de las quebradas.
Guiado por el baqueano González, galopó a lo largo del camino de
Quintero, cruzando ese país seco ``donde hasta los zarzales son clorosos,
pues sólo de atravesarlos queda la ropa perfumada''. Pernoctó en la
histórica hacienda otrora perteneciente a Lord Cochrane, que no le
interesaba por eso sino por los famosos conchales de que estaba sembrado
el suelo hasta una altura de cientos de metros sobre el nivel del mar.
Sospechando que esta comarca, siglos atrás,
pudiese haber sido el lecho de una
bahía, examinó con el microscopio la capa vegetal, de un intrigante
color negro rojizo, y comprobó su formación marina por la multitud de
partículas de cuerpos orgánicos de que estaba llena. Esta convicción se
afirmó en la mente del sabio al recorrer el valle de Quillota, ubérrima
vega de pastizales y frutales encajonada entre cerros estériles y cuyo
panorama le indujo a escribir:

``Estas hoyas o llanuras\ldots, estoy persuadido de ello,
son el fondo de antiguas bahías
semejantes a las que hoy día recortan tan profundamente la Tierra del
Fuego y las costas de más al sur. Antiguamente Chile debió parecerse a
aquella región por la distribución de las tierras y las aguas''. No se
cansa de alabar la riqueza potencial de esta zona privilegiada para la
agricultura, viticultura y ganadería, anotando que con tales dones sus
habitantes ``deberían disfrutar de más prosperidad de la que en realidad
disfrutan\ldots'' Después de mudar cabalgadura y de tomar como guías a dos
huasos de la zona, inició la ascensión del cerro de la Campana. A 1.350
metros se detuvo admirado para examinar la palma chilensis, la palmera
aclimatada a mayor altura en el mundo; árbol poco elegante con su grueso
tronco abotellado pero capaz de
entregar hasta cuatrocientos litros de exquisita
miel. Cerca de la vertiente que
llaman del Guanaco hicieron alto para pasar la noche al reparo de una
ramada de bambúes. Los guías prepararon charqui de buey y mate humeante
en una fogata, mientras Darwin, indiferente al frío de agosto, de la
montaña y del crepúsculo,
contemplaba desde cuarenta kilómetros la rada de Valparaíso, donde eran
visibles las arboladuras de los navíos y el velamen de uno que iba
doblando la punta de Curaumilla. Al ponerse el sol en el océano, ``los
valles quedan sumidos en la sombra, en tanto que los picos de los Andes,
cubiertos de nieve, se encienden de tintes rosados\ldots Hay un encanto
inexpresable en vivir así a pleno aire. La velada transcurre en perfecta
calma; no se oye sino de tarde en tarde el agudo grito de la vizcacha
montañesa o la nota quejumbrosa del chotacabras''.

A la mañana siguiente el explorador siguió subiendo hasta alcanzar la
cumbre de la Campana, a casi dos mil metros de altitud. Tanto como el
panorama de Chile en toda su anchura le sorprendió el aspecto de la cima
del cerro, cuyos inmensos bloques de asperón aparecían rotos por mil
partiduras y grietas, algunas tan limpias cual si se hubiesen producido
el día anterior. Supuso que serían
la obra de los terremotos, pero esta hipótesis iba a tener que
desecharla más tarde en Tasmania, país donde no tiembla, y en cuyo monte
Wellington encontraría igual coronamiento de peñascos resquebrajados.
Aparte los enigmáticos destrozos naturales, la Campana estaba horadada
como un queso de bola por los incontables piques de sus minas de oro;
riqueza secular que, con la plata, el carbón, el hierro, el nitrato, el
petróleo y el cobre no han podido sacar al país de su sempiterna y
desesperante pobreza.

Pasó Darwin una segunda noche en la ramada de bambú, y a propósito de
sus compañeros anotó la impresión que le producía la idiosincrasia de
estos seres tan exóticos para un
inglés como los malayos pueden serlo para nosotros. ``Los huasos'', dice,
``corresponden a los gauchos de las pampas argentinas, pero son en
definitiva por completo distintos\ldots Las diferencias de rango están aquí
mucho más marcadas: el huaso no considera a todos los hombres como sus
iguales, y he quedado muy
sorprendido al ver que mis
acompañantes no gustaban de comer al
mismo tiempo que yo\ldots El viajero no encuentra aquí esa hospitalidad sin
límites que rechaza todo pago y que es ofrecida tan cortésmente que
puede ser aceptada sin escrúpulo. En Chile, en casi todas partes, se os
recibe por la noche, pero con la esperanza de que algo entregaréis al
partir\ldots El gaucho, en toda circunstancia, es un gentleman; el huaso,
siendo preferible bajo ciertos aspectos, y sin dejar nunca de ser un
hombre trabajador, es vulgar''.

Continuando su excursión, el sabio pasó por Quillota, que le pareció una
sucesión de floridas chacras más que una ciudad; y después de atravesar
San Felipe, ``\emph{a pretty little town}'', fue a detenerse en una
mina de cobre que cierto inglés de
Cornuailles explotaba sobre el barranco de un cerro de Jahuel. Mina sin
chimeneas ni máquinas, porque el mineral se procesaba en Inglaterra,
adonde era despachado en recuas de mulas y luego en veleros \emph{round
the Horn}. La legislación del país permitía a cualquiera trabajar la
veta de que fuese descubridor, previo pago de un módico derecho y aunque
ella estuviese en el huerto del vecino. Sea por ignorancia o por la alta
ley de las vetas, el industrial común desechaba las piritas, como si no
contuviesen ni una pizca de cobre, haciendo burla de los ingleses
dedicados a exportar esa ``basura'' y hasta las escorias recogidas de los
pequeños hornos. A falta de cañerías, el agua requerida por la faena y
campamento de Jahuel era transportada al
 hombro en odres de cuero. El minero
trabajaba de sol a sol, percibiendo un jornal equivalente a veinticinco
chelines mensuales y comiendo al desayuno docena y media de higos y dos
trozos de pan, como almuerzo habas cocidas, y por la noche trigo
macha cado y tostado y alguna vez
una ración de charqui\ldots

Cercado
por las tormentas de nieve, el naturalista sólo pudo despedirse al cabo
de cinco días, para proseguir a Santiago con sus auxiliares y mulas de
equipaje. Insistiendo en su idea inamovible, escribiría que para llegar
a la capital atravesó ``una llanura que a primera vista se reconoce que
representa un antiguo mar interior''. Presumiblemente cruzó el Mapocho
por el puente de Cal y Canto, que era el monumento de la ciudad, pero no
lo menciona siquiera; y todo lo que apunta es que pasó aquí ``una semana
muy agradable'', cenando con los comerciantes ingleses, que la
\emph{city} ``no es tan bella ni tan grande como Buenos Aires'', y que el
panorama desde el roquerío del Santa Lucía era ``muy bonito''. En total,
trece líneas; cierto es también que nuestra metrópoli era entonces menos
que un suburbio de Londres. Y ni una palabra acerca de los logros ya
evidentes y notables del régimen portaliano, que en apenas cuatro
años había transformado a Chile en
la nación modelo de la América española. Silencio tanto más inexplicable
cuanto -que el viajero estuvo más de una vez en palacio en demanda del
pasaporte especial que el Presidente Prieto le concedió para su
proyectada visita a Mendoza.

La pluma que pasara por alto al orgulloso Calicanto, detiénese a
contarnos cómo eran los puentes colgantes del río Maipo, uno de los
cuales debió cruzar de paso para las termas de Cauquenes. ``¡Triste
cosa son esos puentes!'' exclama con
horror de europeo delante de
aquellas pasarelas de cimbra introducidas por los constructores
incaicos. Y la descripción que les dedica es pieza documental para
el estudioso de la historia de la
ingeniería en el país. ``El tablero o piso'', dice, ``se presta a todos los
movimientos de las cuerdas que lo sostienen y consiste en tablones
colocados uno al lado de los otros; a cada paso se encuentran aberturas
o boquetes, y con el peso de un hombre que conduce su caballo de la
brida, todo el puente se bambolea de un modo espantoso''. Así tuvo que
atravesarlo él mismo, haciendo de tripas corazón, mientras escuchaba el
crujir de los tirantes de cuero y veía. desde esa altura de vértigo la
fragorosa correntada del Maipo.

Puentes similares colgaban sobre el Cachapoal,
pero en la estación en que Darwin
subió a Cauquenes ---comienzos de septiembre--- el disminuido caudal del
río permitía a los jinetes pasarlo por sus rugientes vados de lecho
pedregoso. Los renombrados baños termales tenían entonces por toda
hospedería un conjunto de míseras chozas sin otro menaje que una mesa y
un banco\ldots Conversando con el guardián del establecimiento, supo Darwin
que a raíz del terremoto de 1822 las fuentes de agua caliente quedaron
secas y no reaparecieron hasta un año después. (Y el propio autor
refiere que el sismo de Concepción, cuyas ruinas le tocaría ver de ahí a
poco, modificó la temperatura de las aguas de Cauquenes, haciéndolas
descender desde 47,7 hasta 33,3 grados centígrados. Lo cual autoriza a
deducir que en esta relación entre los terremotos y las vertientes
subterráneas podría encontrarse una pista para la moderna ciencia
encaminada a predecir los fenómenos tectónicos.)

Otra cosa que el guardián contó a Darwin, para acabar de inquietarlo,
fue que por ese valle del Cachapoal ``se dejaba caer'' la banda de
salteadores de los Pincheira cuando invadía los fundos y pueblos de la
zona para robar ganado y mujeres. Pequeño ejército inubicable, buscado
desde Chile y Argentina en los mil vericuetos de la cordillera y que
Pincheira movilizaba con infalible agilidad, pues era fama que
seleccionaba a sus jinetes volándole la cabeza a todo el que no era
capaz de seguirle.

Su gira por el Valle Central llevó al sabio hasta San Fernando, desde
donde se desvió en demanda de la mina de oro de Yáquil. Desde una colina
divisó la célebre laguna de Tagua-Tagua, que también describiría Claudio
Gay, y donde el viento llevaba a la deriva islotes formados por plantas
muertas enmarañadas, de cerca de dos metros de espesor, ``transportando a
veces caballos y, vacunos a guisa de pasajeros''. La mina de Yáquil era
explotada por un norteamericano de apellido Nixon e imperaba en ella una
modalidad de trabajo que hacía a los peones envidiar la suerte de sus
cofrades de Jahuel. Por el pique de ciento y tantos metros de
profundidad bajaban los apires, muchachos de dieciocho a veinte años,
semidesnudos (y cuya palidez impresionó al
visitante), para volver a la
bocamina trayendo a la espalda, en una bolsa de cuero, noventa kilos de
piedras de mineral. ``Con esa carga el minero debe trepar por
entalladuras hechas de troncos de árboles dispuestos en zigzag\ldots Un
hombre vigoroso, no habituado a esa labor, tiene bastante trabajo para
poder izar su propio cuerpo y llega a la superficie bañado en sudor. A
pesar de tan dura faena, estos mineros son alimentados exclusivamente
con habas hervidas y pan. Ellos prefieren el pan a secas, pero sus
patrones, sabiendo que tal alimento no les permite un esfuerzo tan
sostenido, los tratan como a caballos y les obligan a comerse las
habas\ldots No abandonan la mina sino
una vez cada tres semanas para pasar dos días en su casa''.

Y con todo, estos galeotes del subsuelo eran a su vez envidiados por los
campesinos, de cuya suerte dice el
autor del \emph{Viaje de un naturalista alrededor del mundo}: ``El
propietario da al inquilino un pequeño lote de tierra ---en el cual éste
debe construir su habitación--- para que lo cultive; pero a cambio de
eso, el campesino ha de proporcionar su trabajo en el fundo\ldots durante
toda su vida, a diario y sin jornal. No tiene a nadie que pueda cultivar
su terreno hasta que cuente con un hijo para reemplazarle en el trabajo
que debe cumplir al patrón. No hay, pues, que asombrarse de la extrema
pobreza que reina entre los obreros de este país''.

Fue mientras descansaba en casa de Nixon que Darwin oyó los comentarios
acerca del ``gato encerrado'' que él disimularía con su recolección de
mariposas, arañas, lagartos y matapiojos mientras husmeaba, a lo mejor,
por cuenta del rey de Inglaterra o los Pincheira\ldots

Pero el espía de la ciencia había dado por terminada su correría y se
alejó, dejando a los suspicaces tranquilos, para ir a reembarcarse en el
Beagle y salir con destino a Chiloé.

\chapter{DARWIN ENTRE LAS RUINAS DE CONCEPCION}

De regreso de su exploración a Chiloé, el Beagle hallábase al ancla en
la ría de Valdivia a fines de febrero de 1835. Los días que Fitz Roy
dedicó a sus trabajos hidrográficos los empleó Darwin en recorrer los
fuertes históricos y los -compactos bosques, toda, vía no arrasados por
la barbarie nacional. A eso de las once y media de la mañana del 20 el
sabio se sentó a la sombra de una araucaria para descansar. Apenas
habían transcurrido unos minutos cuando un ruido siniestro llenó el
ámbito de la selva y luego la tierra empezó a estremecerse. Siendo los
temblores casi desconocidos para él, el naturalista juzgó el fenómeno
\emph{very interesting}. Pronto advirtió que se trataba de un cabal
terremoto. En medio del espantoso tronar subterráneo el suelo ondulaba
como un oleaje y las copas de los árboles se azotaban unas con otras. Un
poco mareado, pero imperturbable, el hombre de ciencia observaba reloj y
brújula en mano. Las ondulaciones venían del este, es decir, del lado de
los volcanes; y cuando cesaron, verificó que había durado casi dos
minutos.

Con sorpresa supo más tarde que Valdivia seguía en pie. Sus casas de
madera habían resistido bien y los daños eran comparativamente
insignificantes. Debió pensar que los célebres sismos de Chile eran más
ruido que nueces. Pero si tal creyó, no tardaría en mudar de opinión.
Valdivia no había sido el epicentro y lo que se sintió en su zona fue
apenas el eco de una conmoción incomparablemente más violenta, que
descargó su máxima fuerza trescientos kilómetros al norte.

Los viajeros del Beagle vinieron a saberlo doce días después y en los
lugares mismos del desastre. Fue al fondear ante la isla Quiriquina, el.
4 de marzo, que Fitz Roy tuvo las primeras noticias. El terremoto, al
que siguió un maremoto, le fue descrito como ``el más terrible que jamás
se produjera en el país''. Supo que ni una casa quedaba sobre sus
cimientos en Talcahuano, y Concepción, que setenta aldeas y pueblos
habían corrido igual suerte, y que trescientos temblores menores habían
seguido al primero en esos doce días. Sin mirar fuera de la Quiriquina
podía tenerse una idea de la magnitud del azote. La isla exhibía ahora
tres pies de sobreelevación y de trecho en trecho, estaba cortada por
grietas de un metro de ancho. En la orilla se veían hacinamientos de
rocas partidas, como si las hubiesen volado con cargas de explosivos, y
muchas de ellas mostraban adherencias de algas y 'moluscos prueba de que
habían sido proyectadas desde el lecho del mar. Confesó Darwin que hasta
entonces nunca tuvo ante los ojos demostración tan patente del poder
destructivo de la naturaleza. ``Desdé que salimos de Inglaterra'', declara
en el Viaje de un naturalista, ``no habíamos contemplado un espectáculo
tan interesante como aquél''.

Al desembarcar. comprobó que la población de Talcahuano yacía en ruinas.
No había otras viviendas que los ranchos y barracas levantadas con el
maderamen de los demolidos edificios. La playa aparecía sembrada de
muebles, techumbres y restos de paredes, que el mar arrastró en su
retirada. Entre los escombros de los almacenes de aduana estaban
diseminados los cargamentos de trigo; de harina y de lanas, revueltos en
los charcos de la inundación. Embarcaciones despanzurradas y restos
náufragos flotaban aún en la bahía, mientras que doscientos metros
tierra adentro una goleta de tres palos permanecía recostada contra una
pirca\ldots

Míster Rouse, el cónsul inglés, condujo a sus compatriotas por el
anegado camino de Concepción en tanto que iba haciéndoles el relato de
la catástrofe. El número de muertos, en ambas ciudades, montaba a cien,
y el de heridos a quinientos. Esta baja cifra de víctimas debíase al
hecho de haber venido el terremoto en pleno día. Coincidiendo con la
información oficial del Intendente Boza, decía Rouse que el movimiento
sorprendió a la gente almorzando y que su anuncio fue un verdadero.
bramido subterráneo. Hubo el tiempo justo para lanzarse fuera de las
casas, porque dos segundos después la corteza terrestre se sacudía como
el lomo de un caballo encabritado. La polvareda de los derrumbes
obscureció el sol. En los puntos bajos de la ciudad el suelo se abrió en
tajos que dieron salida a los pestilentes gases de los pantanos. Los
remezones alcanzaron tal grado de violencia que fue imposible mantenerse
en pie. El cónsul confesaba que, olvidado de la dignidad de su cargo,
corría a gatas esquivando la lluvia de adobes y cornisas.

Aquello duró tres minutos. Pudo verse, entre las nubes de polvo, que el
cincuenta por ciento de las construcciones estaba en tierra.

Entonces, con intervalo de pocos instantes, se produjo el maremoto.
Rouse no lo vio, pero una sensación especial le hizo comprender que algo
tremendo ocurría en la costa.

Los vecinos de Talcahuano observaron que una ola de tamaño descomunal y
desprovista de espuma avanzaba a distancia de tres a cuatro millas en
dirección al seno de la bahía. Su volumen y velocidad crecían de manera
aterradora. Al llegar a la orilla su altura sobrepasaba los siete metros
y engendraba un viento huracanado. Su horroroso impacto barrió con las
instalaciones portuarias, ya medio destruidas, inundó completamente la
población baja y tuvo en tierra y mar los efectos propios de un temporal
de fuerza máxima. Dos buques mercantes hicieron colisión, reventando sus
cascos de madera, otro fue a dar a tierra y luego volvió a entrar al
agua, mientras la goleta de tres mástiles iba a posarse dos cuadras al
interior. Un cañón de cuatro toneladas fue arrancado dc su emplazamiento
en el fuerte y animales que pastaban en las colinas rodaron hacia el mar
y perecieron entre las rocas.

Cumplida su tarea, la ola se retiró con el botín de media ciudad:
techos, camas, tabiques, toneles, bestias y seles humanos. También se
llevó consigo las aguas de la poza, dejando en seco dos navíos que
estaban anclados en nueve brazas.

Como si esto aún no fuese bastante, la embestida se repitió por dos
veces consecutivas, barriendo la demolición que dejara la primera.

En esos momentos comenzó a temblar de nuevo y con aumentada aceleración.
El mar se puso negro e hirvió como si fuese de alquitrán. Del centro de
la rada surgieron dos erupciones, una parecida a una columna de vapor,
la otra semejante al chorro de una ballena. Todo el fondo debía estar
convulsionado, porque saltaban piedras y pedazos de roca y un fétido
olor a fango descompuesto infestaba el aire.

Para Concepción esta segunda sacudida fue el golpe de gracia. Ni una
sola de sus construcciones pudo resistirla. La última en caer fue la
Catedral, orgullo de la ciudad, cuya mole había estado bamboleándose
como en un supremo esfuerzo por sostenerse. Al ceder sus murallas, el
techo se hundió, aplastando cuanto había abajo, excepto la Virgen del
altar mayor, que quedó intacta y en pie. Las torres, con sus campanas
repicando enloquecidas, cayeron sobre la plaza en medio de un pavoroso
estruendo. La multitud, ya fuera de sí, huyó hacia el cerro Caracol en
busca de refugio. Creyendo que ése era el fin del mundo, hombres y
mujeres confesaban sus culpas a gritos, en tanto que a la polvareda de
los derrumbes se mezclaba el humo de los incendios.

No cesó de temblar en los días siguientes; y por último una lluvia
torrencial cayó sobre las ruinas, para que las últimas paredes
resquebrajadas acabasen de desplomarse\ldots

Haya sido por mera casualidad o por un designio histórico, Darwin y Fitz
Roy llegaban a tiempo: el primero para ilustrar al mundo de lo que él
mismo llamó ``uno de los mayores acontecimientos sismológicos de la era
moderna'', y el otro para registrar en las cartas hidrográficas la nueva
profundidad de estos mares y la altura de sus costas removidas por la
catástrofe. Es de presumir el hondo interés con que ambos recorrerían la
zona castigada. Darwin declara no haber tenido un campo de observación
comparable en ninguna otra etapa del viaje. Su mirada de investigador
halló motivo de estudio en cada faceta del drama, y su pluma supo
expresarlo en el más vívido capítulo de su narración.

Lo que ante todo parece haberle sorprendido es el estado anímico de los
habitantes. Allí donde esperó ver una muchedumbre desesperada o
extenuada, encontró una comunidad de familias de distinto rango social
haciendo vida de camping en las laderas de un cerro pintoresco. No sólo
había resignación; reinaban el buen humor y la alegría. Las dueñas de
casa se afanaban en sus nuevos hogares ---ranchos y reparos de
tablas---; los caballeros atendían sus negocios en escritorios
improvisados sobre cajones o bateas de lavar; los niños jugaban a los
buques de papel en las pozas de las inundaciones\ldots La primera impresión
del sabio fue que aquella gente había perdido el juicio; pero no tardó
en penetrar el problema, nuevo para él, y dar con su curioso mecanismo
psicológico. El hombre reacciona ante la desgracia según sea la
situación en que ésta le deje con respecto a sus prójimos. Si queda él
solo damnificado, sufre el infortunio porque se sabe en condición de
inferioridad; pero si todos los que le rodean comparten su ruina,
entonces ya no la siente, pues no tiene punto de referencia para
medirla, y en cierto modo ella desaparece. Tal ocurría en Concepción:
nadie se consideraba en desgracia porque la desgracia era general y
pareja.

Existía, por otra parte, la creencia de que los terremotos habían
asolado el país entero, y la metrópoli del Sur se consolaba imaginando
que la opulenta capital también debía estar demolida\ldots Pero la ``mala''
noticia llegó al fin, de buena fuente y confirmada: en Santiago no había
temblado siquiera. El sismo habíase hecho sentir hasta Chillán por el
norte y hasta Chiloé por el sur. La ciudad natal de O'Higgins fue
borrada del mapa. Desde la víspera de la Ruina, como se llamó a ese
ensayo de cataclismo, habían estado en erupción los volcanes Aconcagua
(cosa muy rara), Corcovado y Osorno. En la isla de Juan Fernández
reventó un cráter submarino y la marejada consiguiente anegó la calle
del poblado.

Paseando por el campamento del cerro Caracol, Darwin se entretuvo
escuchando a la gente del pueblo explicar a su modo el fenómeno. Este
habría sido la obra de hechiceros araucanos, que cerraron la válvula del
volcán Antuco en venganza por el ultraje que ciertos huincas infligieran
a unas ancianas mapuches. La superstición estaba emparentada con la
realidad científica, porque los terremotos tuvieron su origen en la
obstrucción de algún escape de las materias volcánicas. Esta fue la
opinión que Darwin se formó entonces y que estampó en su bitácora
cuando, tres días después, regresó a bordo.

``Creo, por muchas razones'', dice, ``que los temblores en esta línea de la
costa provienen del desgarramiento de las capas terrestres, como
consecuencia de la tensión de la tierra en el momento de los
levantamientos y de su inyección de rocas en estado líquido.
Probablemente existe allí un lago de lava subterráneo, de doble
superficie que el Mar del Norte. La relación íntima de la erupción y del
levantamiento durante esos temblores nos prueba que las fuerzas que
levantan gradualmente los continentes son idénticas a las que hacen
surgir las materias volcánicas por ciertos orificios.''

De esta hipótesis darwiniana podría desprenderse que las erupciones
amortiguan los temblores en vez de agravarlos, como se cree. En otras
palabras, que la intensidad del sacudimiento está en relación con la
cantidad de lava que no encuentra salida al exterior.

Nada refleja mejor la magnitud de la catástrofe que las líneas finales
de la comunicación del Intendente Boza al Gobierno: ``\ldots{}la ruina
es completa. El horror ha sido espantoso. No hay esperanzas para
Concepción. Las familias andan errantes y fugitivas; no hay albergue
seguro que las esconda; todo ha concluido; nuestro siglo no ha visto una
ruina tan excesiva y tan completa''.

Dos meses después, cuando el Beagle ya estaba lejos, aún seguía
temblando en las márgenes del Bío-Bío. Darwin habíase llevado la
convicción de que estas ciudades no volverían a levantarse, creyendo su
suelo condenado a periódicas convulsiones\ldots Pero tanto el sabio como el
Intendente estaban en el error: Concepción y su puerto tenían vitalidad
suficiente para rehacerse y valentía bastante para encarar su destino, y
de entre las pilas de escombros empezaban a surgir los cimientos de su
edificación futura, más sólida y más hermosa que la que acababa de
desaparecer.

\chapter{EL POETA DIEGO PORTALES}

Entre los papeles de Portales se conservan unos versos que compuso para
su novia:

\begin{verse}
Las bellas flores que su aroma exhalan\\
Con sus matices causan mis enojos;\\
No me divierten, porque no se igualan,\\
Bella, a tus ojos\ldots\\
Ni claro arroyo que de peñas duras\\
Brota cristales y a beber provoca,\\
Porque sus aguas no serán tan puras\\
Como tu boca\ldots
\end{verse}

La idea que se tiene de don Diego es la de un carácter contradictorio y
pintoresco: estadista férreo que cultivaba el humorismo y bailaba la
zamacueca, líder político que no hacía discursos y Ministro que no
cobraba sus sueldos. Lo que ignoran casi todos es que este hombre
singular fue también un poeta ocasional. La verdad es que resulta
difícil imaginárselo entregado a las expansiones sentimentales. Su
espíritu zumbón y -realista y su estilo procaz parecen irreconciliables
con la inspiración- romántica, y lo dice él mismo en su Epistolario: ``Mi
genio j\ldots{} no es el mejor para expresar afectos''.

Lo cierto es que escribió poemas de vibrante exaltación amorosa; y
sabiendo lo que fue su carrera de amador, podemos suponer que su
producción no ha sido corta.

Tenemos pruebas fehacientes de que la poesía fue una amiga constante de
este caballero solitario, viudo empecinado que llevaba el frac negro
como el pingüino su tenida antártica. Escribiéndole a Cea, su socio,
desde el Perú, le cuenta que ha estado leyendo a Ovidio. En carta a
Garfias, su corresponsal en Santiago, le anuncia desde Valparaíso, en
1834, el envío de unas estrofas de Pope, ``que me han gustado mucho''.

De paso recordemos que la literatura y la cultura general de Portales
eran las de un autodidacto. Hijo de un padre que tuvo veintitrés niños,
don Diego aprendió a leer con su madre ---tal debía ser la cortedad de
recursos de la familia---, y por ahí confiesa: ``No tuve otra escuela''.

Su biógrafo Encina inventó el mito de que su estirpe entroncaba con la
de los Borgia, y en esa ascendencia tremenda se ha querido ver la clave
de su temperamento excepcional, hecho de encontrados fuegos místicos y
sensuales.

Este catador de mujeres sólo amó realmente a la que fue su esposa: su
prima Chepita Portales y Larraín. Ninguna de sus incontables queridas
logró hacerle olvidar aquel amor supremo, que él recordaba como ``una
dicha infinita''. Hay razones para creer que no estimaba grandemente a la
porción femenina de la humanidad. En una de sus cartas dice: ``Vivir con
mujeres es una broma''. En otro lugar expresa: ``Nunca se incomode usted
con mujeres, porque yerran en cualquiera cosa que no sea su costura, su
canto y las demás ocupaciones de su sexo''.

Como se ha dicho en páginas anteriores, las pasiones más durables de
este escéptico del amor fueron Constanza Nordenflycht, que lo siguió
desde el Perú y le dio tres hijos, y Rosa Mueno, la estrella rutilante
de las fiestas a puertas cerradas de la Filarmónica. Pero fue la Meche
Barros, una campesina de Placilla, la musa que le arrancó el más
encendido de sus homenajes poéticos. Al decir de un historiador, don
Diego habría comprado el fundo de El Rayado (negocio desastroso) con el
primordial objeto de hacerse vecino de esa beldad campestre. Para vivir
esta nueva aventura, el improvisado agricultor púsose en carácter
vistiendo el pantalón de brin y la faja roja de los huasos elegantes, y
le encargó a Garfias: ``Mándeme una guitarra hecha en el país, que sea
decente, de muy buenas voces, blanda, bien encordada y con una
encordadura de repuesto''. ``Igualmente me mandará algunas frioleritas de
mujer, que cuesten poco, pero que sean de gusto, porque no es huasa la
persona a quien voy a obsequiarlas''. ``Por Dios le pido que me mande
también dos matecitos dorados de las monjas, de esos olorocitos\ldots''

La Merceditas Barros era hija de un huasamaco adinerado y se entregó sin
resistencias al hombre avasallador que había hecho entrar al país en
vereda, que lo mandaba aun estando fuera del Gobierno, recluido entre
alfalfas y sandías. Desde el corredor de su casa, el amante espiaba a su
querida con un anteojo de marina, y cuando deseaba festejarla invitaba a
los vecinos con la señal famosa de una bengala disparada al caer la
noche.

De estos amores con, mate y guitarreo fue fruto un nuevo hijo de
ganancia, como llamaban entonces a los ilegítimos. De su suerte nada
sabemos, pero aquí están los versos que la madre inspiró con su belleza
al amigo pasajero:

Se empeñó la agricultura\\
Con anhelo singular,\\
Para poder cultivar\\
La planta de tu hermosura.\\
No se vio más preciosura\\
En el orbe hasta el confín.\\
Plantas de bellas al fin\\
Dio aquel prado soberano,\\
Donde con sus propias manos\\
Plantó Cupido un jardín.\\
Por una casualidad\\
A ver el jardín entré,\\
En el momento miré\\
Entre flores tu beldad:\\
Rendí, pues, mi voluntad\\
A ti, preciosa azucena,\\
Y dije en hora muy buena:\\
Te he de amar constantemente;\\
Porque te vi floreciente\\
En situación muy amena.

Empeñado ya en mi esmero\\
En arrancarte de ahí,\\
¿Será esa flor para mí?,\\
Le pregunté al jardinero.\\
Me respondió placentero:\\
Supuesto que aquí viniste\\
Tómala, pues la elegiste;\\
Y entonces con mil amores\\
Te saqué de entre las flores\\
En donde te produjiste.\\
Cuando te tuve en mis manos\\
Rindiéndote adoraciones,\\
Dije que más perfecciones\\
No caben en lo humano;\\
Mil gracias di al hortelano\\
Con una alegría plena;\\
El alma de angustia llena\\
Cada instante repetía:\\
Ya por fortuna eres mía,\\
Flor de la tierra chilena.

No hay para qué decir que las poesías de Portales sólo tienen un valor
de curiosidad. Fue un poeta de álbum, un versificador como hubo miles en
una época en que hasta las diatribas políticas se escribían en verso. El
mismo fue blanco de una de ellas, y puede presumirse que la contestó de
igual manera.

Es en la prosa donde Portales brilla como un artista, y se le ha hecho
justicia al incluirlo entre los maestros del género epistolar. El más
exigente de dos críticos tiene que admirar, por ejemplo, esa joya
literaria que es la carta No. 248 del Epistolario, escrita al correr de
la pluma para divertir a su comadre doña Rafaela Bezanilla, la viuda del
Presidente Ovalle:

``\ldots{}Vaya, pues, mi comadre querida; dentro de poco será usted
abuela. Así pasan los tiempos, y la mejor hermosura desaparece con
ellos. Consolémonos con que cuando usted esté sentada en su cojín,
tomando el polvillo por arrobas y repartiendo los bizcochos a los
bisnietos, yo iré afirmándome en mi bastón a pasarme muchas noches con
usted, puesto a su lado recordaremos nuestros tiempos, murmuraremos; de
medio mundo, hablaremos de las misiones y vías sacras, de los camisones
almidonados, de manga ancha, que ahora se usan y que no se usarán
entonces. Diremos: aquellos zapatos de cabritilla bordados que ya no
vienen; aquellos atacados; aquellas peinetas grandes, que parecían
canastos de dulces en la cabeza; aquellas bolsas de terciopelo y de
mostacillas tan lindas, en que se echaban los pañuelos, la caja, las
llaves de las cómodas y de los escaparates y en que podía echarse hasta
la sartén de la cocina, etc., y concluiremos diciendo que ya se acabó el
gusto y que todo lo que viene es malo. Ya me parece, comadre, que nos
estamos pasando tan buenos ratos, y que en medio de la conversación me
le quedo dormido, y la Luisa y la Jesús mandan que me prendan la
linterna para despedirme, porque les he revuelto el estómago con mi tos
y lo demás que se sigue, que nuestros padres echaban en el pañuelo y
nosotros en la escupidera. Ya me veo averiguando la vida y milagros de
todo el mundo, y recogiendo cuentos contra el honor de todos para
llevárselos a usted a la noche.

Me parece que estoy oyendo renegar a la Luisa cuando me oiga el Deo
Gratias, porque tiene que pararse a hacer cebar el mate para el perro
viejo odioso.

``Calcule usted, comadre mía, el porte de las visitas que le haré, por
las que le hacía el año pasado. Creo que estaré esperando que se
levante, usted de su siesta para colarme a la casa, y que me despediré
cuando las niñas, después de haber cabeceado bien en sus asientos, se
vayan entrando de una en una a acostarse y nos dejen solos. Me figuro
que los dos nos quedaremos cabezada va y cabezada viene, como si nos
estuviéramos haciendo cortesías; y en una de éstas me sale usted
preguntando, medio dormida, que si me acuerdo de aquella vieja que
parecía choclo y que andaba luciendo con una negra en una calesa, y que
si recuerdo cómo se llamaba; y yo, que he de ser muy torpe y
desmemoriado cuando llegue a. esa edad, me volveré a quedar dormido sin
recordar el nombre de doña Berenjena. ¡Qué porvenir tan halagüeño!

``Basta, comadre, de disparates. Me he extendido en ellos porque no
quisiera dejar la pluma de la mano cuando me dirijo a usted, de quien
soy apasionado amigo y seguro servidor.

D. Portales''

\chapter{RENGIFO}

Nuestro más famoso Ministro de Hacienda contaba treinta y siete años
cuando Portales le confió la tarea de sacar la economía nacional de la
ruina y el caos. Aunque distinto en temperamento y carácter, don Manuel
Rengifo y Cárdenas se asemeja a Portales por su condición de comerciante
desaforado, por su tardía improvisación como estadista y por la
eficiencia prodigiosa con que se expidió. Como su genial colega,
sacrificó tranquilidad y negocios para consagrarse al servicio público;
y realizado el milagro de convertir la bancarrota en bonanza, pudo
declarar al final como un patricio de la antigüedad: ``A mis hijos no les
dejo más que mi nombre''.

Es casi ignorada su emocionante biografía, novela real de viriles
andanzas, de aventuras sin fin y de vicisitudes sobrellevadas con
entereza inmutable. Hijo de un médico prematuramente fallecido, tuvo que
abandonar las aulas primarias para ayudar a sostener a la madre y los
cuatro hermanos menores. Fueron precoces su vocación matemática, su
letra de calígrafo y su pasión por la lectura. De niño no jugaba; leía
sin cesar, y este hábito absorbente convirtióle con el tiempo en un
retraído autodidacto capaz de llegar a contarse entre los hombres cultos
de su medio. A los quince años se empleó como oficinista en el almacén
del vasco Arrué, ganando dieciséis pesos mensuales. Llevaba los libros,
redactaba la correspondencia y manejaba la llave de la caja de fondos.
Llegó el patrón a confiar en él hasta el punto de autorizarle para
imitar su firma en casos de enfermedad o ausencia.

No enturbió su armonía la alborada de septiembre de 1810, con ser
Rengifo patriota y Arrué ardoroso realista. El empleadito de diecisiete
años ingresó en 1813 al batallón de Voluntarios de la Patria, y poco
después sirvió en las guardias cívicas de vigilancia nocturna. Desde que
el vascongado tuvo que eludir la persecución de las autoridades, ya en
plena guerra separatista, Rengifo se hizo cargo del almacén, velando por
los intereses de su enemigo político\ldots{} Pero esto sólo duró hasta
la tarde del 2 de octubre del año 14, cuando llegó a Santiago la noticia
del desastre de Rancagua. En esas horas tremendas se encerró en la
trastienda para practicar el balance del mes, despedirse de Arrué por
escrito y entregar la llave del local a quien correspondía. Demasiado
teñido de insurgente, no podía quedarse; desprovisto de recursos, no
podía emigrar con su familia. Abrazó a los hermanos y entregó a la madre
una parte de los haberes de su alcancía; luego cargó un par de mulas y
partió para Mendoza con el porvenir trocado en noche lúgubre. En
compañía de su amigo Juan Melgarejo pasó los Andes siguiendo la columna
interminable de tropas y soldados dispersos, de recuas con equipaje y
pertrechos, jinetes con gente al anca y carretas y coches que huían,
envueltos en polvo y lamentaciones.

Ajenos a la rivalidad entre o'higginistas y carrerinos, los dos
camaradas siguieron a Buenos Aires, y desde allí, asociados, probaron
suerte en la compraventa de cueros vacunos. La operación consistió en
adquirirlos en Córdoba, yendo de estancia en estancia en una carreta
recolectora, que manejaban ellos mismos, para en seguida salarlos,
conducirlos al Plata y ofrecerlos a los exportadores. El negocio produjo
una ganancia que duplicó el exiguo capital, y este éxito les indujo a
arriesgarse en una especulación de doblada envergadura. Cargaron la
carreta con mercancía surtida, y picana en mano salió Rengifo solo (aún
no tenía veintiún años) hacia el Alto Perú, cuyo comercio con Buenos
Aires estaba interrumpido por causa de la guerra de emancipación. Era un
viaje de seiscientas leguas, y el heroico muchacho atravesó siete
provincias soportando barquinazos, calores tórridos y lluvias
torrenciales bajo el toldo de esa tortuga rodante. Al entrar en
territorio altoperuano se encontró siguiendo la ruta del general
Rondeau, que iba en busca del enemigo realista. Descansaba en una posta
cuando supo que los patriotas acababan de ser derrotados en Sipe-Sipe y
venían huyendo a la desbandada y saqueando los pueblos del camino.
Alcanzó a vender el cargamento, incluidos la carreta y los bueyes, y
emprendió la escapada revuelto con los primeros fugitivos de la deshecha
expedición. Durante meses caminó subiendo y bajando montañas,
extraviándose en selvas espesas, durmiendo a la intemperie, atravesando
ríos desbordados y dando enormes rodeos hasta llegar a Tucumán, donde su
socio tardó en reconocerle por la traza de vago andrajoso que llevaba.

No bien el Ejército Unido reliberó a Chile en Chacabuco, el joven
comerciante retornó a la patria conduciendo una tropilla de mulas
cargadas con mercadería europea. Entre otras cosas traía los primeros
ejemplares del libro de Lacunza que conocerían los chilenos.
Desempacando esa joya literaria, y telas de Francia, botones de nácar,
medias, perfumes, jabones y latas de té, abrió tienda en los portales de
la Plaza de Armas. Las duras correrías habían hecho de él un mozo fuerte
y decidido. Su biógrafo, Ramón Rengifo, cuenta que una noche, en cierta
fonda de los alrededores de la plaza, intervino en defensa de un amigo
atacado a puñaladas por un sirviente del establecimiento; le hizo frente
esgrimiendo el estoque de su bastón, y al cabo de largos minutos de
fintas y cuchilladas al aire puso en fuga al agresor con una estocada
que le hirió en el brazo con que manejaba el puñal.

Fue después de la batalla decisiva de Maipú que instaló el famoso Café
de la Unión, asociado otra vez con Melgarejo, ocupando una casa de la
calle Catedral esquina de Morandé. Dice Zapiola en sus Recuerdos de
treinta años que era un salón provisto de espaciosos ventanales; llegó a
ser el centro de reunión de la gente visible y en él daba sus clases de
baile don Manuel Robles, el autor de la vieja Canción Nacional. Con
todo, fue un mal negocio, explotado a pérdida desde el día de su
apertura, y sus dueños optaron por liquidar adelantándose a la quiebra
inminente.

De la aflictiva situación en que quedara Rengifo vino a librarlo su
amigo Ignacio Urízar, que lo asoció a su empresa de exportaciones al
Perú. Embarcado en la fragata \emph{Peruana}, el antiguo carretero de
las pampas salió con rumbo a Valdivia, convertido en fletero del mar,
para recoger el trigo y las maderas que llevaría al Callao\ldots Todo
marchó bien hasta el momento en que, estando el buque cargado y listo
para levar anclas, fue requisado por orden del coronel Beauchef, que de
su cuenta y riesgo preparaba una incursión militar contra Chiloé. Viendo
que empezaban a vaciar la bodega para ocuparla con tropa y pertrechos,
recurrió el afectado a cuanta gestión cabía, desde la súplica hasta la
amenaza, para impedirlo. Obtuvo al fin la devolución, y los propios
soldados de Beauchef ayudaron a estibar; pero a las pocas horas de haber
salido a la mar, la fragata fue asaltada por un espantable temporal del
norte. Inundada y con la bomba de achique obstruida, la tripulación
perpleja y los pasajeros aterrados, parecía el naufragio inevitable
cuando el futuro Ministro intervino con una inspiración que pudo haber
sido inútil de habérsele ocurrido una hora después. Hizo encerrar en sus
camarotes a mujeres y niños y puso a los hombres, de capitán abajo, a la
tarea de abrir un hueco a través de la carga, hasta llegar a la sentina
para destapar el tubo de la bomba. Operación ejecutada en tinieblas y
soportando los revolcones del barco a merced del oleaje montañoso. Todo
el trigo sacado a cubierta se empapó de agua salada e igual daño causaba
la que entraba por la abierta escotilla, en tanto que las tablas y
tejuelas eran arrebatadas por el viento. Cuando el mecanismo comenzó a
funcionar, la mitad del cereal estaba echado a perder y hubo que
arrojarlo por la borda. Así se salvó la \emph{Peruana} y pudo arribar al
Callao, donde el excelente precio del trigo permitió a los exportadores
cubrir la pérdida y hacer utilidad.

Lejos de quedar escarmentado, decidió Rengifo abrazar de lleno la
carrera de armador, para lo cual se independizó de Urízar y compró en el
Perú el pequeño bergantín José, que destinó a la navegación entre el
Callao y Valparaíso.

Lo mismo que Portales, había sido cogido por el embrujo de los negocios
marítimos, que en sus días eran un puro y constante azar. Con esta
salvedad: que mientras la goleta Independencia dio a su dueño por lo
menos para vivir, el José fue el quebradero de cabeza y el causante de
la ruina del suyo. De regreso del viaje inaugural a Valparaíso dio fondo
en el puerto peruano a últimos de enero de 1824. No había echado sus
sacos y bultos al muelle cuando se sublevó la guarnición de los fuertes,
y como consecuencia estos pasaron a poder de los realistas. Este vuelco
inesperado sorprendió al bergantín dentro del campo de tiro de los
cañones, y para colmo, inmovilizado, pues tenía el timón y las velas en
reparaciones. Cuando las baterías rompieron a disparar contra las naves
de bandera patriota, afloró en Rengifo el hombre de las decisiones
intrépidas. Con ayuda de cuatro marineros tomó el barquichuelo a
remolque y en medio de la lluvia de proyectiles lo arrastró a fuerza de
remos hasta el único refugio seguro:, el fondeadero de la estación naval
británica. Pero a poco, fatalidad inexplicable, el José fue apresado por
orden del almirante Guise, comandante en jefe de la escuadra patriota
del Perú. Nunca iba a saberse el porqué de este despojo gratuito,
perpetrado bajo el pabellón de una nación amiga y por decisión de un ex
oficial de la Marina de Chile. Y el colmo de su tropelía fue que obligó
al naviero a irse a tierra, donde de inmediato cayó en manos de los
españoles. Logró quedar libre, pero ninguna de sus gestiones, incluso
ante Bolívar, pudo impedir que su buque permaneciera en poder de la
armada peruana. De resultas de su demanda de amparo al gobierno de
Santiago se fijó fecha para que un tribunal de presas (controlado por
Guise.) diera su fallo. Y como éste le fue adverso y era sin apelación,
se le escapó de entre las manos el único bien que poseía.

Intentando rehacerse trabajó a sueldo de unos industriales mineros de
Jauja y Huancavelica; y empezaba a divisar una suerte mejor cuando un
decreto del gobierno obligó ``a los extranjeros'' a abandonar el país en
plazo perentorio. No tuvo tiempo sino de hacer su maleta y emprender el
regreso en calidad de indigente.

Cuando estuvo en el Perú hacía ya tiempo que Portales, frustrado
también, había retornado al terruño. Eran los dos de igual edad, y su
primer contacto histórico se produce en 1827, a raíz del golpe del
coronel Campino que dio con Portales y su grupo en la cárcel. Siendo
también de la amistad de Campino, Rengifo obtuvo permiso para visitar a
estos presos incomunicados y entregarles sus socorros caritativos. Y
cosa propia de su carácter: cuando la contrarrevolución derribó a los
golpistas, intercedió ante Portales para que no fuesen demasiado
rigurosos con el encarcelado coronel\ldots

Los nombres de Portales y Rengifo vuelven a relacionarse con motivo de
la liquidación del Estanco, el ruinoso monopolio de tabacos, naipes y
licores concedido a Portales Cea \& Cía. a cambio del servicio del
empréstito inglés. La firma de Manuel Rengifo está entre las de los
cuatro ciudadanos escogidos para examinar sus cuentas y determinar su
responsabilidad ante la ley. Quedó probada sin sombra de duda la
limpieza de las operaciones, y el laudo del jurado liberó a Portales y
sus socios de todo cargo al dictaminar que. sólo eran ``agentes del
Gobierno'' y que ``a éste correspondían las utilidades o pérdidas''. Su
intervención irreprochable atrajo sobre Rengifo los dardos de la prensa
pipiola, que le acusó de parcial y deshonesto y le timbró de pelucón y
estanquero, banderías políticas a las que era ajeno. Pudiendo haberse
defendido (era articulista del diario La Aurora), guardó el silencio
olímpico del hombre colocado por encima de la sospecha.

Quien mejor conocía su desinterés era el viejo Arrué, que en su lecho de
moribundo recordó al niño dependiente de su almacén y quiso legarle la
casa en que vivía, ofrecimiento que él rehusó en favor de los deudos.
Esto, cuando su pobreza era tal que al casarse (con Dolores Vial, la
novia que su madre le señalara al morir), tuvo que irse a trabajar al
fundo de su suegro\ldots

Pero es a esta culminante estrechez a la que hay que agradecer la
repentina mudanza de su destino. Porque habiendo desembocado las
turbulencias políticas en la acción militar de Ochagavía, cerca de donde
él encontraba, fue invitado a interponer sus dotes conciliadoras y luego
a redactar las bases del armisticio. De ahí a la notoriedad y a los
honores sólo había un paso, y así fue que al instaurarse el régimen
portaliano le ofrecieron temible cartera de Hacienda, de la que se hizo
cargo el 19 de julio de 1830.

Jamás se había visto, ni se ha vuelto a ver aún, el estado de. agonía de
las finanzas de la nación con que -se encontró al sentarse en su
despacho. Todos los empleados públicos y todos los militares se hallaban
impagos, las escuelas y los hospitales estaban por cerrar, el bajo clero
mendigaba en la calle, la Aduana era una ruina y muchedumbre de
contratistas, de cesantes y viudas pensionadas invadía diariamente los
pasillos y la sala de espera del Ministerio de Hacienda.

Como Portales, Rengifo puso manos a la obra sin discursos, programas ni
declaraciones, limitándose a operar como su padre el cirujano. Sus dos
primeras medidas fueron reducir la planta del Ejército y eliminar a los
funcionarios superfluos e incompetentes. Uno de éstos era pariente suyo
y no lo pudieron salvar ni las súplicas de la madre y la esposa. En
seguida suprimió los taquígrafos del Congreso, y por otro decreto ordenó
que hasta la más ínfima orden de pago debía llevar su firma, poniendo
fin de este modo a los fraudes y derroches.

En las oficinas del Estado nadie se atrevía ahora a llegar con un minuto
de atraso, a hacer tertulia ni a cometer un error\ldots; pero afuera ---y
aun adentro--- empezaba a levantarse el sol de la esperanza. Estaba
claro que este estadista improvisado era capaz de manejar la picana, la
alcancía y la bomba de achique de la economía del país. Por eso
escribiría Portales: ``\ldots{}el Gobierno tiene en su seno un hombre
con quien puede consultar en todos los negocios en que desee saber mi
opinión, porque casi siempre hemos andado acordes''.

Pero aquella seguidilla de resoluciones elementales y drásticas, de
efecto inmediato, es nada comparada con el recurso que ideó para atacar
el más insostenible de los problemas: el adeudamiento de las
obligaciones y de los sueldos de civiles y militares. Reducido a
fórmula, era esto: ``Se pagará a los acreedores con su propio dinero\ldots''
Hoy se hace difícil concebir que tal procedimiento haya sido aplicado
sin oposición y en medio de la confianza general, producida por no se
sabe qué misteriosa sugestión. (No en balde dijo Encina que en el
régimen portaliano hubo algo inexplicable.) El modus operandi de esta
maniobra portentosa consistió en que cada acreedor tenía que entregar al
Estado, en metálico y al contado, el doble de lo que éste le debía,
recibiendo a cambio de tal -suma el equivalente en libramientos contra
pagarés de la Aduana (R. Rengifo: Memoria biográfica). Tentativa loca o
golpe de genio, dio resultado; y de ahí en adelante nunca volvió a
interrumpirse el pago puntual de las remuneraciones y compromisos
fiscales.

Cosa jamás vista: el Ministro publicaba semanalmente el balance de la
Tesorería. En su afán de aligerar las cargas estatales devolvió a las
congregaciones religiosas las propiedades que les expropiara Freire y
que sólo arrojaban pérdidas al Estado metido a empresario; pero se las
restituyó bajo contrato de que en cada convento, parroquia o fundo
deberían abrir una escuela primaria; con lo cual extendióse la educación
del pueblo sin desembolso del Fisco. La reducción de los gastos
administrativos daba cifras como la registrada en Valdivia, donde el
primer año se habían economizado cincuenta y tantos mil pesos. Se
perseguía el contrabando como un crimen de lesa patria, y para mejorar
la eficiencia de la burocracia introdújose en el Instituto Nacional la
enseñanza del método contable de partida doble. Con su letra de
calígrafo escribió el Ministro en 1832: ``A la sabiduría del Congreso no
pueden ocultarse las ventajas de una ley protectora de la libertad de
comercio marítimo\ldots que concediendo franquicias y seguridades a todas
las naciones\ldots fije en nuestro principal puerto el mercado del
Pacífico\\ldots'' Era su proyecto más ambicioso: el que convertiría a
Valparaíso en un emporio internacional de primera importancia; y no bien
estuvo la ley promulgada fue en persona a instalar los Almacenes de
Depósito y a poner en vigencia su minucioso reglamento.

Con todo, la máxima gloria de su gestión ministerial es la presentación
del primer Presupuesto equilibrado que vieran los chilenos desde 1810.
Se había reiniciado el servicio del empréstito inglés y reducido la
deuda flotante en más de un millón de pesos, mientras que en Tesorería
quedó un superávit de doscientos mil. Meta alcanzada en sólo cuatro años
y cuya realidad se reflejó en el valor de los billetes de la Deuda
Interior, que remontó desde el 24 al 68\%. Y lo que deja perplejo al
estudioso es que esto se lograra sin recurrir a nuevos impuestos.
Incluso fue abolido el derecho de alcabala, que. gravaba los productos
agrícolas y daba pie a los abusos de los recaudadores a comisión; una
medida que produjo el abaratamiento de la vida y fue celebrada con
júbilo popular en la plaza de abastos de Santiago el 1ero de enero de
1832. De igual modo fueron rebajadas las patentes de bodegones,
cigarrerías y negocios minoristas (ley de 30 de agosto de 1833). Este
estímulo al comercio se extendió a las industrias con el otorgamiento de
exenciones y privilegios que vigorizaron la minería, la pesca y la
marina mercante. Operando estos factores en consonancia con el auge de
los Almacenes de Depósito, determinaron que el movimiento marítimo de
Valparaíso aumentara casi al doble en tres años. En sus líneas generales
la estrategia del Ministro dirigíase a adaptar la economía a las
condiciones de la era republicana, fomentando el intercambio
preferencial con las naciones vecinas, el más lejano antecedente del
moderno proceso de integración económica regional.

Al alejarse de su cargo, en 1835, este virtuoso de las finanzas dejaba
consumada una obra que normalmente habría requerido el espacio de dos o
tres períodos presidenciales. Figura demasiado grande para el pobre
medio en que actuara, en muchos de sus contemporáneos despertó envidia
en lugar de admiración, y hasta hubo enanos que le atacaron cuando ya
estaba ausente de la escena política, buscando en el campo la
restauración de su perdido bienestar. No respondió ni permitió que sus
amigos lo hiciesen por él, sabiendo que era la posteridad la que debía
juzgarle. Y\ldots ``a mis hijos tes dejo mi nombre''.

\chapter{DESCUBRIMIENTO Y ESPLENDOR DE CHAÑARCILLO}

Chile, otrora gran productor de plata, tuvo en el cerro de Chañarcillo
uno de los auges más espectaculares de su historia mineralógica. Fue un
chorro de riqueza que empezó a caer sobre la árida Atacama a principios
del decenio de Prieto y no pararía de fluir hasta las postrimerías de la
guerra del Pacífico. Pequeña California argentífera, Chañarcillo
apuntaló al país en bancarrota en los años en que el Ministro Rengifo
llevaba a cabo su portentosa reconstrucción económica. En definitiva,
ignórase quién fue el legítimo descubridor, pues se disputan este
privilegio la pastora india Flora Normilla y su hijo el leñador Juan
Godoy. Oficialmente es Godoy el precursor y por eso es a él a quien
representa la estatua de la plaza de Copiapó; pero aquí juega la
tradición más que la historia documentada, y ante la incertidumbre de lo
legendario es imposible decidirse por una u otra de las versiones. Una
de éstas afirma que Flora tenía descubierto el mineral, con el que había
tropezado mientras iba arreando sus cabras, y que participó del hallazgo
a su retoño y más tarde al magnate don Miguel Gallo, al que profesaba
gratitud porque acostumbraba detenerse en su choza de la quebrada de
Pajonales a beberse un vaso de agua y en ocasiones le daba una moneda o
le llevaba un regalillo. Repetidas veces la india hizo referencia a lo
mismo, y otras tantas el copiapino, harto de esas historias de
derroteros, la escuchó sin prestar atención; escepticismo que no varió
ni cuando ella decidió dejarle como heredero de su mina.

La otra versión tradicional pretende que la anciana ``testó'' también a
favor de su hijo, el que habría tenido tan poca prisa como Gallo en ir a
echar una mirada al fabuloso cerro; y que cuando el susodicho Juan Godoy
dio por fin con la veta, sostuvo que ésta no correspondía a la que
señalara su ya difunta madre, y que él, por lo tanto, era; descubridor
de algo nunca visto por nadie. Puede que lo dijera por error o por darse
pisto, pero no para jugarle una trastada al señor Gallo, pues lo primero
que hizo fue tomar el camino de Copiapó, al trote de su mula, para. ir
en busca del caballero que había sido amigo de la finada. Y de aquí en
adelante siguen paralelas v las dos versiones, porque ambas rezan que el
bueno de Juan iba en cada parada. del trayecto revelando la nueva con
lujo de pormenores y mostrando los trozos de mineral que portaba en sus
alforjas. Era hombre de tal humildad que acaso no aspirara más que a
trasladar a su Ana Alcota y sus cinco niños del rancho en que vivían a
una casita de adobes, La gente de su clase le teme a cambiar de
condición; por eso el asustado leñador venía incitando la codicia
pública para que una manada de lobos acudiera a disputarle la presa; y
de sus derechos en la mina Descubridora iba a cederle la mitad al señor
don Miguel y una parte del resto a su hermano José Godoy. Entró envuelto
en polvareda por las calles estrechas y tortuosas del pueblo y fue a
llamar con timidez a la puerta falsa de la quinta de Gallo, en el
florido barrio de La Chimba.

En Atacama no había clan más poderoso que el de los Gallo, mineros,
alcaldes y regidores constituidos casi en amos de la provincia. Don
Miguel Gallo Vergara era dueño de una mina de cobre y un ingenio y
fundición de metales; durante la Independencia había sido
teniente-gobernador del departamento, y cuando O'Higgins aceptó su
renuncia en 1818 expresó sólo que ''no olvidaría sus méritos y
sacrificios por la libertad del Estado\ldots{} dándole entretanto las
debidas gracias a nombre de la patria''.

Admirable escena, propia del Copiapó de esos días, ha debido ser la. del
señor de levita y sombrero de copa y el montañés de manta y ojotas
presentándose juntos en la oficina del Juez de Minas de la villa (19 de
mayo de 1832) para pedir ---dice el certificado--- ''una veta de metales
de plata que han descubierto en las sierras de Chañarcillo, dando vista
a la quebrada del Molle y a Bandurrias, en cerro virgen\ldots{} Se les
hizo merced de ella, sin perjuicio de tercero y con arreglo a Ordenanza,
para lo cual les extiendo su registro. Doy fe. ---Vallejo''.

No hay testimonio de cómo fue la relación humana entablada entre los
integrantes de la pintoresca sociedad minera Gallo \& Godoy; pero cabe
suponer que habrá sido en la galería de la casaquinta del potentado, y
no en el salón y menos todavía en el comedor de la familia donde el hijo
de la india Flora se sentó, todo confundido y haciendo girar su bonete
de paja, para referir a su patrón, no, no, a su socio, cómo había hecho
el hallazgo de la Descubridora.

Precisemos que el cerro de Chañarcillo se encuentra a dieciocho leguas
al sur de Copiapó y que su cumbre alcanza a unos 1.800 metros sobre el
mar. El triste yermo que es ahora estaba todavía entonces cubierto de
manchones de olivillos, de espinos y de algarrobillas. Fragantes
bosquecitos que el hacha venía talando, como en tantas colinas y valles
atacameños y sin que nadie se preocupara de replantar, hasta que un día
se convertiría en sarcasmo el nombre español de Copiapó: San Francisco
de la Selva\ldots Decía Juan Godoy ---y así lo cuenta José Joaquín
Vallejo--- que había tropezado con la mina mientras iba con sus perros
persiguiendo a un guanaco. Cansado de correr cuesta arriba, se sentó a
la sombra de una algarrobilla\ldots ``¡y un minuto después Chañarcillo
estaba descubierto!'' Otro cronista, Ramón Fritis, escribe que ``la
superficie del suelo aparecía empedrada de trozos de plata, o rodados,
algunos de dos o más quintales de plata maciza\ldots''

¿Cómo se explica que nadie, excepto Flora Normilla, hubiera visto
aquello en años y siglos? El lugar está por el SO del cerro y junto al
viejo camino de Copiapó a Huasco, por el que numerosos viandantes
pasaban cada día; leñadores y cabreros recorrían el cerro en su
cotidiano trajinar, y el propio don Miguel Gallo tenía casi allí mismo
su fundición de metales. Ante este impenetrable misterio hay derecho a
suponer que las vetas superficiales habían sido removidas por los
formidables terremotos contemporáneos como los del 3, 4 y 11 de abril de
1819 y el del 5 de noviembre de 1822, que demolieron Copiapó--- echando
a rodar por escarpas y laderas los millares de corpulentos pedazos de
mineral que ahora estaban a la vista. Así y todo es inexplicable, pues
iban transcurridos desde entonces tres lustros y bien sabemos que
Atacama era un hervidero de cateadores que se conocían cada collado o
quebrada como el seno de la mujer querida.

Chañarcillo es pues un enigma, una rareza o un encantamiento; tómelo
cada cual como prefiera. Aquí se trata de su historia comprobada, y ésta
dice que el poderoso Gallo y su ejecutivo Juan José Sierralta Callejas
pusieron en laboreo la Descubridora, a tiempo que decenas de mineros
tomaban el resto del cerro por asalto. En la Historia de Copiapó,
magnífico libro de Carlos María Sayago se lee que en rápida seguidilla
fueron denunciadas y concedidas las minas de Manto de Volados, Colorada,
Bolacos, Guías, Reventón Colorado, Manto de Cobos, Merceditas y
Candelaria. ``La oficina del escribano'', refiere Sayago, ``veíase
asediada, la mesa del diputado de minería llena de peticiones, la villa
y todo el valle en gran movimiento. Por todos rumbos llegaban gentes al
nuevo mineral, numerosos cateadores recorrían el cerro recogiendo
rodados, picando vetas, quebrando crestones, labrando catad''. ``Todo el
cerro parecía un promontorio de metal: mientras más se le recorría,
mientras más se rebuscaban sus matorrales y trepaban sus riscos, más
plata aparecía''.

De lejos aquello asemejábase a un queso atacado por las ratas. Eran las
bocaminas y las cuevas labradas a modo de refugios por hombres demasiado
impacientes para armar una mediagua de tablas. Lo esencial era extraer
los primeros quintales del tesoro enloquecedor, extraerlos como se
podía, a veces casi a mano limpia. A la vuelta de unos meses había cien
tropillas de mulas ocupadas en el transporte de mineral, de víveres y de
agua en barriles por el polvoriento camino de Copiapó. Y no tardarían en
surgir los galpones, las viviendas y las humeantes chimeneas de la
Descubridora y las minas rivales, donde harían su agosto empresarios
como José Vallejo, Pascual y Manuel Peralta, J. J. Sierralta Callejas y
el barretero Juancho.

En cincuenta años Chañarcillo vería prosperar y brocearse un centenar de
minas, que en conjunto entregaron trescientos millones de pesos y
contribuyeron a hacer de Chile el tercer país del mundo en esta rama de
la minería. Acudieron hombres del Perú, Bolivia y Argentina; y esta
legendaria imagen de riqueza y poderío llegó a despertar en los cuyanos
la añoranza de su ancestro chileno. Los historiadores han hecho poco
caso de la visita de los comisionados mendocinos Jil y Recuero y el
periodista José L. Calle, que viajaron a Santiago para proponer a
Portales la reincorporación a Chile de la provincia de Cuyo. Este
proyecto peregrino, que el Ministro rechazó de plano, tuvo su origen en
la desesperación causada por la anarquía crónica del otro lado de los
Andes, en la fascinación producida por el orden político ejemplar
instaurado en nuestro país y en la prosperidad deslumbrante generada por
las minas atacameñas, cuya abundancia sugería la existencia de una nueva
México.

Anexo al mineral fue tomando forma un caserío que con el tiempo adoptó
el nombre de Juan Godoy y que a su hora incluiría Jotabeche entre las
curiosidades autóctonas: ``un pueblecito en que más de mil hombres viven
sin cargar la cruz, quiero decir, sin mujeres. Aquello es un portento
social. Hombres barriendo, hombres lavando, hombres espumando la olla,
hombres haciendo la cama, hombres friendo empanadas, hombres bailando
con hombres, hombres cantando \emph{La extranjera}, y hombres por todo y
para todo; es una colonia de maricones, un cuerpo sin alma, un monstruo
cuya vista rechaza y que no es la cosa menos notable de nuestro Chile''.

Esta ausencia del sexo femenino en Chañarcillo debíase a un famoso
reglamento que con el fin de combatir el cangalleo le prohibía el acceso
a las minas y al poblado anexo bajo penas de multa y prisión. El robo de
metales ``al detalle'' era una plaga congénita de la minería y suponíase
que las mujeres lo practicaban llevándose trozos de plata ocultos bajo
el manto, en la bolsa de ropa o en el canasto de la merienda. Sospecha
que probó ser infundada cuando se vio que la vejatoria prohibición no
disminuía ni en un gramo el sempiterno cangalleo.

Pero este mal inextirpable era nada comparado con los desórdenes,
reyertas y sublevaciones que el vino, el naipe y el hastío desataron en
los primeros años de la explotación del cerro. Fue menester que la mano
inflexible de Portales se estirara hasta el lejano paraje, cercándolo
con tropa de línea que a fuerza de bala y bayoneta contuvo el caos para
asegurar la normal actividad del mineral.

El chorro de millones que fluía de los piques transformó la vida y
costumbres de la apacible Copiapó. Tal como Darwin la vio en 1835, en su
viaje a caballo desde Valparaíso, se había convertido en ``una ciudad
poco agradable. Cada cual parece no tener otro objeto que ganar dinero y
marcharse lo más pronto posible. Casi todos los habitantes se ocupan de
lo mismo y no se oye hablar de otra cosa que de minas y minerales''. Al
ojo europeo del sabio impresionó el contraste desolador entre esa fiebre
momentánea y el desprecio de la riqueza vegetal, que pudo ser eterna si
la hubiesen protegido y renovado. A poca distancia del lugar descubrió
los restos petrificados de un bosque de pinos cuyos troncos medían
quince pies de circunferencia; señal de que la masacre forestal de
Atacama es de data precolombina y prehistórica. Chocó también a Darwin
el abismo que mediaba entre el vivir ostentoso de los patrones de minas
y la suerte de bestias' de carga reservada a los hombres que sacaban el
mineral de las galerías. Dice que en Coquimbo e Illapel (pues no estuvo
en Chañarcillo) vio a los apires subir desde ochenta metros de
profundidad por las entalladuras de madera transportando pesos de cien y
ciento veinte kilos; su resuello era un silbido lastimero y sudaban como
en un baño de vapor; y mal alimentados y pagados como estaban, solían
traer a la superficie hasta una tonelada de mineral en el día.

Grandes ``máquinas'' o trapiches instalados cerca del cerro, como el de
Fragueiro y Codecido, no daban abasto, y se calcula que la plata
extraída, de ser toda ella acuñada en monedas, habría podido cubrir la
plaza de Copiapó en tres capas sucesivas. Es fama que hubo aposentos
pavimentados con barras de plata y se vieron caballos con herraduras de
plata y calesas con llantas de plata y vajilla de plata y escupideras de
plata. En ese foco de abundancia, especulación y volteretas de la
fortuna inició sus operaciones, a los dieciocho años de edad, el futuro
fundador del Banco de A. Edwards \& Compañía. En sus Memorias, relativas
a la década siguiente, cuenta Treutler que todavía entonces eran tan
frecuentes los descubrimientos de vetas y, reventones que el Juez de
Minas hizo obligatorio anotar la hora exacta de cada pedimento ---las 9
y 14 minutos, o las 12.23 minutos---, para prevenir las encarnizadas
disputas y pleitos promovidos por los solicitantes de registros. En esa
atmósfera caldeada por la codicia los hombres perdían la cabeza y la
compostura: dos miembros de la familia Gallo dieron una paliza en plena
calle al joven Edwards; y en el libro de Treutler puede verse la nómina
impresionante de personalidades copiapinas mandadas a la cárcel por la
justicia implacable del régimen portaliano, que no distinguía entre
gañanes revoltosos, salteadores de caminos y caballeros deshonestos.

Pero en la amena capital atacameña no todo era bastonazos, tiros y
litigios. Reinaba la alegría propia de los centros de aventura, donde el
numerario corre a raudales y hay que gozarlo hoy porque mañana puede
acabarse. Las familias enriquecidas rivalizaban en lujo y boato, en
competencia de elegantes carruajes y pretenciosos salones en donde lucía
el chic de la última moda de París. La juventud paseaba por La Chimba,
el barrio de las quintas perfumadas de flores, o salía al campo de
picnic en birlochos y carretas y hasta a lomo de mula. Pronto tendrían
teatro y ópera ---la Pantanelli cantó en Copiapó antes que en
Santiago--- y a falta de mejor rendez-vous se merendaba en la elegante
chingana de María Tagle, ubicada en la calle Chañarcillo, donde solía
bailarse la zamacueca con castañuelas a la manera en que lo hacían en el
caserío de Juan Godoy los apires, barreteros y cangalleros.

¿Qué suerte había corrido, entretanto, el precursor del famoso mineral?

Se sabe que al cabo de un tiempo vendió su parte en la Descubridora,
quedando, dice la fama, ``rico en dinero sonante''. Y lo inevitable
sucedió: convirtióse en personaje popularísimo, al que brotaban amigos y
remotos parientes salidos no se sabe de dónde, que le buscaban para
festejarle, proponerle negocios e irle aligerando poco a poco la pesada
bolsa. A la postre

vino a quedar desplumado y tuvo que arrimarse al socio de ayer en
demanda de auxilio. El generoso Gallo le adjudicó una doble en su mina,
dejando él mismo de ganar por ayudarle, y en un par de años el antiguo
leñador y burrero embolsó catorce mil pesos fuertes. Capital con que se
trasladó a La Serena para arruinarse de nuevo, y esta vez en definitiva,
trabajando una malhadada chacarilla que compró en -el camino a Coquimbo.

\begin{verse}
``Arre, arre, borrico,\\
que el que nació pa pobre\\
no ha de morirse rico.''
\end{verse}

\chapter{``EL MINISTRO SALTEADOR''}

Al iniciarse el Gobierno de don Joaquín Prieto, el 18 de septiembre de
1831, el panorama político de Chile había variado de manera
impresionante como consecuencia de la gestión ministerial de Portales.
Pipiolos y liberales sólo habían elegido siete diputados en un total de
noventa y cuatro, en tanto que o'higginistas, estanqueros y pelucones,
convenientemente armonizados, apoyaban al nuevo gobernante con su
mayoría incontrarrestable. Portales, elegido Vicepresidente a pesar
suyo, había aceptado además el nombramiento de Ministro de Guerra y
Marina con la condición de que se le permitiera residir en Valparaíso.
Caso sin precedentes, demostrativo de la necesidad que tenía Prieto de
contar con su consejo y de protegerse bajo el ala de su inmenso
prestigio. Cree un historiador que el singular ``ministro a la distancia''
decidió alejarse para observar la resistencia de la obra gruesa de su
edificio y ver si la construcción podía proseguir sin su concurso. Esto
era fundamental en un sistema que no pretendía sostenerse en la persona
de Diego Portales ni de Joaquín Prieto, puesto que descansaba en
cimientos de un material nuevo: el principio del Gobierno impersonal,
cuyo régimen debe ser durable y respetable de por sí y no por la
presencia de tal o cual individuo.

Pero antes de alejarse, Portales había introducido en el Gabinete a
quien podía substituirle con eficiencia. Así como un día descubrió a don
Manuel Rengifo, que sería el más grande de los Ministros de Hacienda de
nuestra historia, ahora puso su ojo infalible en don Joaquín Tocornal,
otro estadista de su escuela, cauto a la vez que imaginativo, minucioso
e infatigable, que iba a dejar huella indeleble de su acción. En un
puesto subalterno iniciaba su carrera el desconocido joven Manuel Montt,
otro descubrimiento de don Diego, llamado a alcanzar veinte años después
la Presidencia para continuar la obra de su maestro y colocarse a un
nivel de grandeza con él.

Para tranquilidad del Presidente Prieto, la Guardia Cívica creada por
Portales contaba ahora veinticinco mil hombres a lo largo del territorio
y sus batallones perfectamente armados e instruidos constituían un
seguro de paz interna.

A quince horas de birlocho de Santiago, el Ministro ad honorem (porque
hay que insistir en que no cobraba su sueldo) dedicóse a poner orden en
sus deterioradas empresas mercantiles. Poco o nada cuentan los biógrafos
de su curiosa actividad de naviero, de la que hay vasta información en
el Epistolario. Era dueño de la goleta Independencia, un ``dos palos'' de
ciento treinta toneladas, en cuya bodeguita cargaba tabaco, yerba mate,
tocuyos, garbanzos, sebo y hasta botellas vacías para competir en el
cabotaje al Norte Chico y en ocasionales exportaciones al Perú y
Centroamérica. Precisamente de los años 31 y 32 son las cartas en que
informa a Garfias, su agente en la capital, del ir y venir de la pequeña
carreta flotante mandada por el capitán Thomas Wilson. ``\ldots{}la
\emph{Independencia} hace por lo regular sus viajes a Copiapó en 35
días; hoy tiene ya 45, y si no llega esta semana. Por la goleta 4 de
Julio (capitán Wheelwright) hemos tenido noticias de la Independencia:
habló con ella a la altura de 28\textsuperscript{0} de latitud''. ``Acaba
de fondear la Independencia. Sólo me ha traído 4.000 y tantos pesos''.
``La Independencia queda fletada para Guatemala''. ``Este buque, que
durante mi permanencia en la maldita política casi no se movía del
puerto, desde que estoy aquí no se ha parado ni parará\\ldots''

Así predicaba con el ejemplo el que afirmó que los chilenos ``tendrán
-que ser un pueblo comerciante y marinero''.

Pero estuviese en la costa o en cualquiera lejanía, todo se lo atribuían
a él en los corrillos políticos de Santiago; tal era la sugestión de
mando y superioridad que emanaba de su persona. Cuando Bulnes partió de
Chillán con una división del Ejército para ajustar cuentas con los
Pincheira, endosaron a Portales un plan que pertenecía a Prieto; y al
anunciarse la carnicería del valle de Palanquén, que borró del mapa a
los bandidos y liberó a sus mil mujeres cautivas, bendijeron al Ministro
como si él hubiese dirigido esa operación de limpieza. Cosa notable que
hacía don Joaquín Tocornal, fuese la creación de una cadena de liceos o
el establecimiento de la cátedra de Medicina, la opinión daba por
sentado que era obra de Portales.

A la Constitución de 1833 se la llama todavía ``de Portales'', siendo que
fue redactada y promulgada cuando él no tenía cargo ministerial, y toda
su contribución había sido una que otra sugerencia a la asamblea
examinadora del proyecto, de la cual tampoco formaba parte. Cierto que
las disposiciones de la Carta coincidían con lo que él quería hacer de
Chile: una nación respetable de ciudadanos respetados. Constitución
destinada a durar noventa y dos años, para asombro de América,
establecía que ``todos los habitantes son iguales ante la ley'', ``tienen
absoluta libertad de movimiento dentro del territorio y para salir de
él''; ``la propiedad es inviolable, salvo las expropiaciones por causa de
utilidad pública, calificadas por ley e indemnizadas''; ``se prohíbe el
tormento como medio de esclarecimiento y la confiscación de bienes como
pena''; ``hay libertad de publicar las opiniones por medio de la imprenta
y sin censura previa''; ``el hogar, la correspondencia epistolar, los
papeles y los efectos de toda persona son inviolables, salvo en los
casos que contempla la ley la industria es libre\ldots''

Esta ``Constitución de Portales'' sólo era portaliana porque sus
redactores Egaña, Bello y Gandarillas estaban \emph{portalizados}.
Conservaba el Parlamento bicameral, propio de una democracia
representativa, y ponía en él el centro de gravedad del Poder a fin de
impedir cualquier desliz dictatorial del Ejecutivo. El Congreso no podía
destituir al Presidente de la República (como puede hacerlo hoy); pero
ese vacío en algo se contrarrestaba al establecerse que ``aunque se
declare el estado de sitio o se den al Presidente facultades
extraordinarias, no podrá la autoridad pública condenar por sí ni
aplicar penas''. Esto dejaba instituida para siempre la autonomía del
Poder Judicial, fundamento de elementales garantías en una sociedad
civilizada. Y, por último, la ``Constitución de Portales'' suprimió el
cargo de Vicepresidente que ocupaba Portales\ldots

Aun después de renunciar al Ministerio de Guerra y Marina y cuando sólo
era gobernador de Valparaíso, e incluso cuando dejó de serlo, don Diego
seguía observando la marcha del Gobierno, y Prieto a su vez no perdía
ocasión de escuchar su consejo. No en balde le llamaba ``el principal
arquitecto''. Por no entender que era el padre del nuevo Chile es que
Gandarillas se permitió decir que Portales pretendía mandar a los que
mandaban.

Creyendo su vez que ya no mandaba, don Ramón Freire apuró desde Lima su
tren de conspiración permanente. Tal fue el origen de la seguidilla de
conjuraciones llamadas de Labbé, de los Puñales y del capitán Tenorio,
todas frustradas por la delación o la vigilancia policial. La última
sirvió para sacar a Freire de su error. Tenorio, confinado en la isla de
Juan Fernández, desarmó a la guardia del penal y escapó con los reos a
Coquimbo, donde el pretendido golpe revolucionario degeneró en matanza
indiscriminada, incendios y violación de mujeres. Detenido Tenorio tras
corta lucha, el Presidente Prieto mandó fusilarlo como criminal ``por
instancias de Portales''.

Este rebrote de efervescencia de los vencidos en Lircay, más la división
del partido gobernante, indujeron a Prieto a llamarlo urgentemente para
que tomase de nuevo las carteras claves. Así fue como volvió a ocupar
las de Guerra y Marina, Interior y Relaciones, dejando a Tocornal en la
de Hacienda, que Rengifo entregó con el primer superávit registrado en
la historia nacional.

Bastó la presencia de Su Señoría para calmar a los inquietos y
temerosos. Y tan firme quedó el Presidente en su sillón, que nadie osó
discutir la conveniencia de reelegirle.

Iba a ser Prieto el primer gobernante chileno que se haya mantenido en
el Poder por diez años consecutivos.

Pero este segundo quinquenio debía señalar la más dramática etapa del
régimen en su lucha con los enemigos de dentro y fuera del país. Desde
el primer instante el Ministro Portales vio dibujarse en el horizonte el
conflicto armado con la Confederación perú-boliviana. Fue el primero en
advertirlo y el único en creerlo inevitable, y ahí está la carta en que
se refiere al mariscal Santa Cruz: ``Ese cholo va a darnos mucho que
hacer''.

Don Andrés de Santa Cruz, pequeño mestizo indoboliviano de piel cobriza
y astucia de reptil, había concebido el sueño delirante de unir a los
países andinos bajo la hegemonía precisamente del más atrasado y
anárquico de todos. ¡Menudo desafío para Portales, el patriota que llamó
a su tierra natal ``la perla del Nuevo Mundo'' y en cuyo mar no debía
tolerarse otros -cañonazos que los de saludo a su bandera\ldots! El Perú,
primera presa del proyecto cesarista, había visto su independencia
ahogada en sangre y sus líderes nacionalistas pasados por las armas. La
Confederación perú-boliviana representaba una fuerza militar y naval
cuatro veces superior a la de Chile, y consciente de esta ventaja
abrumadora el Protector Santa Cruz había dado comienzo a un plan de
hostilidades económicas y políticas cuyo fin ---aseguraba Portales---
era arrastrar a Chile a una guerra desigual. De tiempo atrás el Gobierno
de Lima hacía oídos sordos a la
cobranza de las deudas contraídas con motivo de la Expedición
Libertadora. Luego vino el gravamen del trigo chileno, que debía pagar
en el Callao tres veces su valor comercial mientras que el producto
competidor norteamericano pasaba casi libremente por la Aduana. En
represalia, el Gobierno de Santiago subió en términos equivalentes la
internación del azúcar peruana. En lugar de ceder, el Perú aumentó al
doble los derechos del trigo chileno y gravó toda mercadería que hubiera
pasado por los almacenes de depósito de Valparaíso; y para remate,
acortó el plazo de desembarque y despacho en el Callao hasta hacerlos
imposibles. Esta guerra aduanera conducía a dejar a Chile sin azúcar ni
mercado exterior de trigo e indujo al Gobierno de Prieto a buscar el
intercambio con el Brasil. Como consecuencia, en 1833 el
Plenipotenciario Zañartu abandonó Lima y en 1835 se firmó con el Perú un
tratado de amistad y comercio que no tenía ---a juicio de Portales---
otro objeto que dar tiempo al enemigo para completar sus aprestos
bélicos. Sólo él vio claramente la realidad de esos días. Opinaba que la
Confederación iba a la guerra de todas maneras, porque el mestizo Santa
Cruz quería a todo trance la restauración del Virreinato, sometiendo a
Chile, Ecuador y el norte argentino, y no toleraba además que el mísero
Chile, hasta ayer dependiente del Perú, se levantara ahora como el
campeón del comercio en el Pacífico, con Valparaíso convertido en
competidor del Callao y con su marina mercante penetrando en
Centroamérica, Polinesia y Australia. Y ni una cosa ni otra, el
predominio en el mar y la soberanía de su patria, iban a perderse
mientras Portales tuviera las riendas del poder en sus manos.

Por eso su primera medida al retornar al Ministerio fue pedir a su
colega de Hacienda, Tocornal, el financiamiento de una escuadra que
debía componerse de dos fragatas, dos corbetas, un bergantín y una
goleta.

El Congreso aprobó el gasto, pero no había tiempo de comprar o mandar
construir los buques\ldots Entonces la imaginación sobreexcitada del
Ministro concibió el proyecto sin precedentes, tachado de inaudito y de
loco, de apoderarse de la escuadra de Santa Cruz por sorpresa para
atacarlo con ella antes que él atacara.

Inaudito y loco, evidentemente; pero a su juicio no había otro recurso
de salvación frente a un enemigo cuatro veces más fuerte y dirigido por
el más ambicioso caudillo y solapado intrigante que hubiérase visto en
América. Lo peligroso en él era su extraordinaria inteligencia, rayana
en el genio, que le había permitido organizar y disciplinar la olla de
grillos de Bolivia hasta capacitarla para dominar al Perú y conquistar
los parabienes de Francia e Inglaterra el día en que declaró establecida
la Confederación.

Cuando el Ministro dio la voz de alarma, vivía refugiada en Quillota la
viuda de Salaverry, el general peruano mandado fusilar por Santa Cruz, y
su luto era como una advertencia del peligro que acechaba a los
chilenos. Por otro lado, el general Freire seguía conspirando contra
Prieto desde Lima, ahora con el estímulo desembozado de la
Confederación, en tanto que el espionaje boliviano en Santiago era
dirigido casi sin disimulo por el agente diplomático Manuel de la Cruz
Méndez. Sus sutiles intrigas minaban el Ejército con la connivencia de
los cesantes de Lircay, difundiendo la idea venenosa de que Portales
pretendía sustituirlo por la Guardia Cívica. A un hombre de toda la
confianza del Ministro, el coronel Vidaurre, lograron meterle en cabeza
que Su Señoría buscaba la guerra con la Confederación para aniquilar a
los militares y perpetuarse en el mando.

Con todas estas evidencias, los consejeros de Prieto seguían creyendo en
la buena fe de Santa Cruz y considerando descabellados los planes
belicistas de Portales. Fue menester que el Cónsul en Lima, Lavalle,
comunicara los aprestos y la salida de Freire para Chile con dos buques
armados y financiados por la Confederación, para que en Santiago
abrieran por fin los ojos. La información, traída por una goleta
expresamente fletada, revelaba que Freire y un centenar de chilenos se
dirigían a Chiloé en el bergantín Orbegoso, de cuatro cañones, y la
fragata Monteagudo; de doce, llevando en bodega veintitrés cajones de
fusiles y carabinas, proyectiles de artillería, pertrechos menores y
dinero para organizar una sublevación y dar comienzo a la guerra civil.

Apenas salida la expedición del Callao, Santa Cruz envió a su
diplomático en Santiago una nota calculada para ser interceptada ---como
lo fue--- en la cual ``desaprobaba altamente'' la conducta de Freire,
``como desaprobaré siempre lo que propenda a turbar el orden de los
Estados americanos''. Tal era la doblez y la astucia del enemigo que
Chile se había echado encima.

Pero la jugada no engañó a Prieto, y Portales, el visionario, quedó
dueño y señor de la política exterior del país. Nunca tuvo tanto poder y
prestigio como en esos días cruciales en que la clase dirigente volvió
sus ojos hacia él y el Presidente parece haberle dicho: Haga lo que
quiera.

Ahora sería lo que una vez ofreció ser: el Ministro Salteador; y su
duelo con Santa Cruz iba a decidir la suerte de cinco países.

La ``empresa criminal de Freire'' ---así la llamó O'Higgins--- salvó a la
patria de convertirse en provincia de un imperio. Sin sospecharlo, el
eterno conspirador antiportaliano venía a servir los fines de Portales;
a tal punto es cierto que los actos de los hombres obedecen a designios
providenciales, y lo más profundo en metafísica es que nadie sabe para
quién trabaja.

A la altura de Valparaíso se amotinó la tripulación chilena de la
Monteagudo para entregarla a las autoridades. Siete días después este
buque salió a dar caza al Orbegoso, que ya se había apoderado de Ancud,
lugar en donde Freire cayó con sus cómplices cuando festejaba la efímera
viçtoria.

El enardecido Portales no esperó el resultado de esa operación naval. El
mismo día del zarpe de la Monteagudo despachaba con destino al Callao al
bergantín Aquiles y la goleta Colo-Colo, que hasta la víspera
constituían toda la marina de guerra de Chile. A cargo de la misión iba
el coronel Victorino Garrido, un español de rompe y raja, y como
comandante del Aquiles y de la dotación de asalto, el capitán Pedro
Angulo, ave de presa en quien el Ministro había puesto su ojo certero.
Objetivo de la incursión: quitarle sus buques a Santa Cruz.

Sin mediar declaración de hostilidades, Angulo se apoderó en un golpe
nocturno de cuatro barcos que. estaban amarrados al pie de los fuertes
del Callao. Echó a pique el que .no le, servía y sacó los• tres
restantes: la Santa Cruz, la Peruviana y el Arequipeño sin causar una
muerte entre las tripulaciones. El propio Lord Cochrane. no hizo nunca
nada igual, y en la historia de, país alguno se registra un episodio
semejante: arrebatarle media escuadra al enemigo para embotellarlo en
sus puertos y luego atacarlo con sus propias naves.

Este zarpazo mortal sorprendió a Santa Cruz el día en que se homenajeaba
a sí mismo con una brillante parada militar, y le dejó turulato,
humillado y hundido en el ridículo. Su primera reacción fue meter a la
cárcel al cónsul chileno, lo que equivalía a echar leña a la hoguera.
Los nacionalistas peruanos bailaron de júbilo en las calles y
congratularon a Victorino Garrido cuando éste desembarcó de uniforme
para pasear por Lima y asistir al teatro\ldots Después vendrían los cambios
de notas, las negociaciones y el tratado de paz Garrido-Santa Cruz, que
Portales desautorizó y tiró al canasto; lo que cuenta es que Chile
señoreaba ahora en el mar con una flota de siete unidades (incluidas las
dos quitadas a Freire), y este vuelco de repercusión continental había
dejado sellada la suerte de la Confederación perú-boliviana.

En rápida sucesión de medidas el Gobierno chileno apartó los escollos y
personas que obstaculizaban el camino hacia el objetivo supremo. Hizo
salir a Freire desterrado a Australia, expulsó al diplomático De la
Cruz-Méndez, obtuvo del Congreso facultades extraordinarias y declaró la
guerra.

El orden riguroso de las finanzas, obra de Rengifo y Tocornal, iba a
hacer posible un nuevo milagro: sostener la campaña con las entradas y
recursos ordinarios del Estado.

Cierto que el Ejército expedicionario sólo contaría tres mil hombres,
contra doce mil del enemigo; pero Portales nunca se detuvo a pensar en
la inferioridad numérica, en la desventaja del clima ni en la eventual
defección de la recluta peruana que esperaba reunir. Su decisión era a
prueba de consideraciones negativas, a prueba de dudas, porque se
fundaba en el más trascendental de sus sueños políticos, claramente
expresado en su carta al almirante Blanco Encalada: ``La Confederación
debe desaparecer del escenario de América, y nosotros los chilenos
debemos dominar para siempre en el Pacífico''.

\chapter{EN QUILLOTA CAE EL ILUSTRE VARON}

A lo largo de su carrera ministerial, Portales se caracteriza por la
rara habilidad, tal vez intuitiva, con que escogió a sus grandes
ejecutivos. La mayoría fueron hallazgos suyos, talentos hasta entonces
ocultos que él detectó y aprovechó con resultado asombroso. Confió el
mando del Ejército al apacible general Prieto, y éste dio en Lircay una
batalla digna de Napoleón. De Rengifo y Tocornal hizo dos Ministros de
Estado que ilustran la historia de la República. En el joven Manuel
Montt, de veinte años, adivinó ``algo'' y lo inició dándole un puesto de
amanuense. Sacó del anonimato a Andrés Bello, futuro fundador de la
Universidad y redactor del Código Civil. Para llevar a cabo la fabulosa
captura de la escuadrilla peruana eligió al capitán Angulo, antiguo
oficial mercante que vegetaba en la Capitanía del puerto de Valparaíso.
Era infalible: encontraba para cada destino al colaborador preciso.

Desde que concibió la más audaz de sus empresas, la guerra preventiva
contra la Confederación perú-boliviana, pensó en José Antonio Vidaurre
para el nombramiento clave de jefe del Estado Mayor de la expedición.
Con este fin aceleró su ascenso en el regimiento Maipo, y a los treinta
y cinco años este coronel era envidiado por el grueso de sus colegas. En
prenda de amistad (que Vidaurre no correspondía, pues jamás simpatizó
con su favorecedor), el Ministro le hizo obsequio de una hermosa espada
con empuñadura de plata. Como era de estatura más bien reducida y el
arma le caía incómoda, el comandante se la endosó poco después a su
hijastro, el turbulento y tabernario teniente Florín.

¿Se había equivocado esta vez el descubridor de hombres idóneos,
eligiendo al que iba a sublevarse y hasta proporcionando el arma con que
el asesino debía ultimarle?

La historia escrita a lo notario dice que sí, • que Portales cometió un
error y Vidaurre una traición\ldots

Pero ya es tiempo de que alguien se decida a examinar los hechos por
debajo de la trama, para ver si esta tragedia no tuvo a lo mejor un
sentido completamente distinto del que muestran las apariencias.

Un observador superficial habría recibido la impresión de que la guerra
se preparaba bajo inmejorables augurios. El Día de la Virgen de 1836
entraba a Valparaíso el más lucido buque de la escuadra peruana, llamado
precisamente Libertad, del que se habían apoderado sus oficiales,
enemigos de la dominación boliviana, para entregarlo a la Armada
chilena. Esta inesperada adquisición dejó a Blanco Encalada con ocho
barcos de línea, suficientes para dominar en la ruta al Callao. Ya
prácticamente sin fuerzas navales, Santa Cruz pensó invadir a Chile por
tierra, cruzando el desierto; pero, falto de la audacia que sobraba a
Portales, abandonó la idea, disculpándose en la escasez de agua, y optó
por esperar el ataque repartiendo en lugares estratégicos su Ejército
cuatro veces más numeroso. Su ``carta de triunfo'' era la soledad de
Chile, porque Ecuador no entraba a la guerra, y en Argentina, el
orgulloso Rosas había rehusado la alianza propuesta por el Gobierno de
Santiago.

Acamparon en Las Tablas, cerca de Valparaíso, los regimientos Maipo y
Valdivia, un escuadrón de cazadores y. una columna de emigrados peruanos
entre los que descollaba el general Castilla. Mientras Tocornal
financiaba la expedición sin endeudar al Estado ni gravar a los
particulares, Portales ordenaba que la recluta se hiciera entre solteros
desocupados, o casados con la aprobación de sus esposas, y que todos
fueran voluntarios, pues ``no hay que afligir injustamente a ninguna
madre ni a ninguna mujer de hombre honrado\\ldots''

Pero bajo las auspiciosas apariencias existía terreno movedizo. El
dinero y las intrigas de los agentes de Santa Cruz habían dejado en
marcha una conspiración que tenía por objeto impedir la campaña sobre el
Perú mediante el asesinato del Presidente Prieto y su Ministro de
Guerra. El primer atentado fue encargado al homicida prófugo Nicolás
Cuevas, al que introdujeron en una casa de la calle Santo Domingo
situada frente a la de don Antonio Garfias, donde a la fecha vivía
Portales. Avisado a tiempo por un papel anónimo, Su Señoría sorprendió a
Cuevas en persona, secundado por una pareja de guardianes, le quitó las
pistolas cargadas que ocultaba en su cuarto y lo hizo apresar. Vino
después la tentativa de soborno al comandante de la escolta presidencial
para capturar a Prieto y negociar su vida a cambio de que el general
Bulnes desarmara el Ejército del sur; plan frustrado por la lealtad de
Soto Aguilar, que entregó las dos bolsas de oro a Su Excelencia y
denunció a los instigadores Hidalgo y Fontecilla.

Portales, enemigo de las medidas draconianas, pensó que toda dureza era
poca para castigar a los vendidos que amenazaban la estabilidad interna
en momentos en que el país se jugaba su suerte. Del 2 de febrero de 1837
data su ley terrible de los Consejos de Guerra Permanentes, remachada
por otra del 27 de marzo que fijaba la pena de muerte ``dentro de
veinticuatro horas'', y sin apelación, para los conspiradores convictos.

Nadie creyó que podría llegarse a ese extremo, y consta en documentos
que el Ministro era el que menos lo deseaba. Pero el complot de tres
agricultores pipiolos para sublevar el batallón cívico de San Fernando y
asesinar al intendente de Colchagua no dejó otra alternativa que aplicar
la ley con todo su rigor; porque la menor señal de debilidad, en el
punto adonde habían llegado las cosas, significaba abrir las compuertas
del caos. Confesos los reos, señores Valenzuela, Barros y Arriagada,
fueron pasados por las armas en la plaza de Curicó en medio del
estremecido horror de los vecinos.

Sucedió lo inevitable en una sociedad inmadura: una marea de
impopularidad comenzó a levantarse contra el Ministro. La guerra, cuya
necesidad nunca fue bien comprendida, acabó por repugnar a todos como el
capricho de un mandón empecinado. Extendióse el rumor de que no había
expedición, que el Ejército sólo esperaba la orden de insurrección.
Bulnes y Blanco Encalada advirtieron a Portales que Vidaurre tramaba en
su contra; y Constanza Nordenflycht, la madre de sus hijos, viajó desde
Valparaíso para implorarle que extremara sus precauciones\ldots Sordo a las
voces de alarma, el Ministro seguía paseándose sin guardaespaldas, como
corresponde a un valiente, y dando seguridades de la lealtad de su jefe
de Estado Mayor. ``Nunca dudo de mis amigos, como que estoy seguro de que
no me traicionarán''. Llegó a tal punto la insistencia de las denuncias,
que por último interpeló a Vidaurre: ``Comandante, me dicen que usted me
hará revolución''; y obtuvo esta fría y clarísima respuesta: ``Señor
Ministro, cuando yo le haga revolución, Su Señoría será el primero en
saberlo''.

Como Ministro de la Guerra no podía ignorar que Vidaurre gozaba fama de
deliberante, que el regimiento Maipo se había sublevado en 1821 y
amotinado en 1828; que Santiago Florín, en estado de ebriedad, se
acriminó una vez en Concepción y otra en la Quiriquina; que era fanático
partidario de Freire y llevaba al cinto la espada que don Diego
obsequiara a Vidaurre\ldots

¿Cómo explicarse de manera satisfactoria el drama inminente de Quillota
y la tragedia de El Barón? Ningún historiador ha filosofado en torno a
su enigma. ¿Cómo se entiende que Portales, el que olfateaba las
conspiraciones desde lejos y las deshacía una tras otra a manotazos, no
tomara ninguna medida en presencia de un motín anunciado a los cuatro
vientos y cuyo propio gestor le había predicho? Una sospecha ha estado
gritando desde 1837 sin que nadie le prestase oídos: ¿se metió el
Ministro deliberadamente en la boca del lobo? ¿Buscó el sacrificio de su
vida como único recurso a su alcance para sacudir a ese pueblo
indiferente y ciego, para encender con su martirio el patriotismo
dormido y sacar adelante la guerra con que soñaba salvarlo\ldots? Sospecha
imposible de confirmar en este mundo, pero que seguirá dando voces
eternas, porque sin ella todo cae en el absurdo. ¿Habrá sido don Diego
Portales un espíritu sublime, un superhombre, un santo? ¿No puede una
nación en pañales, en los confines de la tierra, haber producido
semejante arquetipo universal, semejante argumento digno de la tragedia
griega?

De ser así, no hubo error de parte del héroe al elegir al que encarnaría
el papel de ``traidor''. Lo escogió con su clarividencia que no fallaba,
como escogiera a todos los grandes personajes de su reparto: a Ovalle, a
Prieto, Rengifo, Tocornal, Bello, Angulo y Montt. Con la sola diferencia
de que a José Antonio Vidaurre lo ofendió y desairó una y otra vez, al
mismo tiempo que lo distinguía y ascendía, para provocar el
resentimiento en su alma tortuosa. Así ``preparado'', Vidaurre llegó a
detestarle y disfrazó su odio y su conspiración con la excusa de que la
expedición al Perú tenía. por objeto destruir el Ejército para afirmar a
un déspota en el poder.

Todo estaba cuidadosamente dispuesto; ¡ya podía levantarse el telón!

Después de revistar la escuadra y el convoy de dieciséis transportes
pertrechados, Su Señoría se dirigió al cantón de Quillota a pasar
revista a los mil quinientos hombres del Maipo. Viajaba en un birlocho
de alquiler en compañía de su secretario Cavada y el coronel Necochea, y
sin más escolta que los nueve húsares de Soto Aguilar. En Valparaíso,
dos de sus amigos y el superior de la Merced habían tratado de
disuadirle del paso suicida que iba a dar, Una leyenda lugareña refiere
que un ángel se esforzaba en detener el cochecillo en el camino,
mientras el diablo lo empujaba por la culata.

Al descender el Ministro en la desértica plaza de Quillota, frente a la
puerta del gobernador, acudió Vidaurre a saludarle; y la nerviosidad del
comandante llamó la atención de Necochea. Posteriormente se. supo que
estaba insomne y pasaba las noches dando vueltas en la cama y suspirando
como un desesperado.

Igual desazón mostraban al día siguiente los Oficiales del regimiento al
dar comienzo a la parada. Dos o tres veces equivocaron las voces de
mando a unos soldados de instrucción descuidada y a los que se había
asegurado que no irían a la guerra. El plan consistía en simular que era
la oficialidad la que se sublevaba, para en seguida presentar al
comandante el ``hecho consumado'' a fin de hacerle aparecer ante el
Ministro como ``sometiéndose a la voluntad de la mayoría\ldots'' Así se hizo,
y en una de sus defectuosas evoluciones, dos de las compañías
encerraron. sorpresivamente a la víctima y sus acompañantes dentro de un
cerco de fusiles con bayoneta calada. Todo pasó en un minuto o menos. El
capitán Narciso Carvallo sé adelantó diciendo con arrogancia: ``Dése
preso, porque así conviene a la República''; luego ordenó a los soldados
retirar. armas, explicándoles que debían ser ``generosos''. En ese
instante llegaba corriendo: el capitán Arriaga a la cabeza de su
compañía, y, con la precipitación de una escena mal ensayada apuntó sus
pistolas al pecho de Portales. Desde el centro de la plaza gritó
entonces Vidaurre: ``¿Qué tumulto es ése?'' A la declaración del
pronunciamiento, de que fue ``portavoz'' Carvallo, contestó el comandante:
``Señores, estoy con ustedes. ¡Viva la República! ¡No más tiranos!'' Y sin
otro expediente, el Ministro y su comitiva fueron sacados de la plaza
arreándoles con la punta de las bayonetas. El teniente coronel García
intentó desbaratar el motín agrediendo a Vidaurre a sablazos, pero fue
apresado y conducido a los calabozos preparados de antemano en la vecina
Casa de Ejercicios.

Así se consumó el motín porfiadamente provocado por Portales y
atolondradamente tramado por Vidaurre, que fue incapaz de darle
ramificaciones y lo dejó reducido a cuartelazo local condenado a
frustrarse.

El ilustre prisionero pudo habérselo hecho notar, pues sabía que el
Maipo estaba solo en su aventura, que tenía diez proyectiles por hombre,
nada más que diez, y que en el enfrentamiento inevitable iba a ser
aplastado por los otros regimientos y por la milicia cívica. Pero nada
dijo, y con resignación y dignidad que ningún historiador ha analizado,
se dejó remachar la barra de grillos.

Ni un grito, aplauso o llanto se oyó en las calles del pueblecillo
cuando se lo llevaron en el birlocho por el camino del puerto. En la
parada de Tabolango, donde comió al cabo de veinticuatro horas de
hambre, Vidaurre le ordenó escribir a Blanco Encalada invitándole a
rendirse en, aras de la paz. Como el caído vacilase, Florín
(¡recientemente ascendido por él a capitán!) le amenazó: ``Si no escribe,
se le darán cuatro tiros. Hace tiempo que debíamos haberlo fusilado''.
Portales contestó: ``En nada estimo mi vida\ldots''; y redactó la carta al
almirante y al gobernador militar de Valparaíso, a sabiendas de que no
rendirían la plaza.

El descabellado motín se fundaba en la idea de extorsionar al Gobierno
amenazando con la muerte del Ministro si el resto de las fuerzas no se
plegaba a la sublevación. Vidaurre era incapaz de cometer el crimen más
atroz de la historia nacional; pero llegado el caso, tenía para eso a
Florín, al que bastaría entregarle la custodia del preso para que éste
fuera sacrificado. En el proceso declara Florín que su jefe y padrastro
le dio esa orden precisa; pero ¿cómo creerle al autor de dos homicidios
impunes y borracho consuetudinario, que iba casi cayéndose del caballo?
Vidaurre no necesitaba darle orden alguna, y la prueba es que Florín
ofreció disparar sobre el Ministro en Tabolango, antes de reemplazar al
oficial que iba custodiándole.

El dispositivo mortal funcionó automáticamente a partir de la parada en
Viña del Mar, donde Vidaurre supo que Valparaíso lo esperaba en pie de
guerra. Ya no cabía la negociación, y Portales sobraba. Todavía más: si
vivía, el muerto sería él, Vidaurre, con seguridad absoluta, mientras
que si moría el Ministro le quedaba la esperanza de sobrevivir porque
sólo Portales era capaz de mandar al patíbulo a un coronel. Entonces dio
la orden de que Florín relevara al capitán Díaz en la custodia del
prisionero.

Nadie más que Florín era hombre de hacer lo que hizo; de ahí lo acertado
de su elección por Vidaurre y lo acertado de la elección de Vidaurre por
Portales. Nadie sino Florín podía hacerlo bajarse del birlocho en el
alto de El Barón, sin siquiera quitarle los grillos, al sentir el
tiroteo del combate nocturno de Valparaíso, para proceder a su
fusilamiento. Previamente mandó matar a Cavada por el solo hecho de
intentar huir, y a Necochea lo dejó vivo para que oyera su declaración
de que fusilaba al Ministro por orden expresa de Vidaurre. Como los seis
soldados se resistieran a disparar, Florín tuvo que repetir dos veces la
orden de fuego. Producido el contagio sanguinario, llovieron las balas
sobre el rostro y luego sobre el cuerpo del caído, y en esa masa
convulsa de la que escapaban gemidos y gritos se ensañaron las bayonetas
hiriéndole treinta y cinco veces; y por último el oficial ebrio de vino
y sangre le asestó la estocada de remate con la espada que la víctima le
había proporcionado por conducto indirecto.

Se confundieron las detonaciones del crimen con las descargas cerradas y
los cañonazos de la escuadra en la caja de resonancia de la quebrada de
El Barón. Dantesca batalla en tinieblas donde el regimiento amotinado
peleaba casi sin municiones y que acabó con su desbande y
aniquilamiento.

En un recodo del camino quedaba el mártir de cuarenta y cuatro años,
acribillado y desangrado, pero ahora más poderoso que nunca. Era la
Guardia Cívica de Valparaíso, creada por él, la que había decidido la
victoria de las fuerzas leales de Blanco Encalada. Y fue la rígida
justicia militar portaliana la que castigaría a los reos con su
escarmiento memorable: ocho penas capitales cumplidas en la plaza de
Orrego en presencia del pueblo. El cadáver de Florín fue descuartizado a
hachazos y la cabeza de Vidaurre cortada para ser exhibida en la plaza
de Quillota ensartada en una pica.

Estos dos desdichados, a los que todos llamaron felones, asesinos y
traidores a la patria, ¿qué es lo que habían hecho en realidad? Si la
historia es algo más que un libro de actas, hay que usarla para sacar
conclusiones y trasladarlas al plano filosófico. Creyendo derribar el
régimen de Portales, Vidaurre y Florín lo reafirmaron porque la
ciudadanía volvió sus ojos hacia el Presidente Prieto, temblando de
miedo ante el peligro de que renaciera la anarquía, y dispuestos todos a
apoyar a su Gobierno sin fijarse en diferencias ideológicas. Creyendo
que paralizarían la expedición al Perú, Vidaurre y su secuaz diéronle un
formidable impulso cuando quedó comprobado que el motín y el asesinato
eran obra de las intrigas y sobornos de Santa Cruz. Ardió entonces la
llama del patriotismo, ésa que el propio Ministro no había conseguido
atizar en vida, y el pueblo, la aristocracia, los soldados, los marinos
y los reclutas coincidieron en su impaciencia por atacar en seguida para
destruir al que pretendía avasallarles desde suelo extraño.

Si
Portales marchó conscientemente al sacrificio con ese fin --- porque
veía que la empresa dejaba frío al país e iba al fracaso--- quiere decir
que Florín y su jefe sirvieron con eficiencia portentosa a los planes
del hombre que odiaban. Lo salvaron de un más que probable eclipse
político e histórico al conferirle la fuerza sobrehumana y eterna que se
deriva del martirio. En una palabra, construyeron el pedestal de su
estatua y fueron los inconscientes
estrategas de la guerra que Portales ganó desde ultratumba. Porque en
última instancia, a ellos se deben las aplastantes victorias del general
Bulnes en Guías, Buin y Yungay y el derrumbe y fuga de Santa Cruz, ese
otro gran artífice indirecto de la unidad y el orgullo nacional de los
chilenos.

\chapter{ANGULO}

A medida que vaya adentrándose en esta crónica, se preguntará el lector
por qué don Pedro Angulo ocupa sólo unas líneas en la historia naval de
Chile y cómo es posible que ni una torpedera lleve su nombre\ldots Nadie
sabe dar razón, y persiste el hecho asombroso de que el más original de
nuestros marinos de presa, ejecutor de una hazaña sin paralelo en el
mundo, sea un solemne desconocido. Las pruebas documentales existen y el
retrato del personaje está a la vista en el Museo Histórico Nacional. De
haber hecho para Inglaterra lo que hizo para Chile, su hoja de servicios
se enseñaría a los cadetes de la Royal Navy como ejemplo de habilidad y
coraje en ``misiones imposibles''. Encina, único historiador general que
le hace justicia, atribuye su semi anonimato a ``la falta de lustre
social'' y a ``la repulsión invencible de la aristocracia castellano-vasca
por toda aptitud superior\ldots''; explicación que no aclara en un décimo el
misterio de esta obscuridad contemporánea y de esta postergación
póstuma.

En los días en que el ojo de Portales se posó en él, Pedro Angulo Novoa
ocupaba el cargo de Capitán de Puerto de Valparaíso. Servía este puesto
desde 1831 y por expresa decisión del Ministro, quien al remitirle el
nombramiento le escribió expresándole: ``siendo Ud. de los mejores
empleados en los destinos públicos, he tenido a bien mandar extender el
despacho que incluyo, para que haciendo tomar razón de él en la
Comandancia y Comisaría de Marina, lo mantenga en su poder como debe
ser. Dios guarde a Vd. muchos años. ---Diego Portales''.

Era Angulo un hombre de expresión tranquila pero enérgica, según 'lo
muestra el retrato, de rostro agradable con patillas sanmartinescas, y
por lo que alcanza a verse, debió poseer una ancha y fuerte con.
textura. Hijo de un naviero de Concepción, sirvió en sus barcos como
piloto y capitán. Al organizarse la Escuadra Libertadora, Lord Cochrane
le confió el mando del transporte Hércules. En la costa peruana fue
capturado por un corsario realista que lo envió prisionero a las
casamatas del Callao. Permaneció allí hasta las postrimerías de la
guerra de la Independencia y sólo le sacaron del encierro para
embarcarlo con el contingente de presos que el Virrey ordenó remitir a
España en los últimos barcos que habían escapado de la persecución de
los patriotas. La flotilla aportó en la isla Guam, archipiélago de los
Ladrones, para hacer aguada. La noche misma del arribo, Angulo intentó
adueñarse del buque-cárcel Clarington para huir en él. Descubierto en
sus planes, le prendió fuego, obligando a la tripulación a abandonarlo.
A raíz del atentado lo trasladaron al Aquiles; pero cuatro días más
tarde consiguió burlar la vigilancia para repetir la tentativa. Por un
descuido inverosímil habían dejado el armero abierto. El conspirador
tenía comprometidos a cinco de sus coterráneos, a dos peruanos, dos
colombianos y dos españoles. Al filo de la medianoche se apoderaron por
sorpresa de pistolas y machetes, y en un santiamén redujeron a la
dotación, de capitán a paje, sin causarle otro daño que el susto y
ligeras contusiones. Después de desembarcarla en los botes, el Aquiles
emprendió la fuga amparado por las sombras. Era un dos-palos de
trescientas cuarentas toneladas y veinte cañones; ``un hermoso bergantín''
al decir del doctor Page, que lo vio meciéndose al pie del Cerro Alegre.
Su impávido captor lo condujo hasta los 36 Norte y desde allí puso proa
hacia la Alta California. El casco hacía agua y un ventarrón huracanado
le rifó parte del velamen. Con cuarenta y seis días de travesía
arribaron a Santa Bárbara para reponer los víveres agotados; pero la
ambigua actitud del gobernador mexicano les indujo a alejarse a toda
vela con rumbo al sur. Ahora padecieron hambre y sed, comiendo restos de
carne rancia y peces voladores y bebiendo agualluvia recogida en trozos
de lona. En otros cuarenta y siete días, sufriendo los síntomas del
escorbuto, llegaron a Valparaíso. El Aquiles entró a puerto luciendo un
improvisado pabellón nacional y Pedro Angulo entregó la presa a las
autoridades, ``deseoso'', escribió, ``de dar al público un testimonio de mi
adhesión a la gran causa de América''.

Fue el recuerdo de este incomparable episodio lo que movió a Portales a
distinguir al intrépido oficial, sacándole de la obscuridad del olvido
en 1831 y confiándole el rol estelar con que iba a culminar su carrera
en 1836.

Uno de los raros papeles de Angulo que se conservan, el que escribiera a
su benefactor a raíz del nombramiento en la Capitanía, pone al trasluz
su ardiente amor a la patria y su ansia de darle gloria en no importaba
qué lugar o circunstancia. ``El nombre de Chile'', dice, ``es para mí una
palabra mágica que promueve en mi pecho todas las ideas generosas, que
me hace arrostrar los peligros en cualquier punto del globo, en las
playas del Perú como en los remotos mares que circundan las islas
Marianas, en la península de California como en el puerto de
Valparaíso''. Once años después de su captura en Guam, el Aquiles
continuaba en servicio activo, exhibiendo la insignia de una ``fuerza
naval'' que compartía con la destartalada goleta Colo-Colo. Y desde hacía
cinco años el comandante Angulo vegetaba en su puesto de tierra firme,
seguramente anheloso como un gavilán al que hubiesen amarrado las alas y
las garras\ldots Lo que menos pudo esperar es que le sacarían de esa jaula
burocrática para soltarlo a volar en la incursión más audaz que hubieran
visto estos mares.

La empresa que iba a encomendarle Portales determinó que el Ministro
fuera tildado de loco por los ponderados consejeros del Presidente
Prieto. Portales sostenía que el Protector Santa Cruz proyectaba invadir
a Chile para anexarlo a la Confederación perú-boliviana; con el objeto
de impedirlo había que adelantarse invadiendo el Perú; y como Chile
carecía de poder naval, era menester apoderarse previamente de la
escuadra confederada. Vale decir (exclamaban los consejeros) un plan sin
pies ni cabeza y humanamente irrealizable\ldots Pero como Su Señoría les
da.ba a elegir entre dejarle hacer o aceptar la dimisión de sus tres
carteras ministeriales ---y ausente Portales no había quién contuviera
la anarquía------ el Consejo de Estado se resignó a patrocinar esa
aventura temeraria en la que el país se jugaría su suerte.

Para apreciarla en lo que vale, recordemos que la población conjunta de
Bolivia y Perú triplicaba . a la de Chile, en tanto que el ejército
confederado era cuatro veces mayor y la escuadra del Callao sumaba
quince buques contra dos de Valparaíso\ldots Utilizando este par de
cascarillas pretendía don Diego Portales recuperar el dominio del
Pacífico Sur conquistado por O'Higgins. Con sobrada razón habían hecho
mofa de sus planes; pero él pensaba con la clarividencia del genio y
sabía que Santa Cruz no esperaba semejante zarpazo e iba a pillarlo
absolutamente desprevenido. Las relaciones con Lima presagiaban un
conflicto armado que los demás estadistas no eran capaces de prever ni
de prevenir. El país estaba minado por el espionaje boliviano, la
Confederación negábase a cancelar las deudas de la Independencia,
hostilizaba el comercio chileno, celosa del predominio mercantil de
Valparaíso, y alistaba a toda prisa el ejército más poderoso del
continente. Con todo, el astuto enemigo concebiría cualquier jugada
menos la que le estaban preparando.

En julio de 1836 Portales ordenó a Angulo pertrechar goleta y bergantín
y designó como jefe de la expedición al coronel Victorino Garrido. Era
éste un español retinto y de barba salvaje, que no sabía sino andar a
empellones. Su misión, dicho en lenguaje sencillo, consistía en ir a
buscar camorra, y la de su subordinado, en traerse todos los buques que
fuera posible quitar al enemigo.

Los que aún se resistían a creer en los propósitos imperialistas de
Santa Cruz, cambiaron de parecer cuando dos fragatas peruanas
aparecieron en aguas chilenas. Traían a su bordo a don Ramón Freire y
sus secuaces políticos y militares, conduciendo armamento y caudales
facilitados por misteriosos financistas limeños con el fin de fomentar
en Chile un golpe revolucionario. Sólo entonces la opinión pública abrió
los ojos.

En las cercanías de Valparaíso se amotinó la marinería de la Monteagudo
para entregarla precisamente al Aquiles, a cuyo mando estaba ya el
impaciente Angulo. Horas después se hacía de nuevo a la mar, con
tripulación nacional, para dar caza al Orbegoso y a Freire. La marina de
Portales tenía ahora tres barcos; una semana después tendría cuatro.

Junto con zarpar la Monteagudo para Chiloé, partieron Garrido y Angulo
con destino al Perú. Fue todo tan rápido y sigiloso que nada llegó a
conocimiento de los espías confederados.

A las 9 horas del 21 de agosto el Aquiles entró al Callao y saludó la
plaza con las salvas de rigor. Contestaron los fuertes, cuyo comandante
era el general chileno Oscar Herrera. Mirando con el catalejo, Angulo
observó cuatro buques de guerra fondeados al abrigo de las baterías.
Eran la corbeta Santa Cruz, el bergantín Arequipeño y las goletas
Peruviana y Congreso. Inmediatamente Garrido desembarcó para
entrevistarse con Herrera en visita protocolar. Supo que don Andrés de
Santa Cruz preparaba para el día siguiente una parada militar, a la que
seguiría una recepción de gala en palacio. Averiguó también que la
escuadrilla estaba a medio tripular y la goleta Congreso hallábase en
reparaciones. Terminó la entrevista sin que Herrera lograra entender a
qué había ido Garrido al Perú. El uniforme chileno del coronel llamó la
atención de los transeúntes y uno o dos le susurraron al pasar: ``¿Cuándo
vienen a librarnos de los bolivianos?''

A las 12 en punto de la noche Angulo echó al agua cinco botes,
distribuyendo en sus bancadas y bordas ochenta individuos armados de
machetes de abordaje y. garrotes. No llevaban armas de fuego, pues
debían evitar la alarma de los centinelas nocturnos de la fortaleza. De
otra parte, las instrucciones escritas de Portales recomendaban:
``Respetar las vidas y propiedades de las dotaciones peruanas''.

Bogaron sin prisa, buscando en la obscuridad el sobresaliente edificio
del Arsenal. Protegida por su artillería estaba la Santa Cruz, el buque
insignia armado de doce cañones y tripulado por cuarenta y. tantos
hombres. Conduciendo a los suyos en persona, el comandante Angulo se
trepó a pulso por las cadenas de fondeo, y una vez invadida la cubierta
mandó atrancar las puertas del castillo y camarotes. Simultáneamente sus
marineros viraban las anclas, y mitad a vela y mitad remolcada por los
botes, la corbeta se deslizó hacia afuera de la bahía. Arrancada a su
sueño por los golpes de martillo en las puertas, la tripulación no tuvo
más que resignarse ante el hecho consumado y quedóse quieta en sus
encierros.

A la una de la madrugada los cinco botes abordaron el Arequipeño, de
nueve cañones y treinta y cuatro hombres que dormían a pierna suelta.
Cuando éstos despertaron, el bergantín ya iba navegando para reunirse
con el Aquiles y la Santa Cruz, y tampoco hubo esta vez amago de
resistencia.

Capturar y sacar la Peruviana, que estaba amarrada a los pies del
Arsenal, fue más fácil todavía, pues no había alma viviente a bordo. Y
en cuanto al desmantelado Congreso, todo se redujo a abrirle las
válvulas para echarlo a pique.

Muertos, ninguno. Heridos, ninguno. Sin gasto de un gramo de pólvora
había cambiado el cuadro estratégico del Pacífico Sur, a las 3 de la
mañana del 22 de agosto de 1836. Ahora Portales tenía siete piezas en el
tablero y Santa Cruz nueve. Ya no estaba expedita la vía de Chile por el
mar, y en cambio quedaba posibilitada la invasión libertadora del Perú
por los chilenos.

Al salir el sol apareció el apostadero naval vacío. La Peruviana, el
Arequipeño y la Santa Cruz permanecían agrupados alrededor del Aquiles y
fuera del alcance del fuego de los fuertes. A través de los anteojos se
divisaba el humillo de las cocinas de a bordo en el acto de preparar los
desayunos.

A las 9 llegaba a Lima un jinete militar del Callao que entró al palacio
del Protector gritando: ``¡Se robaron la escuadra\ldots!'' Santa Cruz y sus
Ministros, allí reunidos para dirigirse a la parada, fueron actores de
una escena de gritos, puñetazos en las mesas y mutuas recriminaciones
por una falta de que todos eran culpables: mantener los barcos sin
guardia nocturna en tiempo de tensión exterior y en presencia de una
nave sospechosa. La Confederación acababa de sufrir su primera derrota y
tendría que afrontar el ridículo el mismo día en que iba a ser
proclamada con actos oficiales ante el Cuerpo Diplomático.

En su furor, el pequeño y teatral Santa Cruz sólo atinó a ordenar el
arresto del señor Lavalle, Encargado de Negocios del país ``agresor''.
Lloviendo sobre mojado, llegó la nota humillante de Garrido en que ni
siquiera se nombraba a la Confederación para no reconocer su existencia:

``La inexplicable conducta del Gobierno peruano ha obligado al mío a
tomar por su propia defensa las medidas de que US tendrá noticia por
otros conductos. La intención del Gobierno de Chile es retener los
buques como una prenda de las disposiciones pacíficas de la República
peruana, y con la mira, quizá, de devolverlos en el momento en que se le
den suficientes garantías de paz\\ldots''

La intervención apaciguadora de O'Higgins, que llegó a palacio apoyado
en su bastón de anciano, convenció al Protector de la conveniencia de
soltar a Lavalle\ldots Pero la sola idea de encarcelarle había sido una
osadía que Portales no iba a perdonar; como tampoco perdonaría a Garrido
el traspié de firmar con Santa Cruz, a bordo de un buque neutral, el
tratado de tregua y relaciones comerciales con que el boliviano
consiguió ablandar al español. El tratado iría al canasto y el coronel a
su casa.

Diez días después la escuadrilla se alejó embanderada -con sus nuevos
colores y mandados los navíos capturados por sus flamantes capitanes:
Domingo Salamanca (Santa Cruz), Pedro T. Martínez (Arequipeño) y Rafael
Soto Aguilar (Peruviana).

Cuando arribaron a Valparaíso para reunirse con la Monteagudo y la
Orbegoso\ldots, la gente aglomerada en la ribera y en los botes fleteros
atestiguó el toque final de humorismo de la triple conquista: en la proa
de cada barco venía atada una escoba, como señal de haberse barrido el
mar.

Sin variar el distingo que hacía entre Bolivia y Perú, el Ministro
ordenó que las banderas peruanas fueran repuestas ``hasta que el Gobierno
disponga otra cosa''.

Tal es el episodio que nuestros historiadores clásicos cuentan como al
pasar y que Vicuña Mackenna califica de ``odioso''. La captura naval más
limpia y perfecta en cualquier época y país, no les dice mucho ni poco.
Y peor es todavía la incapacidad para valorar su trascendencia
histórica. Basta una pizca de imaginación para darse cuenta de que
Angulo es el hombre que hizo posible desbaratar el sueño imperialista de
Santa Cruz. Concretamente, la guerra preventiva de Portales era
irrealizable sin la escuadra que el oscuro oficial ayudó a completar sin
gasto ni sangre. Su mérito es el de un estratega providencial que en la
primera escaramuza paraliza al enemigo y deja allanado el camino de la
victoria. Si fuésemos a medirlo por los frutos de su obra, no habría más
remedio que revisar la historia nacional para colocarle en la línea de
honor de sus héroes decisivos.

Héroe anfibio, por añadidura, porque el desconocido Pedro Angulo fue el
ayudante de campo de Blanco Encalada en El Barón, y en ese combate
librado en tinieblas diose el lujo de abatir de un pistoletazo a
Arrisaga, el militar amotinado que en Quillota intimara rendición al
Ministro Portales.

Querrán saber todos, ahora, cómo fue su desempeño en la guerra cuyo
resultado decidió él mismo\ldots Pero aquí viene el broche inconcebible de
su carrera: el ave de presa del Callao, del Barón y de Guam no
interviene en la Expedición Restauradora del Perú.

¿Por qué? No se sabe. La hoja de servicios lo pasa por -alto y sus
cartas y papeles personales se perdieron.

Sólo consta que entregó el mando del Aquiles a Roberto Simpson, para
volver a sentarse en la oficina de la Capitanía del puerto de
Valparaíso.

En su retrato luce tres medallas, pero nunca pasó más arriba del grado
de capitán de fragata que llevaba al cumplir su mejor hazaña.

\chapter{EN LOS DIAS DEL PESO DE LA NOCHE}

El período portaliano es abundante en noticias suministradas por
viajeros extranjeros. Este aporte es precioso para la historia de un
país entonces desconocido y que era raramente visitado por escritores. Y
de otro lado, el testimonio del observador foráneo es insubstituible por
la limpieza de la mirada desprejuiciada y por el relieve extraordinario
que el contraste imprime al país exótico puesto en comparación con el
del visitante. Desde Graham hasta Treutler, estos cronistas de paso han
sido la más generosa fuente de anécdotas y datos curiosos de que podamos
servirnos en la recreación del pasado. Tres de ellos parecen concertados
para iluminar el panorama de la vida chilena en los días de Portales.
Uno está recién traducido, otro permanece inédito; de manera que nadie
hasta ahora pudo utilizar el arsenal informativo contenido en sus
manuscritos. El autor inédito es Thomas Stokes Page, un joven cirujano
de New Jersey que vino a ejercer en Chile y fue testigo de memorables
acontecimientos. Su barco, el bergantín B. Mezick, arribó a Valparaíso
rodeando el Cabo de Hornos con ochenta y cinco días desde Burdeos. Algo
que no sorprendía a los chilenos quedó anotado en el Diario de Viaje del
doctor: la reciedumbre de los cargadores semidesnudos, a los que vio
echarse al hombro bultos de cien kilos durante la inhumana faena de diez
horas diarias. Eran los hombres-grúa del Chile viejo, alimentados con
leche al pie de la vaca, pescado fresco y porotos picantes. Uno de ellos
cargó el enorme baúl con libros, instrumental y cama del médico y caminó
dos cuadras hasta depositarlo en la Aduana.

Ciudad de treinta mil habitantes ---de los cuales seis mil eran
extranjeros--- consistía Valparaíso en una calle angosta y una plaza que
ocupaban todo el espacio llano entre la playa y la base de los cerros.
Con su ojo penetrante, captó Page imágenes que nadie supo valorizar
antes ni después; hecho doblemente encomiable en un relato escrito sólo
``para instruir e interesar a mis hermanos y hermanas menores\ldots'' Muchos
hablaron del viento de Valparaíso, pero sólo él se fijó en las
enfermedades de la vista que causaban las polvaredas y en la cantidad de
gente que se protegía de los surazos usando gafas verdes. Anotó que ``la
mitad de la población vive aparentemente a caballo''. Observó las jaurías
que asaltaban a los jinetes, a las rechinantes carretas y a las recuas
de mulas ocupadas en el acarreo del agua de las quebradas. Vio un pueblo
risueño, pero también respetuoso, ``que rara vez se dirige o pasa ante un
caballero o señora sin descubrirse''.

Dondequiera que mirase advertía la huella del exgobernador Portales.
Para reprimir el contrabando se prohibía que los botes se alejasen de la
playa después de la puesta del sol, como tampoco podían los de a bordo
venir a tierra. El anuncio de los serenos de la Colonia había sido
reemplazado por un enérgico: ``¡Viva Chile, las siete han dado!'' Al
hacerse de noche los buques de guerra disparaban un cañonazo y arriaban
su pabellón, lo que de inmediato imitaban las cuarenta naves al ancla en
la bahía. Otra noticia exclusiva es que la explotación del camino a
Santiago se remataba cada año a un concesionario que debía costear las
reparaciones de la carreta a cambio de la percepción del peaje. Sus
obreros eran los presos de la cárcel, que trabajaban vigilados por
guardianes con bala en boca y eran recogidos en la noche en carros de
rejas similares a los de los circos de fieras.

El Dario de Viaje de Page parece hoy un inventario de costumbres
desaparecidas, como aquella de que los entierros sólo podían hacerse en
horas nocturnas. Un reglamento precisaba que únicamente los viernes y
sábados se permitía mendigar en las calles; y entonces el gremio tomaba
su desquite pidiendo limosna a pie, en angarillas y (textual) ¡a
caballo\ldots! A la hora de la oración, al oírse las campanadas de las
iglesias, peatones y jinetes se detenían, descubriéndose, para murmurar
la plegaria de la tarde, y sólo reanudaban la marcha al escuchar el
último de los sones místicos.

La escena costumbrista de mayor carácter se producía en la plaza de
Orrego, al caer la noche, cuando los comerciantes abrían su feria al
aire libre. Alumbradas con velas o farolillos, las mercancías eran
expuestas en grandes canastos redondos; y hasta ahí llegaban las
mujeres, de la plebeya a la señora visible, para regatear los precios de
las telas, polvos, calzado y abalorios de ultramar. Curiosamente, la
mujer elegante sólo se dejaba ver en la plaza y en la iglesia. Para
salir adornaba su cabeza con un par de flores naturales y una enorme
peineta española de cincuenta centímetros de alto. Si iba a los
servicios religiosos se envolvía en su manto de brillante seda negra y
se hacía llevar por un sirviente la estera o alfombra que desplegaba
para arrodillarse. Esto ya lo sabíamos, pero el doctor Page añade que en
los intervalos de la misa las devotas se sentaban en dicha alfombra con
las piernas cruzadas a la usanza oriental\ldots{}

Temblaba una vez por semana y entre dos temblores se supo que un motín
militar había depuesto al Ministro Portales en Quillota, y que el
sublevado coronel Vidaurre marchaba sobre el puerto. Page vio cómo
Blanco Encalada organizaba el rechazo a toda prisa, poniendo las tropas
en pie de combate y requisando los caballos, ``excepto los de los médicos
y cónsules''. Esta caballería improvisada llenó pronto la plaza de Orrego
en tanto que los soldados cívicos se alistaban en la calle tirando
estocadas y mandobles.

Cuando los dos mil milicianos partieron al encuentro de Vidaurre, una
larga columna de curiosos, mujeres y hasta niños salió en su seguimiento
por el camino.

A las tres de la mañana el doctor divisó los fogonazos del bergantín de
guerra Arequipeño y de los batallones que ``sembraban su semilla de
muerte en la quebrada del Barón\ldots''

Consumada la victoria gobiernista, el cirujano ofreció su concurso para
socorrer a los heridos. Llegaban carretas cargadas con el armamento
abandonado por los fugitivos y ``en la bahía ondeaban las banderas a
medio mástil en. señal de duelo por el padre de Chile, el infortunado
Portales''. Tocó a Page atender a un soldado leal que agonizaba sin
soltar su fusil, y a otro que se paseaba con dos balas, en el muslo.
Pero su máxima contribución documental es la descripción de los restos
de Portales como testigo ocular y como espectador de sus exequias. Algo
hubieran dado Encina y Vicuña Mackenna por conocer esa página exclusiva
y macabra.

Le mostraron el desnudo cadáver del Ministro en su propia casa, situada
``al fondo del Almendral'', adonde lo condujeron después del asesinato
perpetrado al empezar el combate. ``Jamás'', reza el diario, ``he
presenciado un espectáculo tan horrible. Tenía como veinticinco heridas
de bayoneta, tres de bala y una o más de sable. Todas en el pecho y
abdomen, excepto una en la mano y otra de bala que entró por la boca y
pasó a través de la mejilla llevándose consigo todas las partes
intermedias''.

Contaron al doctor que Portales había dicho a sus victimarios:
``Malvados: yo moriré, pero mi sangre será vengada muy pronto, porque el
país no podrá soportar vuestro crimen''.

Las honras fúnebres costaron seis mil pesos fuertes. En el cortejo iba
el birlocho del Ministro, enlutado en negro y plata. El carro mortuorio
había sido decorado por un comerciante norteamericano y estaba
ostentosamente coronado con plumas de avestruz. Al frente del ataúd se
veían colgados los grillos que cargó la víctima, y mientras avanzaba el
enorme desfile hacia la iglesia Matriz, retumbaban las salvas de la
fragata Libertad entre los aires lúgubres de las bandas militares. Con
su último cañonazo el buque insignia izó el pabellón al tope y así lo
hicieron las banderas de todas las naves, cuarteles y edificios
oficiales.

Seguía temblando y entre dos remezones sacaron de un pontón a Vidaurre,
Florín y seis de sus secuaces para fusilarlos públicamente en la plaza.

Las tropas cerraron el patíbulo por tres de sus costados, dejándolo
abierto hacia el mar para dar paso libre a las balas perdidas. Los
condenados ocuparon los banquillos ``tan tranquilos como si estuviesen
sentados a la mesa''. Vidaurre pidió que le quitaran la venda de los
ojos, merced que no le fue concedida. Los fusileros hicieron fuego desde
tres pasos de distancia. Un proyectil desviado hirió de muerte a un
caballo.

El Diario de Viaje de T. S. Page es hoy propiedad del abogado Alberto
Ibáñez Page, su bisnieto; y a la gentileza de este amigo debo mi acceso
a la copia mecanografiada del texto original.

Libro casi tan desaprovechado como el de Page

es Poezdka chrez buenossaireskia pampy (Un viaje a través de las pampas
de Buenos Aires), del ruso Platón Alexandrovich Chikhachev, fundador más
tarde de la Sociedad Geográfica Imperial de su país, este sabio recorrió
las Américas influido por Humboldt; y de su relato nada sabía el lector
común hasta que el erudito Patricio Estellé publicó el corto capítulo
referente a Chile en el Boletín de la Academia Chilena de la Historia
(segundo semestre de 1967). Tiene el privilegio de ser el primer
testimonio de un súbdito del Zar en esta parte del mundo.

Leyendo a Platón Alexandrovich se admira la impresión que le produjo
Santiago, adonde llegó a caballo después de recorrer las cien verstas
que la separan de Valparaíso. Los huertos como alfombras verdes que
rodeaban a la capital, la compostura de la gente y la tranquilidad
política diéronle una visión de la obra organizadora que Portales había
dejado en herencia a sus compatriotas. ``Chile debe su prosperidad a su
Gobierno, superior a todos los de Hispanoamérica, a sus leyes, que son
fundamentales, y a su administración, que es la más honesta de todas las
antiguas colonias españolas,'' ``El orden subsiste en Chile con el
consentimiento, la convicción personal y la vocación a la
institucionalidad y por la aversión a la anarquía''.

Chikhachev traía una carta de presentación dirigida ``al Muy
Excelentísimo señor Almirante de la Flota Chilena'', residente en
Santiago si hemos de creerle, y que no puede haber sido otro que don
Manuel Blanco Encalada. Refiriéndose a él con zumba evidente, escribe:
``Como no había ningún velero de cien cañones digno de ostentar la
bandera, este segundo Nelson decidió, con el general beneplácito,
desplegarla sobre la chimenea de su casa''. ¿Por qué este pullazo?
¿Quizás el almirante no fue amable con él, o no accedió a recibirle?

Un tal coronel B. le condujo a la Alameda para mostrarle una ``asamblea
feliz de gente contenta''. Se refiere al clásico paseo de la hora del
fresco; y estos son los sorprendentes pormenores con que lo describe:

''La realmente hermosa Alameda es una graciosa miniatura del Prado de
Madrid, alineada con espléndidos follajes y rodeada de lujuriante flora.
Este paseo es lo sumo de lo elegante. Los jóvenes con ropas de muchos
colores, pintorescas chaquetas, ponchos bordados en plata y sombreros de
copa alta, se mueven por las avenidas con gracia andaluza; sus
movimientos son suaves y graciosos a pesar de sus feas espuelas, algunas
de las cuales son del tamaño del tope de sus sombreros. Pero las
mujeres, ¡las mujeres! No hay mujer en el mundo más encantadora que la
española nacida en cielos tropicales (sic). No son coquetas, pero el
ardor y el encanto de su sangre criolla están en ellas en todo momento:
vivaces ojos negros, tez marfileña, celestiales figuras dibujadas por
mantillas suaves y transparentes\\ldots''

Para proseguir su viaje a Mendoza había contratado Platón Alexandrovich
un arriero cordillerano. Este individuo pasó a recogerle al hotel
llevando una mula de carga, otra de silla y la indispensable madrina
premunida de su campanilla colgada al cuello. Entre la ropa de la maleta
puso el sabio unas cuantas piastras españolas, un compás, dos o tres
libros y algunos termómetros; y al cinto se colocó las pistolas de
chispa que le protegían durante los cuatros días y noches de la travesía
de los Andes. La compadecida propietaria del hotel quiso
desanimarle de emprender esa aventura de riesgos mortales, a la que hoy sólo podría lanzarse un
contrabandista o un fugitivo de la justicia. Le sirvió la última taza de
chocolate y le dijo con lástima: ``Lo siento mucho\ldots, vaya con Dios''

Guiado por Antonio, el arriero, Chikhachev partió al trote de su mula,
``y en diez minutos nos encontrábamos fuera de las puertas de Santiago''

Le pareció el valle del Maipo ``una de las vistas más impresionantes de
América'', aunque en ese entonces se hallaba todavía casi deshabitado y
sin más vegetación que los árboles y pastos naturales.

Los que se preguntan por qué el pueblo de Puente Alto lleva este nombre,
encontrarán la respuesta en el párrafo en que el viajero cuenta cómo
salvó el río. Existía allí ``un puente colgante hecho de cordeles y
cueros'', que describe como sigue:

``\ldots{}muy similar a los puentes europeos de cable, con la diferencia
de que este modelo se ha construido en tiempos inmemoriales y la gente
que lo construyó nunca había estudiado geometría o mecánica de técnica
europea. Nuestras mulas lo cruzaron sin tropiezo, aunque no dejó de
inquietarnos su movimiento. Desde aquí la tierra se empieza a elevar y
en algunos momentos pareciera que se pudiese tocar los Andes con la
mano, aunque nos quedaban cuarenta horas de marcha\ldots''

El arriero, un mocetón despreocupado, iba adelante cantando tonadillas y
fumando su cigarrillo de hoja. Una manera de ser y de conducirse que con
el tiempo habría de modificarse en el chileno popular y de la cual
apenas si quedan vestigios. Después de cinco horas de viaje, Antonio se
detuvo y anunció que era el momento de merendar y dormir una siesta.
Desmontaron entre rocas y cactus y el guía ``esparció sobre el Suelo una
singular comida que con toda su cortesía española me invitó a
compartir\ldots ¡Qué sorprendido estuve cuando pegué el primer mordisco!
Sentí como si la garganta se me quemara; y la colación instantáneamente
saltó fuera. La causa: un ají rojo entre el pan. Mi boca continuaba
ardiendo y Antonio se reía a mandíbula batiente, diciéndome que todo
estaba en regla y que no entendía él extraño paladar de los ingleses''.

Un sueño de tres horas, tendidos sobre las mantas, y reanudaron la
marcha adentrándose en la Cordillera. La mula madrina caminaba a la
cabeza con el tintineo inmutable de su campanilla.

``Rápidamente'', termina el relato, ``comenzamos a remontarnos\ldots
Alrededor nuestro emergían las altas murallas de granito, y de tarde en
tarde, arbustos de un verde amarillento. Peñascos y rocas caían a veces
en repentinas cascadas, obligando a las mulas a detenerse. Todo lo que
nos rodeaba asumía un aspecto majestuoso y solemne\ldots''

Casi en los mismos años de Chikhachev pasó por Santiago William
Ruschenberger, cirujano norteamericano embarcado en el Falmouth y
conocido como autor de Three years in the Pacific: libro imprescindible
en la bibliografía extranjera sobre Chile.

Ruschenberger viajó desde Valparaíso en un birlocho y se apeó en la
Plaza de Armas para hospedarse en la obscura y sucia Fonda Inglesa de
William Milligan. Era éste un escocés abúlico, envuelto en una capa de
piel de león, que pasaba su tiempo sentado ante la mesa de billar del
hotelucho. Por la plaza desierta circulaban coches, carretas y jinetes
levantando polvaredas en torno al céntrico pilón de agua. Encuadraban
este yermo casas de un piso con techo de tejas y ventanas de fierro
forjado. La Catedral seguía inconclusa después de sesenta años de
trabajo esporádico. Cerca del Palacio de Gobierno y la Intendencia
---únicos edificios sobresalientes--- estaba la cárcel, donde cada
mañana se exhibían los cadáveres de gente apuñalada, con un platillo
sobre el pecho, para reunir dinero con que pagar su sepultación.

La ciudad de cuarenta mil habitantes tenía pavimento de piedras de
huevillo y sus aceras estaban cubiertas con losas. Por en medio de la
calle corría la acequia colonial que se llevaba. las basuras y
desperdicios de cocina del vecindario. Pensó Ruschen berger que ésta era
la capital más limpia de América del Sur, aunque no la más dinámica.
``Desde las dos de la tarde hasta la puesta del sol no se ve un alma en
la Plaza; las tiendas cierran sus puertas y todo el mundo duerme la
siesta. Como a eso de las seis vuelve otra vez la animación; se reabren
las tiendas y la Plaza se llena de señoras que hacen sus compras o que
van o vuelven del paseo de la Alameda. Andan solas por la calle, con la
cabeza descubierta, salvo cuando llevan una mantilla o alguna flor de
jardín en el cabello\ldots, y jamás se las ofende con palabras
impertinentes''.

Curioseando en el comercio detallista, el viajero observó que no
existían los negocios especializados: todos vendían de todo, desde
comestibles y miriñaques hasta velas de sebo y libros. Sin embargo,
buscó inútilmente el Quijote en los almacenes, paqueterías y boticas de
la plaza y calles vecinas.

Esta sociedad poco inclinada a la lectura mostraba una decidida
predilección por la música. En cada hogar pudiente había un piano
presidiendo el salón, y Ruschenberger anotó que ``las jóvenes tocan las
composiciones de los mejores maestros alemanes e italianos, como ser:
Mozart, Weber, Rossini y otros, con mucho gusto y buena ejecución''. El
naciente mundo musical santiaguino era fruto de las actividades de la
Sociedad Filarmónica, creada hacia 1826, que semanalmente ofrecía una
función de música vocal y orquestal en que participaban señoras y
caballeros; ``también había baile y conversación, prohibiéndose los
bailes nacionales\ldots''

Aparte las diversiones hogareñas y sociales, el capitalino de entonces
no tenía más entretención que asistir al reñidero de gallos de la
plazuela del Tajamar (actual Plaza Bello), y practicar o presenciar el
juego de pelota española, que gustaba por igual al gañán y al pelucón.
Podría agregarse ---aunque era un deber--- el ejercicio en las milicias
cívicas, que creó, Portales como disciplina de la juventud y barrera de
la anarquía. El propio Ministro mandó. Uno de estos regimientos, y como
sabemos, fue la milicia civil la que decidió la derrota de Vidaurre en
el combate del Barón.

Tal como hoy, el mayor atractivo de Santiago estaba en sus alrededores,
y Ruschenberger no se arrepintió de haber aceptado el convite de ``Don
Vicente'', senador y mayorazgo que poseía en Colina un fundo de cincuenta
millas cuadradas. No sonaba aún la hora de la subdivisión de la tierra,
y es interesante ver cómo juzgo un norteamericano esa anacrónica
institución colonial:

``Estas grandes haciendas han sido un obstáculo para el progreso del
país, porque la ley de España, para mantenerlas intactas, colocaba todos
los bienes raíces en manos de unos pocos individuos, haciendo que se
heredasen de padre a hijo ad infinitum. Por necesaria que sea .la ley de
mayorazgos en países monárquicos para mantener una aristocracia, no
tiene razón de ser en una república, y, por consiguiente, hoy día el
mayorazgo sólo existe en los casos de primogénitos nacidos antes de
anularse dicha ley por el Congreso Nacional''.

Vivían en el fundo de Don Vicente cuatrocientas familias (de dos mil
quinientas a tres mil almas), trabajando ciento veinte yuntas de bueyes
roturando la tierra con arados no más modernos que los del tiempo del
Imperio Romano; así y todo, la hacienda producía veinticinco mil dólares
en trigo.

Algún día se escribirá la novela de aquellos formidables y pintorescos
magnates agrícolas del pasado. Aunque se hallaba a sólo siete leguas de
Santiago, don Vicente asistía al Senado cuatro o cinco veces en el año.
Su régimen de vida no era ni más ni menos que el de sus vecinos, tan
poderosos y despreocupados como él. Se desayunaba a las diez, luego leía
el Quijote en edición de lujo y jugaba al ajedrez con el Cura de Colina.
Almorzaba en mesa de dieciséis asientos, saboreando en servicio de plata
trece guisas y cuatro postres, todo ello regado con vino, chicha y
clarete. Después del almuerzo, que duraba dos horas, dormía la siesta y
descansaba hasta las diez, para volver al comedor a cenar\ldots Este
pobrecillo inapetente le pareció a Ruschenberger ``justo y caritativo
para con los que de él dependían'', añadiendo que le importaba un bledo
el resto de sus semejantes. Ni siquiera leía la correspondencia de sus
conocidos, y lo explicaba diciendo: ``Mucho me alegra saber de su
prosperidad; si son desgraciados, lo siento. ¿Qué más? ¿Para qué
molestarme con sus cartas?'' Y en cuanto a ir a su mansión de Santiago, o
al Senado, decía: ``¿Para qué? Tienen bastante sin mí''

La hospitalidad de los chilenos arrancó al viajero yanqui expresiones
admirativas; pero el concepto que se formó de su carácter ---bastante
ceñido a la verdad--- dista de ser halagador:

``Son inconstantes y sus afectos enteramente superficiales; sus
sentimientos son inestables; se entusiasman con facilidad pero con igual
veleidad vuelven a ponerse indiferentes.''

\chapter{BELLO, REDACTOR DE ``EL ARAUCANO''}

Tarea difícil es captar a pequeña escala al hombre desmesurado que por
espacio de treinta y cinco años iluminó a Chile como un sol intelectual,
dejando entre sus obras el Código Civil, la organización de la
Cancillería, la depuración de la lengua castellana, el Derecho de Gentes
y la fundación de la Universidad. Humanista, filósofo, jurisconsulto,
cosmógrafo, poeta, profesor, senador, consejero de Estado, escritor,
crítico literario y periodista, este ser polifacético ---y brillante en
todo lo que hacía--- no se deja estudiar sino por partes, como
estudiaríamos un archipiélago o un zodíaco, o tal vez a Goethe.

La herencia invaluable que de él recibimos no guarda relación verosímil
con la debilidad de su envoltura material. Era su salud tan frágil que
el sabio Humboldt, en Caracas, le creyó condenado la tisis y recomendó a
sus padres interrumpir sus estudios universitarios. No fue pues abogado
diplomado el autor de nuestro Código Civil y coautor de la Constitución
de 1833, como no fue profesor Sarmiento, fundador y director de la
Escuela Normal de Preceptores, ni lo fue tampoco Barros Arana, rector y
maestro de cuatro asignaturas en el Instituto Nacional\ldots Pero retirar a
Bello de la Universidad no implicaba alejarlo del estudio: aprendió el
francés solo, en su casa, y llegó a dominarlo en tal forma que vino a
convertirse en traductor acabado de Molière y La Fontaine, como después
lo sería de Racine y Víctor Hugo. Aspirante a un empleo en la secretaría
del Gobierno venezolano, obtuvo el nombramiento presentando la pieza
mejor redactada entre decenas de concursantes. Para desempeñarse en este
puesto hubo de estudiar el inglés, sin más ayuda que un diccionario y
una gramática, tal como hiciera con el francés; y fue su dominio de
estas lenguas lo que le llevó a ocupar el cargo de secretario de la
Legación de Colombia en Londres, y posteriormente el puesto similar en
la de Chile, en vista de la mísera esporádica paga que percibía en
aquélla. Allí apreció sus dotes el Ministro Plenipotenciario Mariano
Egaña, que andando el tiempo lo recomendó al Gobierno de Santiago como
persona ``de educación escogida y clásica, profundos conocimientos en
literatura, posesión completa de las lenguas principales, antiguas y
modernas, práctico en la diplomacia y un buen carácter a que da bastante
realce la modestia''. Cuando Bolívar supo que se venía ---contratado por
el Presidente Francisco Antonio Pinto con dos mil pesos anuales---
intentó hacerlo desistir escribiendo desde Quito a su Ministro en
Londres:

``\ldots yo ruego Vd. encarecidamente que no deje perderse a ese ilustrado
amigo en el país de la anarquía'' (Chile). ``Persuada Vd. a Bello de que
lo menos malo que tiene América es Colombia\ldots Su patria debe ser
preferida a todas, y él, digno de ocupar un puesto muy importante en
ella. Yo conozco la superioridad de este caraqueño contemporáneo mío.
Fue mi maestro cuando teníamos la misma edad (sic) y yo le amaba con
respeto. Su esquivez nos ha tenido separados\ldots y por lo mismo deseo
reconciliarme, es decir, ganarlo para Colombia''.

Tardío intento de recobrar a un hombre perdido para siempre por los
desdenes y lá tacañería de sus compatriotas.

Llegó a Valparaíso en el invierno de 1829, desembarcando del velero
Greclan con su esposa británica Elizabeth Dunn y sus hijos, sin
secretario ni sirvientes y con un equipaje de emigrante pobre y muchos
baúles y cajones repletos de libros y manuscritos. Entre éstos, el de la
\emph{Silva a la agricultura de la zona tórrida}, poema compuesto en la
neblina londinense y dictado por la nostalgia del terruño que no
volvería a ver.

Venía a prestar servicios como Oficial Mayor (Subsecretario) del
Ministerio de Relaciones Exteriores, cargo que ocuparía por cuatro
lustros consecutivos; pero sus primeras contribuciones al país de
adopción fueron como profesor de legislación y de literatura española en
el Colegio de Santiago y como redactor de El Araucano, el periódico
fundado por el Ministro Portales y al que ingresó como responsable de
las secciones jurídica, literaria y científica.

Sus módicos sueldos no le. permitieron el lujo de alquilar una casa; se
instaló como pensionista de doña Eulogia Nieto de Lafinur, residente en
la calle Santo Domingo, costado sur, casi esquina de Miraflores. Allí
vivió por muchos años, y la cortedad de recursos de la familia, refiere
don Paulino Alfonso, obligaba a Elizabeth Dunn de Bello a lavar con sus
manos la ropa de su marido y de sus niños\ldots Don Andrés Bello y López
bordeaba ya la cincuentena, y al decir de su nieta Ana Luisa Prats, ``su
fisonomía era extremadamente armónica, suave y atrayente. Su cabeza,
perfectamente conformada, su amplia y hermosa frente, la profunda e
insinuante mirada de sus grandes ojos oscuros, todo el conjunto de su
rostro respiraba bondad e inteligencia''. Así lo pintó Monvoisin, y en
los Recuerdos Literarios de Lastarria se le ve ``serio e impasible,
hablando parcamente y fumando un enorme habano''.

El Araucano, insigne creación de Portales y del tipógrafo y relojero
Gandarillas, había advertido en su prospecto que haría una crítica veraz
y severa a las medidas administrativas que estimase desacertadas o
reñidas con la justicia. Esta declaración sin precedentes anunciaba el
nacimiento de una nueva ética en el periodismo oficialista, algo jamás
concebido hasta entonces en la América española, y ha debido fascinar al
ecuánime y apolítico Bello, que bebiera en Londres la mesura y
objetividad ejemplares de la prensa inglesa. El país de la anarquía,
como Bolívar llamó a Chile, de la noche a la mañana había entrado en
vereda y daba al resto del continente la asombrosa lección de un
Gobierno que editaba un periódico imparcial, capaz de censurar sus
propios errores; un órgano cuyo objetivo declarado era ``agradar e
instruir a los verdaderos amantes de la ilustración, sin fomentar
rencores ni dar pábulo a esas pasiones lastimosas que se alimentan con
las discordias, con las animosidades, con la burla del hombre y con la
ofensa del ciudadano''. A esta magna finalidad iba a contribuir el
redactor Bello con el aporte impagable de su exhaustiva cultura
universal, que le permitiría introducir el mundo en el periodismo
chileno, circunscrito hasta entonces al estrecho marco de los hechos
locales. De sus estantes atiborrados de libros y revistas empezó a
extraer el vasto material de asuntos históricos y científicos, de poesía
clásica, viajes y vidas famosas que fue traduciendo o adaptando para
proporcionar a sus lectores el alimento espiritual que nadie había
entregado antes en el ignaro Chile. Pacientes eruditos han rastreado en
la colección de El Araucano hasta completar la lista de cuanto publicó
en sus páginas en más de veinte años de trabajo permanente. Entreverados
con sus colaboraciones culturales están los incontables artículos que
dedicó al examen de los problemas de Estado, a las cuestiones
internacionales, a la educación, la justicia, la química aplicada a la
agricultura, la internación de libros (cuya censura combatió, ganándose
el timbre de hereje), la conveniencia de reanudar relaciones con España,
la vacuna, los hospitales, el bandidaje en los campos\ldots Nada escapaba a
su curiosidad y sólo Vicuña Mackenna podría comparársele más tarde por
la variedad sin límites de su producción periodística. Uno de sus temas
favoritos fue necesariamente la defensa de la pureza del idioma,
horrorizado como vivía de oír decir hasta en los salones haiga por haya,
flaire por fraile, sandiya por sandía, dentrar por entrar, yo tueso por
yo toso, curto por culto y celebro por cerebro. El autor de la célebre
Gramática no daba tregua en su lucha contra la incorrección, mala
pronunciación y peor construcción del habla criolla. Por eso compuso,
entre innúmeros trabajos similares, el manifiesto en serie
\emph{Advertencias sobre el uso de la lengua castellana}, dirigidas a
los padres de familia, profesores de los colegios y maestros de escuela.

Este renovador del periodismo nacional laboraba sobreponiéndose a
continuos y fuertes dolores de cabeza, sentado en la ``dura silla de
trabajo'' descrita por Vicuña Mackenna y emitiendo las ideas al correr de
su pluma de ganso. Refiere Ana Luisa Prats: ``Agotábasele a veces el
papel y entonces echaba mano al margen de los diarios, que aparecían
llenos de esa escritura minúscula, casi indescifrable, que él mismo a
veces no podía entender\\ldots''

Con esta faena abrumadora sobre los hombros, más las pesadas funciones
ministeriales y docentes, encontraba tiempo todavía para cumplir como
padre solícito. Cada mañana acompañaba a misa a su prole ---porque era
un creyente de fe profunda--- y sólo después- de dejarla de vuelta en el
hogar tomaba el camino de la Subsecretaría de Relaciones. Solía pasar
los días de descanso en Peñalolén, en la casa de campo de don Mariano
Egaña, en cuyo parque señorial escribió su poema imperecedero: \emph{La
oración por todos}. Frecuentaba también el fundo de doña Javiera Carrera
en El Monte, adonde iba atraído por sus jardines maravillosos y donde
leyó por primera vez ante una concurrencia los versos concebidos en
Peñalolén, que ideó como una imitación de Víctor Hugo y terminaron
superando al autor original.

Sin dejar la redacción de El Araucano ---que incluso dirigió por un
tiempo a raíz de la salida de Gandarillas--- el infatigable trabajador
púsose a colaborar con Egaña en el estudio de la nueva Constitución
Política, que Prieto y Portales iban a dar a la República. El año
histórico en que ella entró en vigencia, Bello fue elegido senador.
Convertido ya en el hombre más atareado y consultado que podía darse,
convertido en un pilar de la conducción interior y exterior del país,
prosiguió imperturbable su labor de traductor y articulista de El
Araucano. A quienes vayan creyendo que era un vocero sumiso del
pensamiento del Gobierno, conviene recordarles que desde esas columnas
oficialistas ``hizo la guerra'' (palabras del Presidente Prieto) a ciertos
capítulos de la Carta Constitucional que no eran de su agrado. Con igual
independencia había censurado al propio Portales, cuando ocupaba la
gobernación de Valparaíso, por haber condenado a muerte a un ballenero
americano que perdió la razón y recorrió las calles del puerto hiriendo
y matando gente a puñaladas. Tragedia que dio pie al abogado sin título
para emprenderlas de nuevo contra el Poder Judicial (M. L. Amunátegui le
atribuye cincuenta publicaciones sobre esta materia) señalando la
arbitrariedad y el despotismo de los jueces, insistiendo en la
conveniencia de dar publicidad a los juicios y sentencias y sugiriendo
la adopción de medidas que impidiesen a los magistrados ``inspirarse en
complacencias o en influjos funestos\\ldots''

Fiel a su norma de escritor multifacético, interesado en todo y
entendido en todo, combinaba sus campañas de bien público ---trabajando
ahora día y noche--- con las crónicas de ciencia y literatura que
ofrecía para solaz de la pequeña élite intelectual santiaguina. La pluma
que ayer analizara La Araucana o el Poema del Cid hoy comentaba la
expedición de Wilkes al Pacífico y mañana disertaba sobre el sol y las
estrellas y más tarde se ocupaba del aerolito caído en los alrededores
de la capital. Escribía sin imposición de temas ni de criterios, que
para eso estaba el admirable periódico gobiernista. Ya vimos al autónomo
redactor atacar sin miedo al omnipotente Portales, esto es, a su
fundador, escapando ileso de meter la cabeza entre las fauces del tigre.
No sucedió nada, y ésta fue la demostración concluyente del equilibrio y
solidez alcanzados por el más ilustrado régimen que hubiera conducido al
país.

Bello, por lo demás, sentíase a sus anchas a la sombra de un gobierno
autoritario, pero a la vez justiciero y creador, ideal que dejara
estampado en cartas escritas mucho antes de venirse a Chile. Cierto que
como jurista y consejero de Estado fue contrario a la decisión
portaliana de apoderarse de las fuerzas navales perú-bolivianas (lo que
hizo caer sobre él la mirada fulmínea del ``Ministro Salteador''); pero
como periodista y como chileno de corazón celebró con júbilo exultante
el desenlace de esa guerra preventiva que Portales impuso y ganó desde
el sepulcro. Se preguntó si hubo alguna emprendida por motivos más
justos, más grandes y más generosos; y he aquí su análisis de la batalla
de Yungay:

``Todos los rasgos que vemos esparcidos en la multitud de acciones
gloriosas de que nuestro hemisferio ha sido testigo, y algunos más, los
hallamos reunidos en la del 20 de enero: inferioridad numérica de los
vencedores, que no alcanzaban a igualar los dos tercios de la fuerza
enemiga; cuerpos compuestos de hombres que hacían en la campaña el
aprendizaje, no sólo de la guerra, sino del servicio militar; una
oficialidad reclutada entre jóvenes imberbes, cuya tierna edad había
dado materia a las chocarrerías del populacho de Lima; un suelo extraño;
un clima insalubre; escasez de todo género de recursos; desnudez y
hambre. Por parte de los contrarios, abundancia de todo; posiciones
ventajosísimas; terreno escabroso; cerros de difícil acceso; trincheras
y fortificaciones\ldots; batallones aguerridos, bien disciplinados y
equipados; un jefe favorecido hasta allí por la fortuna y de cuya
vigilancia y actividad han dado muestras relevantes los diarios y
registros militares que forman parte de nuestro botín\ldots En la acción
misma, nada debido al acaso, nada a la sorpresa, nada a la infidencia,
nada a la cobardía; las posiciones defendidas a todo trance; cada palmo
de tierra disputado con tenacidad y furor; y por resultados inmediatos,
la destrucción completa de ese enemigo, la caída de un imperio; la
emancipación de dos naciones.''

Por ese tiempo no existía en su medio un escritor que pudiese
equiparársele, y su sapiencia había llegado a ser a tal punto
considerada que los altos poderes no podían ya pasarse sin él. Un hecho
-lo prueba como nada: en 1839, precisamente el año de Yungay, el
Presidente Prieto le encargó la redacción de su Mensaje al Congreso
Pleno\ldots y el Senado le confió la del discurso de respuesta.

Era la eminencia gris de que habla Joaquín Edwards Bello, aquel que
``desde la quietud de su silla dirige la república''; el hombre medular e
insubstituible llamado por el destino a dar contenido espiritual a una
construcción política que iba a perdurar hasta el ocaso del siglo y a
constituirse en ejemplo de la posteridad.

\chapter{BELLO, RECTOR DE LA UNIVERSIDAD}

En sus Memorias íntimas dice Pedro Félix Vicuña refiriéndose al general
Bulnes: ``Había sido un guerrillero y sus jefes me han dicho que jamás
supo mandar un escuadrón de caballería\ldots Jefe del ejército al (sic)
Perú y pretendiente a la Presidencia, todo también era obra de su
parentesco con Prieto, porque era ignorante en todo el sentido de la
palabra, sin más que la suspicacia y pillería de un araucano, tribus con
quienes había pasado su juventud en íntimas relaciones''.

Así juzgaban la pasión política y la chatez lugareña al estratego que
nos dio la victoria en la guerra contra el imperialista Santa Cruz,
abrochada con la fantástica batalla de Yungay; así juzgaban al
gobernante visionario que salvó la soberanía de Magallanes ocupándolo
veinticuatro horas antes del arribo de los franceses; al protector de la
enseñanza que fundó la primera Escuela Normal de Preceptores de América
del Sur y con ayuda de Bello creó la Universidad de Chile.

Continuador de Prieto y Portales e introductor de Montt, don Manuel
Bulnes fue el hombre clave en el afianzamiento de aquel régimen
incomparable, que él preservó al precio de una guerra civil y de una
batalla campal en las calles de Santiago.

La más importante de sus iniciativas culturales encontró a don Andrés
Bello con sesenta y un años a la espalda, todavía en la flor de su
intelecto luminoso, pero ya herido por los crueles golpes con que el
destino se esmeró en abrumarle. Habían muerto en plena juventud, algunos
en edad infantil, la mayor parte de sus hijos, y el padre inconsolable e
insomne vagaba de noche por los corredores de su casa, ``penando en vida''
al decir de sus sirvientes, llorando por los retoños perdidos y rezando
los salmos de David. No tenía la frialdad emocional de su esposa
---enigmática mujer que no se sabe si era inglesa o irlandesa y de la
cual se dice que había ejercido en Londres el oficio de institutriz--- y
para aliviarse sus pesares obsesivos sólo conocía el sabio la medicina
del trabajo. Y este antídoto lo tomaba en dosis que ningún facultativo
sensato ha podido recetarle, porque a las ocupaciones agotadoras que le
conocemos, redactor de tres secciones en El Araucano, subsecretario de
Relaciones Exteriores, senador y consejero de Estado, había añadido unas
clases privadas de gramática y literatura y otras de derecho romano que
dictaba en la biblioteca de su domicilio. Clases sui generis a las que
asistía José Victorino Lastarria, en cuyos Recuerdos Literarios se lee:
``El estudio de la lengua era un curso completo de filología, que
comprendía desde la gramática general a la historia del
castellano\ldots allí seguía el profesor su antigua costumbre de escribir
sus textos a medida que los enseñaba\ldots Nunca explicaba, sólo
conversaba\ldots casi siempre fumando un enorme habano, hablando
parcamente, con pausa y sin mover un músculo de sus facciones\ldots''

Había llegado a ser una institución nacional, y era propio de la
pequeñez del medio que hubiese envidiosos empeñados en negar su talento,
afirmando que sólo poseía paciencia y buena memoria. No tardarían sus
grandes méritos en rebasar el marco americano para alcanzar resonancia
en el Viejo Mundo, donde Schiller le llamó en su Gedanken der
anterrkanische latinien: ``el padre de la pedagogía en América''; donde
Menéndez Pelayo lo situó entre los mejores traductores clásicos del
siglo, y un autor inglés vio en él a ``un self-made man que ha explorado
casi todos los campos del conocimiento humano''.

Tal era el hombre que el ignorante guerrillero Bulnes comisionó en 1842
para organizar con el Ministro Manuel Montt la Universidad de Chile,
foco del saber y la investigación que iba a transformar la vida
espiritual de un país colocado a sangre y fuego en la senda de la paz
creadora.

Cuenta Lastarria que la fundación del establecimiento fue para el
maestro ``un motivo de regocijo, que le infundió un verdadero
entusiasmo'', y refiere que les decía con pasión a sus futuros
colaboradores: ``Probemos ahora que hay hombres de estudio, para quienes
no son ingratas las ciencias; y aunque tengamos una Academia en lugar de
un cuerpo docente, desde ella impulsaremos la enseñanza y elevaremos la
instrucción al nivel que le corresponde''.

Nombrado rector por derecho propio e indiscutible, don Andrés se reservó
para sí la presidencia de la Facultad de Filosofía y Humanidades. El
muchacho de contextura enfermiza que Humboldt creyó incapacitado para la
vida activa parecía poseer ahora, a las puertas de la ancianidad, alguna
fórmula secreta que le permitía cumplir el trabajo de cuatro hombres
normales sin agotarse la mente ni destruirse la salud.

Instalóse la ``Casa de Bello'', como se la llamaría con el tiempo, en el
edificio que dejara vacío la fenecida Universidad de San Felipe, sobre
el terreno hoy ocupado por el Teatro Municipal. Allí funcionaría por
espacio de veinte años, hasta el día de su traslado a la sede ad hoc de
la vieja Alameda.

Cabe preguntarse si sus fundadores tuvieron una idea de las proporciones
que iba a tomar esa entidad que en nuestros días ocupa seiscientos
edificios a lo largo del territorio, emplea a dieciséis mil docentes y
funcionarios y da enseñanza a sesenta mil estudiantes. Con estos números
en la mente resulta conmovedor representarnos lo que fue la primera
Universidad chilena, creada con un presupuesto de catorce mil pesos
anuales\ldots De hecho, era una transformación de su antecesora colonial
---donde estudiaron Lacunza y Molina--- entidad ``extinguida'' por decreto
del Ministro Egaña en 1839 y resucitada y adaptada a la era republicana
por ley de 19 de noviembre de 1842, que la definió como ``una casa de
estudios generales'', esto es, no docente, pues la enseñanza superior
seguiría impartiéndose en el Instituto Nacional. En el fondo sería una
Superintendencia de Educación Pública, cuya creación había quedado
señalada en la Constitución de 1833, donde Egaña, Bello y Portales
fijaron el postulado de que la instrucción debía ser ``una atención
preferente del Gobierno''.

Cinco Facultades compondrán la Universidad, y es menester fijarse en los
nombres de los decanos y subdecanos que Bello, Montt y Bulnes eligieron
entre la crema y nata de la intelectualidad de la época: Filosofía y
Humanidades, Miguel de la Barra y Antonio García Reyes; Ciencias
Matemáticas y Físicas, Andrés Antonio Gorbea e Ignacio Domeyko;
Medicina, Lorenzo Sazié y Francisco Javier Tocornal; Leyes y Ciencias
Políticas, Mariano Egaña y Miguel María Güemes; y Teología, presbíteros
Rafael Valentín Valdivieso y Justo Donoso. Secretario general fue
designado el poeta Salvador Sanfuentes y patrono el Presidente de la
República\ldots Como si dijéramos, catorce futuras calles y plazas de
Santiago prestaron su nombre para decorar la lista de las cabezas
organizadoras y ejecutivas de la corporación. Su Ley Orgánica decía que
``corresponde a este cuerpo la dirección de los establecimientos
literarios y científicos y la inspección de todos los demás
establecimientos de educación\\ldots'', ``\ldots{}tendrá comisiones
examinadoras para todos los colegios y liceos''; y sus supremos
objetivos serían propender a la expansión cultural y al nacionalismo y
``a la grande obra de crear una literatura y una ciencia de relativa
originalidad''.

Fue inaugurada la Universidad de Chile el 17 de septiembre del año 43 en
ceremonia de sin par esplendor. Salió el Presidente Bulnes del palacio
de la Plaza de Armas encabezando un desfile de tres cuadras de largo y
caminando los participantes de dos en dos y en orden de jerarquía:
detrás del general iba su Estado Mayor universitario, compuesto por
Montt, Bello y Egaña; en seguida el resto del gabinete y los
parlamentarios, jefes militares y dignatarios de la Iglesia e invitados
de honor. Los ``individuos'' de las cinco Facultades vestían el uniforme
de gala ideado por Egaña: sombrero negro de puntas cori cucarda
tricolor, casaca de paño azul con botones dorados, chaleco y pantalón
corto de color gris, medias blancas de seda, zapatos con hebillas de
plata y espadín al cinto. La tenida del rector Bello, del poeta
Sanfuentes y los decanos era de pantalón largo, llevando la casaca su
cuello y bocamangas orlados con hojas de palma y olivo bordadas en seda
verde, símbolos de triunfo y paz, y el sombrero rematado por plumas
negras en lugar de escarapela. Compacto gentío, una banda de música y
una batería de cañones esperaban a los desfilantes en la plazoleta de la
Universidad. Abierto el acto por boca del secretario general, pronunció
el Ministro Montt un discurso cuyo texto no se conserva, y en seguida
tomó juramento a los decanos y a los académicos, nombrándolos uno por
uno; tras lo cual el Presidente de la República les hizo entrega de las
medallas de oro y plata con cinta de color que se pusieron al cuello.
Seguramente han estado presentes en la sala el almirante Blanco
Encalada, don Antonio Varas, don Manuel Rengifo y doña Javiera Carrera,
y probablemente otras lumbreras de esos días, como los argentinos Mitre
y Sarmiento, la poetisa Mercedes Marín del Solar, el pintor Monvoisin y
doña Isidora Zegers. Llegado su turno, el rector Bello se puso de pie
para decir en su histórica pieza oratoria:

``\ldots{}soy de los que miran la instrucción general, la educación del
pueblo, como uno de los objetos más importantes y privilegiados a que
pueda dirigir su atención el Gobierno; como una necesidad primera y
urgente; como la base de todo sólido progreso; como el cimiento
indispensable de las instituciones republicanas. Pero por lo mismo, creo
necesario y urgente el fomento de la enseñanza literaria y científica.
En ninguna parte ha podido generalizarse la instrucción elemental que
reclaman las clases laboriosas\ldots sino donde han florecido de antemano
las ciencias y las letras''. ``La instrucción literaria y científica es la
fuente de donde la instrucción elemental se nutre y se vivifica, a la
manera que en una sociedad bien organizada la riqueza de la clase más
favorecida de la fortuna es el manantial de donde se deriva la
subsistencia de las clases trabajadoras, el bienestar del pueblo''. ``El
fomento, sobre todo, de la instrucción religiosa y moral del pueblo es
un deber que cada miembro de la Universidad se impone por el hecho de
ser recibido en su seno''. ``La utilidad práctica, los resultados
positivos, las mejoras sociales, es lo que principalmente espera de la
Universidad el Gobierno\ldots''

Terminado el acto, la concurrencia dirigióse a la Catedral, ensordecida
por las salvas de artillería, para asistir al Te Deum, adecuado broche
del acontecimiento y de ahí a palacio, donde el Presidente recibió los
parabienes de estilo a la hora en que se encendían los faroles de vela.

Con este aparato digno de la Colonia quedó instalada la casa
universitaria, a la que ingresó un contingente de antiguos individuos de
la de San Felipe. Aun con esto, la ``aduana del saber'', como la llamaría
Pérez Rosales, sólo pudo reunir ochenta y cinco de los ciento cincuenta
académicos que debían integrar las Facultades. Señal concluyente del
atraso científico y literario del medio, y prueba indubitable, a la vez,
de la vital misión que venía a cumplir la Universidad.

Tal como Bello lo tenía previsto, su nombramiento por cinco años, que
sería renovado hasta el fin de su vida, fue discutido en corrillos y
tertulias\ldots porque era extranjero y porque se había opuesto a la
censura literaria. Sus brillantes y patrióticos servicios no contaban, y
así le viesen a diario en la misa de 8, permanecía indeleble la marca de
impío, y su esposa acostumbróse a oír a su paso por la calle: ``Ahí va la
gringa del hereje'' y cosas de este jaez. Pequeñeces de pueblo chico que
antaño arrancaron chispas y denuestos a la pluma de Portales pero que no
rozaban la epidermis del inmutable Bello, absorbido por la atención de
su círculo de amigos y discípulos, por el cultivo de las dalias en
Peñalolén y por la labor inconcebible que soportaba sobre sus hombros
titánicos. Fue en ese entonces que el educador máximo de América
emprendió el aprendizaje del alemán, para iniciar la documentación
preparatoria del Código Civil, idioma que llegaría a dominar en seis
meses sin ayuda de profesor.

De sus múltiples funciones públicas, fue la de rector la que había de
reportarle la cuota mayor de problemas y sinsabores. A los pocos meses
de estar sirviendo este cargo, el alumno de Leyes Francisco Bilbao
publicó en el diario ``El Crepúsculo'' su libelo titulado
\emph{Sociabilidad chilena}, virulento ataque a la Iglesia y a la
estructura política y social heredada del coloniaje. Acusado de
``blasfemo e inmoral'' por un jurado de imprenta, Bilbao (discípulo de
Lastarria) fue condenado a prisión o multa mientras el diarito era
quemado en una fogata en la Plaza de Armas. El fanatismo de la curia y
la ingenuidad oficialista convirtieron con esto a un sujeto
inconsistente y más bien risible en héroe popular, que fue paseado en
andas y cuya multa fue pagada por sus amigos, algunos de los cuales eran
profesores de la enseñanza superior. Reunido el Consejo de la
Universidad para tratar su caso, dictaminó Bello que ``el autor de un
escrito que ha sido condenado en tercer grado por inmoral y blasfemo no
debe permanecer en una casa pública de educación''; y Bilbao fue
expulsado del Instituto, de su empleo fiscal y de la pensión en que
vivía, y abandonó el país envuelto en la capa romántica de un perseguido
luchador de la justicia social. Así lanzaron a la fama al futuro
revolucionario comunista, que daría su nombre a una de las principales
avenidas de Santiago\ldots No se habían apagado los ecos de aquella
pelotera cuando Lastarria (discípulo de Bello y profesor de Derecho
Público) leyó en sesión, solemne su politizada monografía sobre la
influencia ``funesta'' que la Conquista y la Colonia legaron a la
República; trabajo considerado como un refuerzo del de Bilbao y que de
acuerdo con el reglamento tuvo que insertarse en los Anales de la
Universidad. Consecuencia de lo cual fue el deterioro de la amistad
existente entre el rector y el insigne pisacallos que era Lastarria.

Dos veces se libró la Universidad de caer en fatal colapso, cuando los
diputados conservadores (1845) y los liberales (1849) pidieron la
supresión de su presupuesto, que estimaban ``inútil e injustificado''. No
pudiendo consumar esta barbaridad, trataron de reducir sus gastos y por
20 votos: contra 17 declararon ad honorem al personal ejecutivo,
exceptuando el secretario general, que retendría su sueldo de cincuenta
pesos. A todo lo cual, para honra de la nación, se opuso el Senado,
donde no en balde estaba Bello influyendo con su oratoria y su prestigio
inmenso.

Como todo constructor ocupa en Chile una parte de su tiempo en parar los
golpes de las fuerzas negativas, el senador Bello no sólo tuvo que
salvar la supervivencia de la Universidad sino también la introducción
del ferrocarril. A este poderoso agente de progreso oponíase su colega
José Miguel Irarrázaval, alegando que el proyecto de Wheelwright iba a
asestar un golpe de muerte a las empresas de carretas y birlochos. Ajeno
al problema de la cesantía de mulas y bueyes, don Andrés rebatió al
retrógrado en memorable sesión de 1847 e influyó directamente en el
ánimo de Bulnes para no retardar la construcción del camino de hierro de
Caldera a Copiapó.

Hacia 1850 bordeaba los setenta años y seguía sin dar señales de fatiga.
A esa década deben referirse los recuerdos que de él dejó don Paulino
Alfonso, en los que describe su sala de trabajo en la casa que había
adquirido en el número 100 de la calle Catedral; ``aposento rodeado de
estantes sin puertas'' donde escribía en la silenciosa compañía de ``un
hermoso gato romano, entre blanco y plomo'', que era tolerado sobre el
escritorio, comía con su amo y acostumbraba a dormir a sus pies sobre
una piel que había bajo el sillón y la mesa. Tal cual aparece en el
dibujo de Luis F. Rojas, que captó el detalle anecdótico desdeñado por
Monvoisin en sus retratos al óleo. Allí redactó Bello la Gramática de la
lengua castellana, el Compendio de la Historia Literaria y la
Descripción del Universo y estudió y ordenó el monumental Código Civil,
que entregaría en 1856. De allí salieron también las cartas impregnadas
de nostalgia que el anciano trasplantado enviaba al terruño inolvidable:
(a su hermano): ``\ldots recuerdo los ríos, las quebradas y hasta los árboles
que solía ver en aquella época feliz de mi vida. Cuantas veces fijo la
vista en el plano de Caracas que me remitiste, creo pasearme otra vez
por sus calles buscando en ellas los edificios conocidos y
preguntándoles por los amigos, los compañeros que ya no existen. ¿Hay
todavía quien se acuerde de mí\ldots? ¡Daría la mitad de lo que me resta de
vida por abrazaros, por ver de nuevo el Catuche, el Guaire, por
arrodillarme sobre las losas que cubren los restos de tantas personas
queridas!'' (A una sobrina): ``Dile a mi madre que no soy capaz de
olvidarla; que no hay mañana ni noche que no la recuerde; que su nombre
es una de las primeras palabras que pronuncio al despertar y una de las
últimas que salen de mis labios al acostarme. bendiciéndola
tiernamente\ldots''

No volvería a verla; y cierta noche, ya muy viejo, despertó sobresaltado
y con el presentimiento angustioso de que algo irreparable había
sucedido, a la hora exacta en que ella expiraba en el lejano hogar.

El desterrado voluntario no desaparecería a su vez sin ver su Obra más
amada, la Universidad, establecida en la casa que él pidió construir
expresamente para sede suya. Cumplido este anhelo, debió sentir que
había llegado el momento epilogal de su existencia. A los ochenta y
cinco años era casi una leyenda viviente este fabuloso trabajador que
sólo abandonó la rectoría y soltó su pluma de escritor -el día en que la
mortal enfermedad lo derribó. Una junta que fue como la reunión del
Cuerpo Médico de Santiago no pudo prolongar la vida de ``una naturaleza
que se agota'', como dijera el doctor Blest discutiendo con su colega
Sazié.

Entregó el ánima el 15 de octubre de 1865. Durante la agonía el gato
romano permaneció echado sobre el lecho, y refiere un testigo que al
percatarse de su fin ``le olfateaba y lamía lanzando maullidos
lastimeros''. Siguió al cortejo multitudinario hasta la Catedral, y a
partir de entonces se negó a probar alimento y murió en algún rincón
oculto de la casa de su amigo.

\chapter{GAY}

Es misterioso el hecho de que los hombres que harían el aporte
intelectual y científico de la restauración portaliana, Gay y Bello,
hayan arribado a Chile en vísperas de producirse ese fenómeno de nuestra
historia política. Como actores escogidos por la suerte para representar
en un drama en gestación, pero todavía ni anunciado, ambos circulaban
entre -bastidores sin sospechar qué papeles irían a confiarles, y así
fue que al levantarse el telón sólo tuvieron que dar un paso para entrar
en escena\ldots Más raro todavía: llegaron casi juntos, en 1829: Bello
contratado como oficial mayor del Ministerio de Relaciones Exteriores y
Gay como profesor de Ciencias de un colegio particular. Dieciocho años
menor que el venezolano, este francés de origen campesino tenía
veintinueve al poner el pie en su ``futura patria adoptiva. Era un
competente zoólogo y botánico y, como Darwin, pronto había de
convertirse en un naturalista completo. En la primera carta despachada a
París cuenta que ya tiene reunidas cerca de dos mil especies chilenas de
botánica y zoología ---muchas de las cuales iba a donar para la
formación del Museo Nacional de Historia Natural--- y en otra misiva
dice que a pesar de las convulsiones intestinas del país se ha ocupado
además de la geología de los alrededores de Santiago. Traía un
inmejorable equipo de trabajo: ``mi colección de instrumentos'', le
escribe al director del Jardín Botánico de Ginebra, ``es una de las más
ricas y más perfectas\ldots; aquellos relativos al magnetismo terrestre\ldots
son la admiración de todos los físicos''.

No tardó en hacerse notar la presencia de este profesor de facha algo
excéntrica, que en días de descanso incursionaba con su mula cargada de
aparatos desconocidos, herramientas de minero, herbarios y bolsas
rellenas de piedras. Apagado el estruendo de Lircay, fijóse en su
persona el ojo del Ministro Portales, infalible en la búsqueda de
quienes formarían el elenco dirigente del nuevo Chile. Es asombroso que
un Gobierno que carecía de fondos para pagar los sueldos públicos haya
contratado la obra cultural de mayor envergadura emprendida hasta hoy
entre nosotros. De ese contrato iba a salir como fruto imperecedero la
Historia Física y Política de Chile, monumento de veintiocho volúmenes y
dos corpulentos Atlas anexos con grabados en colores, que costaría al
Estado cien mil pesos fuertes y tomaría a su autor cuarenta años de
exclusiva dedicación.

No se conocen comunicaciones entre Portales y Gay, cosa que deja un
lamentable vacío documental; pero hay una carta de aquél, fechada en
Valparaíso, en enero de 1832, en donde le describe a Garfias los
característicos afanes del sabio. ``M. Gay\ldots en el tiempo que está aquí
ha gastado más de \$150 en pagar a peso cada objeto nuevo -que le han
presentado. Con esto ha puesto en alarma a todos los muchachos, que
trasnochan buscando pescaditos, conchas, pájaros, cucarachas, mariposas
y demonios, o salen a expedicionar hasta San Antonio por el sur y hasta
Quintero por el norte. El dueño de la posada donde reside ya está loco,
porque todo el día hay en ella un cardumen de hombres que andan en busca
de M. Gay: siempre que sale a la calle\ldots le andan gritando y
mostrándole alguna cosa: ``Señor, esto es nuevo, nunca visto; Vd. no lo
conoce\ldots''

Visto que tenía aquí para largo, viajó a Francia para formalizar su
matrimonio con mademoiselle Sognier, la prometida que dejara en Burdeos
al partir para América; ``persona de carácter muy tierno'', ``verdadera
amiga'' que decía estar ``encantada'' de colaborar en su obra. En tan
prometedora compañía volvió Gay a Valparaíso en la primavera del 34; y a
poco salió para Valdivia llevando una carta del Presidente Prieto en que
ordenaba al Intendente facilitarle personal militar y un intérprete
araucano y proporcionarle una embarcación para movilizarse por ríos y
lagos. Este sólo trabajo preliminar de exploración científica e
investigación histórica iba a demandarle de ocho a nueve años,
obligándole a viajar constantemente a lo largo del territorio y desde la
costa hasta la cordillera. De estas expediciones daba cuenta en los
artículos que remitía a El Araucano, material que aun hoy podría
reunirse en un libro de indeclinable interés.

En este lapso tuvieron lugar los acontecimientos que más fuertemente
afectaron su ánimo y su suerte: nació en Valdivia su hija única,
Thérése; sucumbió en el Barón su protector Portales, y comenzó la gran
desilusión de su vida conyugal. Supo demasiado tarde (única manera de
saberlo) cuán inmensa distancia puede mediar entre la novia y la esposa;
vio cómo la criatura en que pusiera sus esperanzas se convertía como por
obra de maleficio en un ser violento, agresivo y por añadidura
deshonesto, acicateado por la madre y el padre y los parientes
constituidos en clan enemigo. Culminó este desastre en el viaje de
regreso, donde la ``verdadera amiga'' fue la comidilla de los pasajeros
por la asiduidad con que se colaba en el camarote del capitán y casi a
la vista de su hija. Con este drama hogareño agobiándole, dio principio
en París a la redacción de la Historia Física y Política, por la que
había recibido dinero adelantado y cuya entrega, tomo por tomo, estaba
sujeta a plazos estipulados en el contrato. Sólo jugaba a su favor la
amistad que le unía al nuevo padrino de la obra, el Ministro Manuel
Montt, que hacía ojos distraídos a la lentitud de su tarea y cursaba sin
objeción sus continuas demandas pecuniarias. Confió a él antes que a
nadie su decisión de divorciarse, y como ambos problemas llegaron a
entrecruzarse, le expresa en carta de enero de 1843: ``Esta remesa me
sería tanto más necesaria cuanto que si obtengo mi separación de la
señora Gay, el contrato que me obligaron a firmar sus padres me obliga a
darle no solamente la mitad de lo poco que he ganado sino aún de todo lo
que me ha tocado de los míos\ldots'' Cuando aquella situación se hizo
intolerable, llevó su demanda a los tribunales y ganó el pleito y con
éste la tuición de Thérése, quedando la madre condenada a no poder
visitarla sino dentro del internado y bajo vigilancia de la maestra.

Esta. niña de notable belleza e inteligencia era el bálsamo que aliviaba
los pesares y desengaños de don Claudio Gay cuando podía tenerla consigo
en su casa de la rue de Saint-Victor, atiborrada de libros, paquetes de
apuntes históricos, plantas, pájaros embalsamados, mapas y pruebas de
imprenta. Allí trabajaba el cientista, historiador y dibujante hasta
doce horas diarias, afanado en recuperar el tiempo perdido en
comparendos y litigios, y dirigiendo a la vez a más de cuarenta
diseñadores, iluminadores, grabadores, redactores y traductores al
castellano. Merced a este esfuerzo extraordinario pudo imprimir en el
invierno de 1844 el primer tomo de la obra y remitir a Santiago los
trescientos cincuenta ejemplares que tenía suscritos el Gobierno,
dejando los mil restantes para unos hipotéticos lectores futuros\ldots Esta
parte inicial alcanzaba hasta Pedro de Valdivia inclusive y el autor
ufanábase de haber aventajado a cualquier investigador conocido por la
amplitud de su documentación, que incluía las inéditas cartas del
conquistador a Carlos V. Su acuciosidad era tal que para informarse
sobre el Estrecho de Magallanes estuvo un mes y medio en Londres
copiando la cartografía y papeles oficiales de las últimas expediciones
británicas que habían pasado por allí. En los volúmenes de Historia
Natural iba a consignar ocho mil especies, la mayoría desconocidas o no
clasificadas hasta entonces. Tuvo siempre un alto concepto de la tarea
que estaba realizando, y así es como asegura a Montt, en septiembre del
45, que ningún país americano y muy pocos europeos podrían contar con
una obra semejante a la que él iba a ofrendar a Chile. Cuando esto
expresó, ya el ilustre Bello había saludado el tomo primero en El
Araucano; y no es de extrañar que lo hiciera, porque Gay estaba
precisamente fundando los estudios históricos nacionales, la escuela
historiográfica erudita y narrativa por la cual abogaba don Andrés desde
la rectoría de la Universidad.

Liberado de su infierno conyugal, no podían faltarle otros purgatorios
que estorbasen su trabajo y le agriasen el ánimo. De uno de estos se
encargó el Ministro de Instrucción Pública, el conservador ultramontano
Silvestre Ochagavía, que quiso dar guerra al divorciado y excomulgado
ordenando el desahucio de su contrato. Otro de sus gratuitos enemigos
fue el Ministro en París, el apolíneo y mundano Francisco Javier
Rosales, de quien se queja dirigiéndose a Montt: ``\ldots siempre atento a
criticarlo todo, no puede comprender que el Gobierno y las principales
familias de Chile tengan que dar algunas pruebas de estimación a una
persona de apariencia modesta e incapaz de ponerse bien la corbata; por
eso, con ese aire de superioridad y ese tono de grandeza que lo
caracterizan, me mira casi al igual que a un obrero, tratando de esta
manera de rebajarme y de hacerme pagar muy caro el insigne honor que el
Gobierno ha tenido la generosidad de dispensarme al decretar que mi
retrato fuese colocado en el Museo\ldots Me ha hablado de ello con ironía
irritante, capaz de herir al hombre más tranquilo y moderado\ldots Tiene
conocimiento de que todos los sabios de París manifiestan la mayor
consideración por mis trabajos\ldots, pero • no teme. censurar al Gobierno
por haberse suscrito a mi obra\ldots''

En 1850 tenía escrito el grueso de la sección narrativa de la Historia y
copiaba en Sevilla los últimos documentos, cuando una horrorosa
desgracia cayó sobre él al modo de un rayo celeste. Le avisaron desde el
internado que su hijita de quince años había muerto de una hemorragia
fulminante que apagó su vida en pocos minutos. ``El golpe lo anonadó.
Quedó enloquecido'', escriben Feliú Y Stuardo, comentaristas de su
Epistolario, el cual recoge la carta patética en que comunica la
catástrofe a Adrien de Jussieu, presidente de la Academia de Ciencias:
``\ldots Fue su maestra la que me dio esta espantosa noticia, y lo dejo a
Vd. pensando en la manera cómo la recibí. Yo que, después de tantas
contrariedades y penas, miraba a esta hermosa niña como el único objeto
que podía hacérmelas olvidar un poco\ldots'' El viaje de Sevilla a París en
diligencia le impidió llegar a tiempo para contemplar los restos de
Thérése, y no tuvo otro consuelo que oír sus últimas palabras
reproducidas por boca de quienes presenciaron su fin. Y nada pinta mejor
su infinita desolación que este desahogo dirigido a Montt: ``Presiento
todo mi porvenir roto y sin esperanza de cambio feliz''.

En este aplastante estado anímico volvió a sumergirse en el océano de
papeles del Archivo de Indias, secundado por una pareja de paleógrafos,
para proseguir su febril tarea de duplicación de documentos. Es dable
presumir que fue la entrega total a esta empresa, ya casi sin horas de
descanso, lo que le salvó de caer en la desesperación a que lo
arrastraba el recuerdo obsesivo de su hija. Un acontecimiento previsto
le trajo por otro lado un soplo de brisa tranquilizadora: la ascensión
de Montt a la Presidencia de la República, que implicaba la llegada a la
cumbre de su predilecto amigo chileno y la seguridad de que la Historia
Física y Política arribaría a su término sin nuevos tropiezos. Con
escaso intervalo fue objeto del honor que más podía complacerlo: la
elección como académico del Instituto de Francia con el rango de primer
botánico. Toda la década del 50 iba a serle propicia para dar a su ``gran
obra'', como la llamaba, un feliz remate. Se inició su amistad con el
joven discípulo Diego Barros Arana, que le facilitó sus documentos de la
época de O'Higgins y más tarde escribiría el estudio biográfico
\emph{Don Claudio Gay: su vida y sus obras}. En 1854 la Litografía de
Bécquet Fréres entregó la sección más costosa y llamativa de la
Historia: el célebre \emph{Atlas cartográfico, botánico y zoológico}
adornado con los paisajes dibujados por el autor y por Moritz Rugendas
que mostrarían al mundo el exótico Chile con su Valparaíso, su Mapocho y
su isla de Juan Fernández, con sus damas paseando en carreta, sus
malones araucanos, las cacerías de cóndores, los huasos de bonete
maulino y las carreras del Dieciocho. Fue el año en que tuvo el gusto de
ver reemplazado al Ministro Rosales por el almirante Blanco Encalada,
archivo viviente de recuerdos de la Independencia que le franqueó las
puertas de la Legación para platicar con él en largas veladas a las que
asistían las sombras augustas de los Carrera, Rodríguez, O'Higgins,
Henríquez, San Martín, Lord Cochrane y su amigo Portales\ldots Un
contertulio de carne y hueso, su colega Benjamín Vicuña Mackenna, se
sumó a las recepciones en que el Ministro reunía en sus salones de la
rue Lille, como la cosa más natural, a escritores del calibre de Merimée
y Lamartine y a las princesas Matilde y Carlota Bonaparte. Eran los días
del fastuoso apogeo de Napoleón III, en el París de los bulevares
panorámicos diseñados por el genio de Haussmann, el París de la
culminación colonialista y económica del Segundo Imperio, el de los
bailes feéricos en las Tullerías y los desfiles de cien mil soldados
siguiendo al Emperador por los Campos Elíseos\ldots Indiferente a este
boato deslumbrante, el hombrecito ''de hábitos sencillos y solitarios'',
el pelo sobrándole y el corbatín cayéndosele, trabajaba abstraído en su
libro de catorce mil páginas. Tanto abusó de su vista que en ocasiones
quedaba ciego y tenía que estarse días enteros sumido en la obscuridad y
dictando a su secretario de redacción.

Desde el primer tomo impreso aparece Gay ostentando en la portadilla,
entre su nombre y el escudo de Chile, estos títulos: ``ciudadano chileno''
y ``caballero de la Legión de Honor''; el primero colocado con tipografía
mayor que la del segundo\ldots Así dejó estampada su gratitud a la pequeña
nación que descubriera su talento y en cuyo suelo (carta a Domeyko)
decía haber pasado los diez mejores años de su vida.

Que fue chileno de corazón, lo demostró cien veces, como en esa misiva a
Vicuña Mackenna en que refiriéndose a los santiaguinos dispersos por
Europa le confiesa: ``Feliz si pudiera reunirlos a todos en una comida a
la chilena con charquicán, valdiviano, cazuela, empanadas, guachalomo y
\emph{tutti quanti}''. Su interés por la suerte de la segunda patria era
vivo y constante. Había sugerido al Presidente Montt la creación de un
servicio de Estadística Nacional, que pronto fue hecho realidad; y se
pasó la vida recomendando el cuidado y mejoramiento del Museo de
Historia Natural, del que era virtual fundador y al que. remitía piezas
y colecciones obtenidas por compra o canje, porque a su juicio este
establecimiento hacía honor al país y no se encontraría uno igual en
ninguna capital hispanoamericana. Por pura nostalgia vino a Chile en
1863, al cabo de veinte años de ausencia, se fue más encariñado todavía,
si cabe, con la pensión vitalicia que le concedió espontáneamente el
gobierno de Pérez. Al producirse el conflicto armado con España, envió
dinero para contribuir a las colectas patrióticas. Cuando conoció el
proyecto de transformación de Santiago, que daría su nombre a una nueva
calle, escribió al Intendente Vicuña Mackenna para agradecérselo e hizo
valer la ocasión para referirse al cerro Santa Lucía con el amor de uno
que hubiese nacido a su sombra.

Fue Vicuña Mackenna su confidente epistolar en los años postreros de la
empresa iniciada cuatro décadas atrás. En uno de esos desahogos
confiados al correo le informa de la desgracia sufrida en una estación
ferroviaria de París, donde un ladrón le robó la maleta en que llevaba
el manuscrito del último tomo de la Historia Física y Política de Chile,
que iba a ser entregado al traductor. ``Comprenda usted mi desesperación
al verme obligado a rehacer este trabajo\ldots Si fuera un tema
independiente renunciaría a. él, pues no me encontraría con fuerzas para
volver; a comenzarlo''.

Este volumen final, que hizo luchando con la ceguera y con la gota que
lo llevaría al sepulcro, alcanza hasta la abdicación de O'Higgins; y lo
mismo que el penúltimo, fue editado a sus expensas porque la materia
excedía la extensión convenida.

El escaso interés del público dejó el grueso del tiraje amontonado en un
desván de la casa del autor; precario resultado que podría atribuirse al
fallo no del todo favorable de sus críticos. La amistad no había
impedido a Vicuña Mackenna afirmar que ``las más brillantes indagaciones
históricas han quedado deslucidas en los indigestos volúmenes\ldots M. Gay
es un sabio profundo, pero no es un literato, ni un historiador, ni un
filósofo\ldots sólo la parte científica tiene mérito positivo\ldots'' Y a su
turno Barros Arana opinaría refiriéndose a los capítulos de la
Independencia: ``historia escrita sin animación y sin relieve, con un
cuidado particular de no emitir opiniones y juicios que pudieran
desagradar a los hombres que figuraron en esos sucesos''.

Se sabe que Gay reconoció la validez del comentario de Vicuña Mackenna,
puesto que ni se consideraba escritor ni había compuesto él solo la
sección narrativa, que parcialmente confió a redactores contratados para
poder entregar a tiempo este libro monumental. Lo que sí le hirió y le
sacó de sus casillas fue la actitud de Rodulfo Amando Philippi, director
del Museo (obra de Gay, como sabemos) al acusarle de incompetente por no
haber reunido todas las especies de insectos autóctonos de Chile\ldots En
respuesta le llamó alemán envidioso y pretencioso, y tras esta bofetada
explicó que cómo quería que hubiese catalogado el total de esos seres
minúsculos, la mayoría no más grandes que la cabeza de un alfiler,
cuando en Europa, al cabo de siglos de investigación, no pasaba un año
sin que se agregasen centenares de especies desconocidas\ldots

\chapter{LA TARDIA GLORIA DE PORTALES}

Como prueba de que los gobernantes trabajan para la posteridad ---tanto
demora su siembra en echar raíces y dar el fruto---, rara vez les es
dado a los contemporáneos tener la apreciación correcta de su obra. No
escapó a esta ley don Diego Portales, con haber sido el autor de dos
logros espectaculares y de efecto inmediato, que todos pudieron
contemplar y de los cuales nadie dejó de beneficiarse: el aplastamiento
de la anarquía, causa de la ruina del país, y la reorganización
administrativa y económica, que trajeron sobre los chilenos una
prosperidad desconocida hasta entonces. Cuesta hoy creerlo, pero todo el
homenaje oficial rendido al egregio ciudadano redújose a sus funerales,
sin duda magníficos, a un discurso del Ministro Tocornal y al acuerdo
del Senado de levantarle una estatua, que tardaría veintitrés años en
pasar del papel al bronce; Aparte esto, silencio, silencio hasta en los
labios del general Prieto, que en su siguiente Mensaje no mencionó al
hombre que lo había puesto en la Presidencia de la República, que lo
sostuvo en el poder e ilustró su gobierno con los hechos de su
prodigiosa gestión como Vicepresidente y titular de tres carteras
ministeriales.

Y como si una cosa se sincronizara con la otra, ¡qué escasos despojos
materiales dejó el que tanto bien legara a sus paisanos! Su acribillado
cadáver quedó en el camino del Barón ``completamente desnudo'', dice la
fama, porque desde la capa hasta las prendas íntimas le fueron robadas
por los asesinos. De su figura no existe imagen fiel, puesto que el
retrato por Domeniconi está basado en el apunte que tomara en Valparaíso
al rostro desangrado por las balas, cuchilladas y culatazos que lo
dejaron irreconocible. El cochecillo en que viajara aherrojado, como un
malhechor peligroso, desapareció. De la horrorosa tragedia sólo se salvó
como reliquia la barra de grillos que llevaba remachada desde su
apresamiento en Quillota. La ``fortuna'' del antiguo. comerciante había
quedado reducida. a. nueve mil pesos después de pagarse las deudas -que
la gravaban; y esto es cuanto dejó en herencia a los tres hijos
ilegítimos. Se sabe que el letrero de la casa de. comercio de Portales
Cea \& Compañía lo. guarda un particular que no se decide a entregarlo a
quien corresponde. Y en cuanto al retrato por Domeniconi y al escritorio
de Su Señoría, ``estaban'' en. La Moneda el último día de Salvador
Allende\ldots bajo cuya Administración el Liceo Diego Portales vio
reemplazado su nombre por el de Ernesto Che Guevara\ldots{}

Inútil buscar una huella del Ministro en los lugares en donde habitaba o
paraba: ni subsisten de las casas de la calle Catedral y la calle Rosas,
ni la de su quinta del Almendral en Valparaíso, todas ellas prolijamente
demolidas: Esperando encontrar algún vestigio en el campo el sacerdote
Edmundo Marzán visitó el fundito de Placilla, páramo de malezas y
piedras regado por don Diego con un canal que hizo labrar utilizando a
una cuadrilla de presidiarios. La casa patronal tenía detrás una viña y
por delante un jardín que lucía como adorno un ganso tallado en madera
de retamo. Pero el padre Marzán no halló ganso, jardín, viña ni casa,
porque el arado del agricultor Moisés Vargas había pasado por allí
arrasando hasta los cimientos; y sólo quedaba en un potrero, como
recuerdo piadoso, la cruz de palo de álamo que unos misioneros
redentoristas plantaron en memoria del antiguo propietario. (Publicado
en el Nro. 55 de la Revista Chilena de Historia y Geografía).

La glorificación del gran chileno vino a pasos cortos y lentos. Puede
decirse que en sus días sólo dos espíritus intuitivos vislumbraron su
grandeza, cuya luz cegadora encandiló a los coetáneos como si
contemplasen el sol de frente. Uno fue un extranjero recién llegado al
país, el cirujano Thomas S. Page, que vio los restos del mártir -en la
carreta en que lo trasladaron al Almendral y escribió en su Diario de
Viaje (todavía hoy inédito) que Portales era ``el padre de Chile'', un
hombre ``cuyo igual Chile nunca vio antes y temo que nunca volverá a
ver''.

La otra intuición es de una mujer, la poetisa Mercedes Marín del Solar,
que 'lloró el fin de su ídolo en el célebre Canto fúnebre de trecientos
y tantos versos:

\begin{verse}
\ldots¡Justicia eterna! ¿Cómo así permites\\
Que triunfe la maldad? ¿Así nos privas\\
Del tesoro precioso\\
En que libró su dicha y su reposo\\
La patria, y así tornas ilusoria\\
La esperanza halagüeña\\
Que un porvenir a Chile prometía\\
De poderío, de grandeza y gloria?\\
¿Dónde está. el genio que antes diera vida\\
A nuestra patria amada\ldots? ¡Oh caro nombre\\
Que en vano intenta pronunciar el labio\\
Mudo por la aflicción! Tu infeliz suerte,\\
Tu prematura, dolorosa muerte\\
No acierto a describir. ¡Ilustre sombra!\\
Perdona el extravío de mi canto\\
Empapado mil veces con mi llanto\ldots
\end{verse}

Un cuarto de siglo después comienzan los escritores a ocuparse del
prócer y mártir y corresponde a José Victorino Lastarria producir ese
monumento de pasión política, pequeñez aldeana e incompetencia
historiográfica en donde niega a Portales todo talento de organizador y
estadista y le llama tirano, burlador de las leyes, corruptor de la
justicia, reaccionario, abusador del poder, filibustero, fusilador de
inocentes y causante de la desgracia del Partido Liberal\ldots

Con dos años de intervalo, en 1863, don Benjamín Vicuña Mackenna
presentaba su biografía de casi novecientas páginas, en donde afirma que
Portales ``es la más alta figura de nuestra historia'', reconociendo su
patriotismo sublime, su desinterés, su originalidad y su don de
organizador, pero calificándole de mandón, déspota, atrabiliario,
enemigo de la democracia, ignorante, soberbio, cruel y despiadado. La
más notable de sus hazañas, la captura incruenta de la escuadrilla
perú-boliviana, con que salvó a Chile de ser invadido, era para don
Benjamín ``uno de los actos más odiosos que se registran en los anales de
nuestras repúblicas'', ``asalto aleve y nocturno'', comparable a ``los
rapaces expedientes de los piratas en los mares''. El libro -está
dedicado, precisamente, a Lastarria, en homenaje a su ``odio a la
tiranía\ldots'', pero éste consideró que el autor no había vapuleado
bastante al destructor del Partido Liberal y le escribió para decirle
que su lectura le había costado ``rabias, dolores de estómago, patadas,
reniegos y cuanto puede costar una cosa que desagrada''.

A este nivel se daba el análisis de la historia en la adolescencia
intelectual de Chile. Se trata de los mismos hombres (agréguese a Barros
Arana) que tronaron también contra Montt, continuador de Portales en la
empresa de hacer de la nuestra la primera nación organizada y solvente
de América del Sur. ¿Qué es lo que pretendían estos prohombres? ¿Que
aquella anarquía de siete años se hubiese perpetuado, que hubiese
seguido el país soportando un motín cada cuarto de hora, con las
provincias del sur debatiéndose en el hambre y los campos asolados por
las bandas de salteadores, con la capital convertida en basural y en
escenario de ochocientos asesinatos por año? ¿Que hubiese seguido el
país sin comercio y sin escuelas, las cajas del erario vacías, el
ejército impago y la Marina sin buques, para entregarse inerme al
aventurero boliviano que se anexó el Perú a sangre y fuego y tuvo a
Chile infestado de espías\ldots? Era la defensa del caos en nombre de un
liberalismo de salón literario y en pro de una democracia pipiolesca que
había sido la ruina de las independizadas colonias españolas.

Podemos disculpar a Lastarria, sabiendo lo que era, pero no a Vicuña
Mackenna, que fue un historiador profesional y el primero que tuvo
acceso a los archivos portalianos y a las cuatrocientas cartas suyas que
conservaba su testamentaría. Guardábanse estos preciosos papeles en
ciento setenta y tantos paquetes numerados y rotulados, algunos de ellos
provenientes de los baúles encontrados en el entretecho de uno de los
paraderos del Ministro, el de la calle Rosas número 28. De sus cartas
faltaban doscientas, que fueron rescatadas más tarde para el Epistolario
reunido por Ernesto de la Cruz; pero aún éstas no son todas, pues se
sabe que hay varias dirigidas al coronel Victorino Garrido y las guarda
empecinadamente un descendiente de éste, ``en una carpeta antigua de
cuero teñido de amarillo''; documentos que ni por dinero pudo obtener ni
leer el investigador Guillermo Feliú Cruz.

Fue menester esperar aún doce años, a partir de la publicación de Vicuña
Mackenna, para que don Ramón Sotomayor Valdés se hiciera presente con el
primer estudio apolítico y ecuánime que conocemos acerca de Portales.
Más que la estatua erigida en 1860 frente a la Moneda, este trabajo de
Sotomayor, editado en 1875, constituye el pedestal de la gloria que
tardíamente empezaba a despuntar para ``el severo guardián del orden
público'', ``el. impertérrito sacerdote de la justicia''.

``La labor de Portales'', dice, ``fue inmensa si se considera el carácter
de la época en que le cupo gobernar, los obstáculos de toda especie que
tuvo que vencer y el breve lapso en que figuró en el gobierno\ldots No sólo
se ha salvado de la mayor de las injurias del tiempo, que es el olvido,
sino que también ha llegado a simbolizar el patriotismo, el espíritu
público y el don de gobierno en el más alto grado''. ``Nada de lo que
interesa a la regeneración y prosperidad de un pueblo escapó a sus
miradas ni a sus propósitos''. ``Portales legó a la República toda su
organización''.

Así reivindicado, don Diego Portales se va agrandando en el concepto de
la ciudadanía pensante, y en las primeras décadas del siglo XX es
calificado de genio por nuestros más altos filósofos políticos
contemporáneos: don Francisco Antonio Encina y don Alberto Edwards. En
uno y otro échase de ver el esfuerzo por controlar su entusiasmo ante la
huella colosal que el estadista dejó impresa en nuestra historia.

``La transformación operada en el espacio de pocos meses bajo la poderosa
mano de este hombre de genio'', dice Edwards en \emph{La fronda
aristocrática}, ``fue tan radical y profunda que uno llega a imaginar,
cuando estudia los sucesos e ideas de ese tiempo, que después de 1830
está leyendo la historia de otro país, completamente distinto del
anterior, no sólo en la forma material de las instituciones y de los
acontecimientos, sino también en el alma misma de la sociedad''. Afirma
que Portales ``estaba dotado de un golpe de vista a la vez microscópico y
telescópico, capaz de percibir distintamente y al mismo tiempo las
líneas de conjunto de una construcción política y los detalles de cada
momento''. ``Era, a la vez, el más perfecto revolucionario y el tipo ideal
del hombre constructivo: por eso se le ha comparado con Julio César''.
``Esa sensación de estabilidad la experimentó el país desde el primer
momento, como por obra de milagro. Nadie se atrevió a combatir un poder
que no dudaba ni un solo instante de sí mismo''.

Edwards y Encina, justo es decirlo, tuvieron a su favor la perspectiva
del tiempo, que les permitió apreciar la solidez de aquel edificio
político sin parangón en América, de esa era portaliana que duró sesenta
años: el sistema de gobierno impersonal y antidemagógico, ``que no debe
estar vinculado a nadie, y mucho menos a sí mismo'', como lo definiera su
creador.

Si Edwards le dedicó unos capítulos del principal de nuestros estudios
políticos, Encina consagróle dos volúmenes de análisis profundo,
verdadera escultura de papel impreso que deja al Padre de Chile expuesto
a la veneración de sus compatriotas.

``El sentido común'', dice, ''más clarividente que la razón, tratándose de
políticos, porque está más próximo que ella de la corriente cósmica, ha
presentido siempre el genio de Portales. Desde el gañán incapaz de
conciencia vigilante hasta el político de instinto, no ha habido quién
no haya advertido que en Portales hay algo extraño, enigmático,
misterioso, que no se encuentra en los demás hombres inteligentes.
Confusamente, todos han creído divisar algo así como un adivino, un
mago, un loco superior, un apóstol de la realidad, un genio. El único
que no lo ha presentido es el intelectual del corte de Lastarria, el
razonador, al cual la ideología pinchó los ojos del espíritu y lo privó
de los más grandes dones que, desgajándose del cosmos, se posaron en
nuestro pequeño microcosmos: el instinto y la intuición, a los cuales
deben el hombre y la sociedad todo lo que son políticamente''. ``Hay en
Portales uno de los mayores genios\ldots su intuición es tan completa que
no tiene siquiera obscura conciencia de ella\ldots También fue un hombre
culto. La leyenda de su ignorancia sólo es un reflejo del prejuicio
español, que no concibe una mujer bonita ni un hombre rico inteligente,
ni un hombre inteligente ilustrado\ldots Pero ni su inteligencia, muy viva
y penetrante, ni su cultura cuentan para nada en la creación política de
su genio. César, como Portales, fue a la vez un intelecto poderoso y un
intuitivo; pero al paso que en aquél el intelecto es lo fundamental y el
intuitivo lo secundario, en el último el intuitivo es todo''. / ``Necio es
decir que el régimen portaliano fue inspirado por amor al despotismo, o
que nació de la ignorancia. Su cuna fue el odio al despotismo y una
concepción sociológica que se levanta a muchos codos de altura sobre las
elaboraciones teóricas. Fue la concepción genial de una forma de
gobierno que hizo posible la libertad, la honradez, la justicia, y la
civilización, en una palabra, en un pue- blo que carecía de las
capacidades necesarias para sostenerlas como expresión espontánea de su
genio y de su grado de desarrollo pero que reunía condiciones para
realizarlas si se las imponía de arriba abajo una poderosa sugestión
apoyada en grandes fuerzas espirituales y en una sanción severa''.

Después de Encina, ya es una llama sagrada la que arde a los pies de la
figura gigantesca, alimentada por las ofrendas de historiadores,
biógrafos, escritores y estadistas. Hablando de él en la Universidad de
Princeton, Estados Unidos, don Eduardo Cruz-Coke le llamó ``el hombre de
Estado más genial que haya producido la historia americana''. Jaime
Eyzaguirre afirma que enderezó el país porque ``por un azar de la
historia, el anhelo inconsciente de un pueblo y la intuición de un
hombre se habían encontrado''. Discrepando en absoluto de Encina, el
áspero Joaquín. Edwards Bello escribió sin embargo: ``Portales se
distingue por su desinterés; por su imparcialidad y sobre todo por su
salvaje independencia, la que suele traer casi siempre la soledad''. ``En
Yungay, Chile se convirtió a la manera portaliana en potencia del
Pacífico''. ``Portales es más que un hombre y un cúmulo de circunstancias:
él mismo creó las circunstancias y en esto fue un genio''. Como
completando esta idea, dijo Sergio Onofre Jarpa: ``Chile, después de
Yungay, empezó a tener imagen propia en América y en el mundo. Y los
chilenos definieron desde ese momento su estilo, su destino, su forma de
enfrentar los acontecimientos. - Empezaron a tener sentido de
nacionalidad''. / ``Portales tenía una concepción geográfica, expansiva y
dinámica de la política externa, que esperamos sea restablecida para
hacer de Chile lo que él proyectaba: una nación grande, próspera y
poderosa en el Pacífico Sur''.

Otro portaliano decidido, Hernán Díaz Arrieta, escribió que cada vez que
los chilenos nos sentimos en apuros, volvemos los ojos hacia el hombre
providencial de 1829. Quedó esto comprobado como nunca el 11 de
Septiembre de 1973, en ese pronunciamiento llevado a cabo en defensa de
los mismos principios restauradores por los que Portales luchó y murió,
y a partir de cuya fecha se rebautizó con su nombre el edificio de
gobierno.

Servicios en vida y servicios póstumos por los cuales Su Señoría
sacrificó bienestar y fortuna y sin cobrar siquiera en compensación sus
sueldos de trabajador del Estado.

\end{document}
